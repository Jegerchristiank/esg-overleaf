\section*{Abstract}
\addcontentsline{toc}{section}{Abstract}

\begin{otherlanguage}{english}
This report examines environmental, social, and governance (ESG) reporting in the healthcare sector with a focus on regulatory frameworks, organizational implementation, and economic relevance for small and medium-sized enterprises (SMEs). The goal is to assess how an ESG-as-a-Service concept and an MVP can translate requirements into data collection, traceability, and report outputs.

The study combines document analysis of CSRD, ESRS, the EU Taxonomy, and GRI with empirical sector sources on climate footprint, waste, and working conditions. It also includes a case-based pilot of the author's company and an artefact analysis of the MVP's data flow and functionality.

The analysis shows that ESG operates as an organizational standard that enhances legitimacy and comparability but also introduces a risk of decoupling between reporting and practice. ESG-as-a-Service can mitigate this risk by providing standardized workflows, validation, and audit trails. The value proposition depends on prioritizing material topics and embedding data practices in the organization.

The conclusion is that the MVP demonstrates practical feasibility for SMEs, but scaling requires gradual implementation, stronger integrations, and sustained focus on data quality. The report therefore recommends a phased reporting approach, starting with minimum modules and expanding as capacity and maturity grow.
\end{otherlanguage}
