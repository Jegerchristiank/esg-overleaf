\subsection{Organisatoriske og økonomiske implikationer}
\label{subsec:organisatoriske-og-oekonomiske-implikationer}

Implementering er i praksis en governance-opgave: hvem ejer data, hvem godkender dem, og hvordan sikres sporbarhed. Standarder og regulering etablerer forventninger til dokumentation, hvilket betyder, at SMV'er må definere ansvar for dataindsamling, validering og rapportering.\parencite{EU2022CSRD,EU2023ESRS} Det skaber behov for en intern funktion eller et eksternt serviceled, der kan oversætte krav til operative beslutninger.\parencite{RoevikTrenderTranslasjoner,TranslationTheoryKnowledgeTransfer}

Ressourcebehovet optræder på flere niveauer: tid til dataindsamling, kompetencer til fortolkning af standarder og tekniske ressourcer til at integrere data. Løsningen kan reducere noget af dette ved at tilbyde standardiserede arbejdsgange og auditspor, men kræver stadig organisatorisk forankring for at undgå dekobling mellem rapportering og praksis.\parencite{BrunssonWorldOfStandards}

Omkostningsstrukturen er typisk fronttung med udgifter til datakortlægning, integration og oplæring, mens den løbende drift domineres af datavedligehold, rapportering og kontrol. Return on investment afhænger af, om virksomheden kan reducere manuelle processer, mindske compliance-risici og anvende ESG-data i beslutninger. Manglende datakvalitet øger revisionsrisikoen og kan gøre rapporteringen mindre brugbar, hvilket svækker både governance og den forventede værdi.

Økonomisk set handler implikationerne om forholdet mellem omkostninger og potentiel værdi. Metastudier peger på, at ESG-arbejde ofte er forbundet med neutral eller positiv finansiel performance, men effekten er betinget af fokus på materielle temaer.\parencite{Friede2015Meta,Khan2016Materiality} For SMV'er betyder det, at ressourcer bør prioriteres til data og indikatorer, der er direkte relevante for sundhedssektorens risici og interessentkrav.

I analysen betyder dette, at servicekonceptet skal vurderes på sin evne til at reducere faste compliance-omkostninger og skabe beslutningsrelevante data, snarere end blot at levere rapporter. Governance-implikationen er, at organisationen må etablere en stabil rapporteringsrutine, hvor dataindsamling bliver en integreret del af driften.
