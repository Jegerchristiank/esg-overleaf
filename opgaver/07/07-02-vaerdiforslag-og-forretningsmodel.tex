\subsection{Værdiforslag og forretningsmodel}
\label{subsec:vaerdiforslag-og-forretningsmodel}

Værdiforslaget i ESG-as-a-Service kan forstås gennem en kombination af stakeholder- og shared value-perspektiver. Stakeholder-tilgangen betoner, at legitimitet og ansvarlighed skaber værdi for flere interessenter end aktionærer, mens shared value argumenterer for, at samfundsmæssige udfordringer kan omsættes til forretningsmuligheder.\parencite{Freeman1984Stakeholder,PorterKramer2011CSV} Dette støttes af pensum om ESG, som understreger behovet for at integrere ESG i kerneforretningen og bruge data som beslutningsgrundlag, ikke kun som compliance.\parencite{ESGBook}

MVP'ens værdiforslag består i at reducere transaktionsomkostninger ved dataindsamling, standardisere input og skabe sporbar dokumentation, der kan anvendes i dialog med banker, kunder og myndigheder (bilag \ref{app:teknisk-dokumentation}). For SMV'er er dette centralt, fordi de ofte mangler interne ESG-specialister og har begrænsede ressourcer til komplekse rapporteringsprocesser.\parencite{Virksomhedsguiden2025SMVDefinition}

Forretningsmodellen hviler på tre elementer: et digitalt produkt (dataindsamling og beregning), en compliance-service (validering og dokumentation) og et rapportoutput (PDF og standardiserede formater). Den reducerer måle- og rapporteringsfriktion, hvilket gør ESG til en operationaliserbar praksis snarere end et abstrakt krav. Det understøtter samtidig et strategisk narrativ om, at ESG-data kan bruges til risikostyring og værdiskabelse, i tråd med evidensen for positiv eller neutral sammenhæng mellem ESG og finansiel performance.\parencite{Eccles2014Impact,Friede2015Meta}

Alternativer til ESG-as-a-Service er typisk konsulentprojekter, interne ESG-teams eller generiske rapporteringsværktøjer. Konsulenter kan levere specialviden, men skaber ofte højere transaktionsomkostninger og mindre løbende standardisering. Interne løsninger giver kontrol, men kræver kapacitet og datafaglighed, som mange SMV'er ikke har. Generiske værktøjer kan være billige, men matcher sjældent sektorspecifikke databehov.

Samtidig peger \textcite{Khan2016Materiality} på, at ESG kun skaber finansiel værdi, når indsatsen er rettet mod materielle temaer. Det betyder, at ESG-as-a-Service skal prioritere data, der er relevante for sundhedssektorens kernerisici (energi, affald, arbejdsmiljø) og ikke blot dække alle standarder mekanisk. Værdiforslaget afhænger derfor af evnen til at koble standardkrav til sektorrelevante indikatorer og til at omsætte dem til beslutningsrelevante outputs.

En bæredygtig forretningsmodel forudsætter, at kunderne oplever et klart compliance- og risikobenefit, at data kan indsamles med begrænset friktion, og at onboarding og support kan skaleres uden at øge omkostningsbasen proportionalt.

\subsubsection{Fra governance til forretning: hvor grænsen går}
\label{subsec:fra-governance-til-forretning}

ESG-as-a-Service bevæger sig fra governance til forretning, når data ikke længere primært fungerer som dokumentation, men som aktivt beslutningsgrundlag for drift, investeringer og markedspositionering. I governance-laget er værdien knyttet til compliance, sporbarhed og risikoreduktion; i forretningslaget opstår værdi, når ESG-data bruges til at optimere omkostninger (fx energi og affald), differentiere produkter eller understøtte pris- og segmentstrategier. Det skift kan operationaliseres ved, at output går fra minimumsrapporter til KPI'er, der indgår i ledelsesrapportering og kundedialog. Grænsen mellem governance og forretning ligger derfor ikke i selve rapportformatet, men i om data anvendes aktivt til strategiske beslutninger og kommercielle prioriteringer.\parencite{PorterKramer2011CSV,ESGBook}
