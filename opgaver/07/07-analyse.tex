\section{Analyse}
\label{sec:analyse}

\begin{quote}
Analysen tester, om ESG-rapportering i praksis bliver styring eller blot dokumentation.
\end{quote}

Analysen anvender den teoretiske ramme fra \ref{sec:teoretisk-ramme} og metodevalget i \ref{sec:metode} til at fortolke empirien og MVP'ens rolle i ESG-rapportering. Fokus er på, hvordan standarder og governance omsættes til praksis, hvordan værdiforslaget kan begrundes økonomisk, og hvilke organisatoriske konsekvenser servicekonceptet skaber for SMV'er i sundhedssektoren.

Analysen tester især, om standardisering via ESG-as-a-Service reducerer risikoen for dekobling, og om rapporteringen bliver beslutningsrelevant frem for blot minimum compliance. Det gør det muligt at vurdere, hvor løftet om governance faktisk indfries i praksis.

Afsnit \ref{subsec:esg-som-organisationsstandard} analyserer ESG som organisationsstandard med udgangspunkt i pensum om standardisering og translasjon. Afsnit \ref{subsec:vaerdiforslag-og-forretningsmodel} vurderer værdiforslag og forretningsmodel i relation til stakeholder-/shareholder-logikker og performance-evidens. Prisfastsættelse og segmentering behandles i \ref{subsec:prisfastsaettelse-og-kundesegmentering}, mens \ref{subsec:organisatoriske-og-oekonomiske-implikationer} diskuterer governance og ressourcekrav. \ref{subsec:implementering-i-sundhedssektoren} samler barrierer og incitamenter for implementering i sektoren.
