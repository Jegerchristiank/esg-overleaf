\subsection{ESG som organisationsstandard}
\label{subsec:esg-som-organisationsstandard}

\textcite{BrunssonWorldOfStandards} beskriver standarder som regler, der skaber ensartethed og legitimitet, men som ikke automatisk følges i praksis. I sundhedssektoren fungerer ESG-krav som en sådan standardiseringsramme: de definerer, hvad der tæller som gyldig dokumentation, og hvilke indikatorer der skal måles. Samtidig skaber sektorens kompleksitet og datafragmentering en risiko for dekobling mellem rapportering og faktisk praksis.\parencite{BrunssonWorldOfStandards,HCWH2019ClimateFootprint,WHO2024HealthCareWaste}

Pensum om translasjon understreger, at ideer og krav altid oversættes til lokale praksisser og dermed forandres i processen.\parencite{RoevikTrenderTranslasjoner,TranslationTheoryKnowledgeTransfer} Det betyder, at ESG-rapportering i SMV'er ikke bliver en direkte implementering af ESRS/GRI, men en lokal tilpasning, hvor ressourcer, kompetencer og datasystemer afgør, hvad der reelt kan rapporteres. Case-materialet viser en tilsvarende prioritering af energi, affald og sociale KPI'er som operationel kerne frem for fuld dækning af alle standardkrav (bilag \ref{app:teknisk-dokumentation}).

Sektorens empiriske profil understøtter behovet for standardisering: klimaaftrykets tyngde i scope~3 og de sociale risici i arbejdsmiljø kræver konsistente datakilder og klar dokumentation.\parencite{HCWH2019ClimateFootprint,WHO2021HealthWorkerDeaths} Samtidig viser litteraturen, at ESG-rammer overlapper med bredere sundheds- og miljøhensyn, hvilket gør standardisering til en koordinationsmekanisme på tværs af faglige domæner.\parencite{Vegro2025OneHealth}

Analytisk betyder det, at ESG som organisationsstandard skaber et pres for ensartethed, men også en systemisk risiko for dekobling, fordi rapportering kan blive et mål i sig selv. ESG-as-a-Service (jf. \ref{subsec:esg-as-a-service-som-servicekoncept}) kan fungere som den infrastrukturelle oversætter, der binder standarder til konkrete datafelter og auditspor, men hvis løsningen primært optimerer rapporteringen frem for praksis, institutionaliserer den netop den dekobling, standarderne skulle modvirke. Derfor behandles dekobling som en hovedrisiko i vurderingen af ESG-as-a-Service.
