\subsection{Implementering i sundhedssektoren}
\label{subsec:implementering-i-sundhedssektoren}

Implementering i sundhedssektoren bremses ikke primært af vilje, men af datafragmentering, kapacitetsmangel og kulturel kompleksitet. Sektoren er afhængig af mange datakilder (energi, affald, arbejdsmiljø), og data er ofte spredt på tværs af systemer og leverandører.\parencite{HCWH2019ClimateFootprint,WHO2024HealthCareWaste} Det gør standardiseret rapportering vanskelig uden en teknisk infrastruktur til indsamling og validering.

Samtidig er der tydelige incitamenter. Regulering og forsyningskædekrav driver rapporteringsbehov, og litteraturen peger på, at ESG og digital transformation kan styrke sektorens bæredygtighed og robusthed.\parencite{Sepetis2024Healthcare,Bosco2024ESGHealth} One Health-perspektivet underbygger, at sundhedssektoren har et særligt ansvar for at koble miljø og sociale hensyn, hvilket forstærker behovet for systematisk implementering.\parencite{Vegro2025OneHealth}

Modenhed og forandringsparathed varierer på tværs af delsektorer. Mindre klinikker har ofte lav datakapacitet og begrænsede ressourcer, mens hospitaler og medtech-leverandører typisk har stærkere governance-strukturer og større compliancepres. Det betyder, at implementeringen må tilpasses organisatorisk modenhed og datainfrastruktur.

En trinvist implementeringsstrategi er derfor mest realistisk for SMV'er: (1) kortlægning af datakilder og etablering af minimumsmoduler (energi, affald, sociale KPI'er), (2) standardiseret indsamling med sporbarhed, og (3) gradvis udvidelse til mere komplekse datakrav og integrationer. Case-materialet peger på samme trappelogik (bilag \ref{app:teknisk-dokumentation}). Dette harmonerer med pensum om translasjon, hvor oversættelse og kontekstualisering er afgørende for, at nye standarder får organisatorisk gennemslag.\parencite{RoevikTrenderTranslasjoner,TranslationTheoryKnowledgeTransfer}

Implementeringsmodenhed afhænger derfor ikke kun af teknologi, men af organisatorisk læring, governance og prioritering. Spændingen opstår, når kravet om standardiseret dokumentation møder hverdagens driftslogik. Servicekonceptet kan reducere teknisk kompleksitet, men organisatorisk forankring er stadig nødvendig for at undgå symbolsk efterlevelse. Prioriteringen bør være at sikre datagrundlag og minimumsmoduler først, derefter standardiseret rapportering og til sidst integrationer, der muliggør løbende forbedringer.
