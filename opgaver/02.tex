\section{Indledning}
\label{sec:indledning}

ESG-rapportering (Environmental, Social og Governance) i sundhedssektoren er i stigende grad et \textit{styrings- og legitimitetskrav} og ikke bare en frivillig kommunikationsopgave. Sektoren leverer kritiske ydelser, har komplekse forsyningskæder og høje dokumentationskrav, hvilket gør kravet om \textit{sporbarhed} og konsistent rapportering særligt krævende for mange aktører.
Indledningen gør tre ting: den placerer ESG-rapportering som styringskrav, afgrænser de centrale EU-rammer og peger på koblingen mellem jura, organisation og MVP. Spændingsfeltet opstår mellem dokumentationskrav og den kapacitet særligt SMV'er realistisk kan mobilisere.

Analysen ser på, hvordan Corporate Sustainability Reporting Directive (CSRD), European Sustainability Reporting Standards (ESRS), EU-taksonomien og Global Reporting Initiative (GRI) oversætter politiske målsætninger til konkrete rapporteringskrav, og hvordan disse krav kan operationaliseres i praksis for små og mellemstore virksomheder (SMV'er) i sundhedssektoren. Tilgangen kombinerer policy- og organisationsperspektiver med et \textit{ESG-as-a-Service}-koncept og en MVP-baseret case, der viser dataindsamling, sporbarhed og rapportoutput.

CSRD forankrer bæredygtighedsrapportering i den formelle ledelsesberetning og kræver digital tagging, så rapporteringen flytter sig fra narrativ til auditerbar data.\parenciteshort{EU2022CSRD} ESRS konkretiserer kravene gennem datapunkter og \textit{dobbelt væsentlighed} og gør datagrundlag og \textit{materialitetsproces} til en styringsopgave snarere end en kommunikationsøvelse.\parenciteshort{EU2023ESRS} Global Reporting Initiative (GRI) fungerer som et globalt frivilligt supplement uden bindende karakter, hvilket understreger behovet for en praktisk oversættelse i SMV-konteksten.\parenciteshort{GRI2021}

\textbf{Den gennemgående argumentation er, at regulative krav skaber en kløft mellem standarder og operationel datapraksis i SMV'er, og at ESG-as-a-Service kan fungere som en \textit{oversættelsesmekanisme}, der reducerer denne kløft.} Krav og kontekst danner udgangspunktet, mens teori, metode og MVP konkretiserer oversættelsen til praksis og gør det muligt at vurdere, hvornår rapportering bliver beslutningsrelevant frem for \textit{symbolsk compliance}.
\subsection{Baggrund og motivation}
\label{subsec:baggrund-og-motivation}

\textcite{HCWH2019ClimateFootprint} dokumenterer, at sundhedssektoren globalt står for ca. 4,4~\% af de samlede netto-udledninger, svarende til omtrent 2~Gt CO$_2$e. Mere end halvdelen af aftrykket kommer fra energiforbrug, og udledningerne fordeler sig omtrent som 17~\% scope~1, 12~\% scope~2 og 71~\% scope~3. Det viser ret tydeligt sektorens afhængighed af indirekte udledninger i forsyningskæder og drift og peger på, at leverandør- og indkøbsdata bliver afgørende for rapporteringskvaliteten, selv i mindre organisationer.

Affaldsprofilen er en anden central brik. \textcite{WHO2024HealthCareWaste} klassificerer omtrent 85~\% af sundhedssektorens affald som ikke-farligt, mens ca. 15~\% er farligt (infektiøst, kemisk eller radioaktivt). Opdelingen indikerer, at affaldsdata må være fraktionsspecifikke for at kunne dokumentere \textit{compliance} og risikostyring.

Den sociale dimension er lige så vigtig. \textcite{WHO2021HealthWorkerDeaths} registrerer mindst 115.500 dødsfald blandt sundheds- og omsorgspersonale over 18 måneder under COVID-19. Det peger på arbejdsvilkår og sikkerhed som en integreret del af ESG-arbejdet i sektoren og understreger behovet for sociale indikatorer, der kan dokumentere både forebyggelse og organisatorisk ansvarlighed.

Samtidig skærper de regulatoriske rammer kravene til rapportering. \textcite{EU2022CSRD} og \textcite{EU2023ESRS} indfører standardiserede krav og digital rapportering; EU-taksonomien kræver dokumentation for overensstemmelse, jf. \textcite{EU2020Taxonomy}, og \textcite{GRI2021} fungerer som et udbredt frivilligt rammeværk for sammenlignelighed.

Kombinationen af et dokumenteret klima- og affaldsaftryk, sociale risici og skærpet regulering skaber et pres for systematisk og sporbar ESG-rapportering. For SMV'er i sundhedssektoren betyder det, at rapporteringsopgaven skal løses under ressourcebegrænsninger, hvilket motiverer en undersøgelse af forenklede processer og ESG-as-a-Service som et operationaliserende svar. Dette leder frem til problemformuleringen.

Baggrunden afgrænser den centrale kløft mellem krav og kapacitet og skærper behovet for en præcis problemformulering og konkrete forskningsspørgsmål.
\subsection{Formål og forskningsspørgsmål}
\label{subsec:formaal-og-forskningssporgsmaal}

Formålet er at analysere ESG-rapportering i sundhedssektoren med særligt fokus på regulatoriske krav, organisatorisk implementering og økonomisk relevans for SMV'er. Undersøgelsen belyser samtidig, hvordan et ESG-as-a-Service-koncept og en MVP kan omsætte krav til konkret dataindsamling, sporbarhed og rapportoutput.

Følgende forskningsspørgsmål besvares:
\begin{enumerate}
  \item Hvordan påvirker CSRD, ESRS, EU-taksonomien og GRI kravene til ESG-rapportering i sundhedssektoren, især for SMV'er?
  \item Hvilket værdiforslag skaber ESG-as-a-Service i forhold til \textit{standardisering}, \textit{governance} og \textit{compliance}?
  \item Hvordan kan en MVP omsætte regulatoriske krav til dataindsamling, \textit{sporbarhed} og rapportoutput?
\end{enumerate}

Spørgsmålene kobler regulative rammer, værdiforslag og teknisk operationalisering, så analysen kan vurdere beslutningsrelevans og governance i en SMV-kontekst.
\subsection{Problemformulering}
\label{subsec:problemformulering}

Analysen tager udgangspunkt i ESG som samlebetegnelse for miljø-, sociale og governance-forhold, der forventes dokumenteret i virksomheders rapportering. I \textcite{Virksomhedsguiden2025SMVDefinition} fastlægges SMV'er som ikke-børsnoterede virksomheder, der falder under EU's SMV-definition, og som typisk har begrænsede ressourcer til compliance og datastyring. ESG-as-a-Service anvendes her som betegnelse for en kombineret software- og serviceleverance, der oversætter standardkrav til operationelle datapunkter, kontroller og rapportoutput.

Reguleringen skaber et pres for ensartet og sporbar rapportering, men i sundhedssektoren er data ofte fragmenterede på tværs af systemer, leverandører og organisatoriske enheder. Det skaber en kløft mellem normative standarder og den praktiske mulighed for at levere valide og sammenlignelige ESG-data.

\textbf{Problemformulering:} Hvordan kan ESG-as-a-Service, understøttet af en MVP, reducere kløften mellem regulatoriske krav og operationel datapraksis for SMV'er i sundhedssektoren, så rapportering bliver både sporbar og beslutningsrelevant?

Problemformuleringen samler analysens centrale test: om oversættelsen fra krav til praksis skaber sporbar og beslutningsrelevant rapportering frem for symbolsk compliance.
\subsection{Afgrænsning}
\label{subsec:afgraensning}

Afgrænsningen omfatter følgende punkter:
\begin{itemize}
  \item Geografisk og regulatorisk fokus er EU/Danmark med CSRD, ESRS, EU-taksonomien og GRI som centrale rammer.
  \item Sektorfokus er sundhedssektoren, primært private klinikker, mindre hospitaler og medtech-leverandører.
  \item Virksomhedsfokus er SMV'er; store børsnoterede virksomheder behandles kun som kontekst.
  \item Emnefokus er ESG-rapportering og compliance; fulde LCA-modeller og dyb klima-science er ikke omfattet.
  \item Empirien bygger primært på sekundærdata samt egen case og softwaremateriale (bilag \ref{app:teknisk-dokumentation}).
  \item Ekstern assurance, fuld finansiel værdiansættelse og brancheregnskaber er uden for scope.
  \item Tidsplaner for CSRD/ESRS behandles som usikre; økonomiske vurderinger er betinget af indfasningen.
\end{itemize}

Afgrænsningen sikrer, at den røde tråd fastholdes omkring oversættelsen fra regulatoriske krav til praktisk datapraksis i SMV-konteksten.
