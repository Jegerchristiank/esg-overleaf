\section{Indledning}
\label{sec:indledning}

ESG-rapportering (Environmental, Social og Governance) i sundhedssektoren er i stigende grad et styrings- og legitimitetskrav snarere end en frivillig kommunikationsopgave. Sektoren kombinerer kritiske ydelser, komplekse forsyningskæder og høje krav til dokumentation, hvilket gør sporbar og konsistent rapportering særligt udfordrende for mange aktører.
Det skaber et spændingsfelt mellem dokumentationskrav og den kapacitet, særligt SMV'er realistisk kan mobilisere.

Analysen undersøger, hvordan Corporate Sustainability Reporting Directive (CSRD), European Sustainability Reporting Standards (ESRS), EU-taksonomien og Global Reporting Initiative (GRI) omsætter politiske målsætninger til konkrete rapporteringskrav, og hvordan disse krav kan operationaliseres i praksis for små og mellemstore virksomheder (SMV'er) i sundhedssektoren. Tilgangen kombinerer policy- og organisationsperspektiver med et ESG-as-a-Service-koncept og en MVP-baseret case, der viser dataindsamling, sporbarhed og rapportoutput.

CSRD integrerer bæredygtighedsrapportering i den formelle ledelsesberetning og kræver digital tagging, hvilket flytter rapporteringen fra narrativ til auditerbar data.\parencite{EU2022CSRD} ESRS konkretiserer kravene gennem datapunkter og dobbelt væsentlighed og gør datagrundlag og materialitetsproces til en styringsopgave snarere end en kommunikationsøvelse.\parencite{EU2023ESRS} GRI fungerer samtidig som et globalt frivilligt supplement, men uden bindende karakter, hvilket understreger behovet for en praktisk oversættelse i SMV-konteksten.\parencite{GRI2021}

Den gennemgående argumentation er, at regulative krav skaber en kløft mellem standarder og operationel datapraksis i SMV'er, og at ESG-as-a-Service kan fungere som en oversættelsesmekanisme, der reducerer denne kløft. Krav og kontekst danner udgangspunktet, mens teori, metode og MVP konkretiserer oversættelsen til praksis og gør det muligt at vurdere, hvornår rapportering bliver beslutningsrelevant frem for symbolsk.
\subsection{Baggrund og motivation}
\label{subsec:baggrund-og-motivation}

Globalt bidrager sundhedssektoren med ca. 4,4~\% af de samlede netto-udledninger, svarende til omtrent 2~Gt CO$_2$e. Mere end halvdelen af klimaaftrykket kommer fra energiforbrug, og udledningerne fordeler sig omtrent som 17~\% scope~1, 12~\% scope~2 og 71~\% scope~3, hvilket understreger sektorens afhængighed af indirekte udledninger i forsyningskæder og drift.\parencite{HCWH2019ClimateFootprint} Fordelingen peger på, at leverandør- og indkøbsdata bliver afgørende for rapporteringskvaliteten, selv i mindre organisationer.

Affaldsprofilen er en anden central dimension. Omtrent 85~\% af sundhedssektorens affald klassificeres som ikke-farligt, mens ca. 15~\% er farligt (infektiøst, kemisk eller radioaktivt), hvilket skaber særlige krav til bortskaffelse og dokumentation.\parencite{WHO2024HealthCareWaste} WHO's opdeling indikerer, at affaldsdata må være fraktionsspecifikke for at kunne dokumentere compliance og risikostyring.

Den sociale dimension er ligeledes betydningsfuld. Under COVID-19 estimerer WHO mindst 115.500 dødsfald blandt sundheds- og omsorgspersonale over 18 måneder, hvilket peger på arbejdsvilkår og sikkerhed som en integreret del af ESG-arbejdet i sektoren.\parencite{WHO2021HealthWorkerDeaths} Det understreger behovet for sociale indikatorer, der kan dokumentere både forebyggelse og organisatorisk ansvarlighed.

Samtidig skærper de regulatoriske rammer kravene til rapportering. CSRD og ESRS indfører standardiserede krav og digital rapportering, EU-taksonomien kræver dokumentation for overensstemmelse med taksonomien, og GRI fungerer som et udbredt frivilligt rammeværk for sammenlignelighed.\parencite{EU2022CSRD,EU2023ESRS,EU2020Taxonomy,GRI2021}

Kombinationen af et dokumenteret klima- og affaldsaftryk, sociale risici og skærpet regulering skaber et pres for systematisk og sporbar ESG-rapportering. For SMV'er i sundhedssektoren betyder det, at rapporteringsopgaven skal løses under ressourcebegrænsninger, hvilket motiverer en undersøgelse af forenklede processer og ESG-as-a-Service som et operationaliserende svar.

Baggrunden afgrænser den centrale kløft mellem krav og kapacitet og skærper behovet for en præcis problemformulering og konkrete forskningsspørgsmål.
\subsection{Formål og forskningsspørgsmål}
\label{subsec:formaal-og-forskningssporgsmaal}

Formålet er at analysere ESG-rapportering i sundhedssektoren med særligt fokus på regulatoriske krav, organisatorisk implementering og økonomisk relevans for SMV'er. Undersøgelsen belyser samtidig, hvordan et ESG-as-a-Service-koncept og en MVP kan omsætte krav til konkret dataindsamling, sporbarhed og rapportoutput.

Følgende forskningsspørgsmål besvares:
\begin{enumerate}
  \item Hvordan påvirker CSRD, ESRS, EU-taksonomien og GRI kravene til ESG-rapportering i sundhedssektoren, især for SMV'er?
  \item Hvilket værdiforslag skaber ESG-as-a-Service i forhold til standardisering, governance og compliance?
  \item Hvordan kan en MVP omsætte regulatoriske krav til dataindsamling, sporbarhed og rapportoutput?
\end{enumerate}

Spørgsmålene kobler regulative rammer, værdiforslag og teknisk operationalisering, så analysen kan vurdere beslutningsrelevans og governance i en SMV-kontekst.
\subsection{Problemformulering}
\label{subsec:problemformulering}

Analysen tager udgangspunkt i ESG som samlebetegnelse for miljø-, sociale og governance-forhold, der forventes dokumenteret i virksomheders rapportering. SMV'er forstås som ikke-børsnoterede virksomheder, der falder under EU's SMV-definition, og som typisk har begrænsede ressourcer til compliance og datastyring.\parencite{Virksomhedsguiden2025SMVDefinition} ESG-as-a-Service anvendes her som betegnelse for en kombineret software- og serviceleverance, der oversætter standardkrav til operationelle datapunkter, kontroller og rapportoutput.

Reguleringen skaber et pres for ensartet og sporbar rapportering, men i sundhedssektoren er data ofte fragmenterede på tværs af systemer, leverandører og organisatoriske enheder. Det skaber en kløft mellem normative standarder og den praktiske mulighed for at levere valide og sammenlignelige ESG-data.

\textbf{Problemformulering:} Hvordan kan ESG-as-a-Service, understøttet af en MVP, reducere kløften mellem regulatoriske krav og operationel datapraksis for SMV'er i sundhedssektoren, så rapportering bliver både sporbar og beslutningsrelevant?

Problemformuleringen samler analysens centrale test: om oversættelsen fra krav til praksis skaber sporbar og beslutningsrelevant rapportering frem for symbolsk compliance.
\subsection{Afgrænsning}
\label{subsec:afgraensning}

Afgrænsningen omfatter følgende punkter:
\begin{itemize}
  \item Geografisk og regulatorisk fokus er EU/Danmark med CSRD, ESRS, EU-taksonomien og GRI som centrale rammer.
  \item Sektorfokus er sundhedssektoren, primært private klinikker, mindre hospitaler og medtech-leverandører.
  \item Virksomhedsfokus er SMV'er; store børsnoterede virksomheder behandles kun som kontekst.
  \item Emnefokus er ESG-rapportering og compliance; fulde LCA-modeller og dyb klima-science er ikke omfattet.
  \item Empirien bygger primært på sekundærdata samt egen case og softwaremateriale (bilag \ref{app:teknisk-dokumentation}).
  \item Ekstern assurance, fuld finansiel værdiansættelse og brancheregnskaber er uden for scope.
\end{itemize}

Afgrænsningen sikrer, at den røde tråd fastholdes omkring oversættelsen fra regulatoriske krav til praktisk datapraksis i SMV-konteksten.
