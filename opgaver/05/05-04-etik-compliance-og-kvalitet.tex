\subsection{Etik, compliance og kvalitet}
\label{subsec:etik-compliance-og-kvalitet}

Etik og compliance vurderes med udgangspunkt i EU's databeskyttelsesforordning (GDPR) og generelle krav til behandling af personoplysninger i en ESG-kontekst.\parencite{EU2016GDPR} Hvor persondata indgår (fx medarbejderdata), anvendes principper om dataminimering, formålsbegrænsning og adgangskontrol. For case og MVP betyder det, at datatyper holdes på et nødvendigt niveau, og at adgang til systemet er rollebaseret.

Kvalitet og sporbarhed understøttes af versionshistorik og auditspor, der dokumenterer ændringer i data og beregninger (bilag \ref{app:teknisk-dokumentation}). Det gør det muligt at efterprøve resultater og identificere antagelser og datakilder bag rapporterede tal. Transparens om antagelser indgår som en eksplicit del af den faglige redegørelse.

Der er en potentiel interessekonflikt, idet casen bygger på egen virksomhed. Dette håndteres ved at dokumentere metodevalg og antagelser eksplicit og ved at afgrænse analysens generaliserbarhed.

De resterende metodiske begrænsninger samles i \ref{subsec:begraensninger}.
