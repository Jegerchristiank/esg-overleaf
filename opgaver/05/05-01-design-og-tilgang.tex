\subsection{Design og tilgang}
\label{subsec:design-og-tilgang}

Undersøgelsen anvender et kvalitativt, eksplorativt design. Formålet er at forklare, hvordan ESG-krav omsættes til organisatoriske processer og tekniske datastrukturer i sundhedssektoren, snarere end at teste kausale effekter. Designet kombinerer dokumentanalyse af regulering med empirisk sektorkontekst og en casebaseret pilot.

Tilgangen er abduktiv: den teoretiske ramme bruges til at strukturere fortolkningen af case og MVP, mens empiriske fund justerer og nuancerer analysen. Der arbejdes i tre analytiske spor: (1) regulatorisk kortlægning og kravfortolkning, (2) empirisk kontekstualisering af sektorens ESG-udfordringer, og (3) artefaktanalyse af MVP'en som operationalisering af krav til data, sporbarhed og rapportoutput.

Triangulering opnås ved at sammenholde regulatoriske dokumenter, empiriske sektorkilder og interne case- og softwarekilder. Det giver et samlet billede af, hvordan standarder, governance og praktiske constraints former ESG-rapportering for SMV'er i sundhedssektoren.
