\subsection{Begrænsninger}
\label{subsec:begraensninger}

Metodens primære begrænsninger er:
\begin{itemize}
  \item Single-case design med egenproduceret materiale, hvilket begrænser generaliserbarhed.
  \item MVP'en analyseres som artefakt og er ikke testet på tværs af flere organisationer.
  \item Datagrundlaget er primært sekundært og aggregeret, hvilket reducerer indsigt i lokale variationer.
  \item Regulering og standarder er i bevægelse, så krav kan ændre sig i projektperioden.
  \item Tidsmæssig afgrænsning betyder, at der ikke er gennemført longitudinelle analyser af effekt og adfærdsændring.
\end{itemize}

Begrænsningerne indebærer, at konklusionerne primært er analytiske og konceptuelle og bør valideres i fremtidige empiriske studier.

Med disse metodiske rammer præsenteres den konkrete MVP og dens operationalisering i \ref{sec:software-og-mvp}.
