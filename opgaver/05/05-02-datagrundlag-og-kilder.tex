\subsection{Datagrundlag og kilder}
\label{subsec:datagrundlag-og-kilder}

Datagrundlaget består af fire hovedkategorier: (1) regulatoriske kilder og standarder, (2) empiriske sektorkilder, (3) teoretisk litteratur og (4) case- og softwaremateriale. Kombinationen giver både normative rammer og praktiske indsigter, som er nødvendige for at analysere ESG-rapporteringens implementering.

Regulatoriske kilder vurderes som høj på autenticitet og stabilitet, men kræver fortolkning i anvendelsen. Empiriske sektorkilder er robuste, men aggregerede og derfor mindre egnede til at forklare lokale variationer. Case- og softwaremateriale giver detaljegrad og sporbarhed, men har begrænset ekstern validitet og behandles derfor som casebaseret evidens.

Tabel \ref{tab:datakilder-overblik} opsummerer datatyperne med periode, rolle og kvalitetsvurdering.
\begin{table}[t]
  \caption{Datakilder og kvalitetsvurdering, der afgrænser evidensstyrken.}
  \label{tab:datakilder-overblik}
  \begin{tabularx}{\textwidth}{X X X}
    \toprule
    Datakategori & Rolle i analysen & Kvalitetsvurdering \\
    \midrule
    Regulering og standarder (2020--2025) & Fastlægger krav, scope og væsentlighed. & Høj autenticitet; fortolkningsrum i praksis. \\
    Empiriske sektorkilder (2019--2024) & Understøtter sektorens ESG-profil og behov. & Robuste, men aggregerede data. \\
    Teoretisk litteratur (1970--2022) & Analytisk ramme for fortolkning. & Relevant, men kontekstafhængig. \\
    Case- og softwaremateriale (2025--2026) & Operationalisering af krav og processer (bilag \ref{app:teknisk-dokumentation}). & Høj detaljegrad; begrænset ekstern validitet. \\
    \bottomrule
  \end{tabularx}
  \TableSource{\parencite{EU2022CSRD,EU2023ESRS,EU2020Taxonomy,GRI2021,HCWH2019ClimateFootprint,WHO2024HealthCareWaste,WHO2021HealthWorkerDeaths}}
  \TableNote{Oversigten er ikke udtømmende.}
\end{table}

Alle eksterne kilder håndteres konsistent via \texttt{references.bib}. Case- og softwaremateriale er dokumenteret i bilag \ref{app:teknisk-dokumentation} for at sikre sporbarhed og efterprøvning.
