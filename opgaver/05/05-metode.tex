\section{Metode}
\label{sec:metode}

Metodeafsnittet redegør for tilrettelæggelsen af undersøgelsen med fokus på sporbarhed, konsistens og analytisk holdbarhed. Tilgangen kombinerer dokumentanalyse af regulatoriske krav med empirisk sektorviden, en casebaseret pilot og en analyse af en MVP som artefakt. Det muliggør en kobling mellem policy, organisatorisk praksis og teknisk operationalisering af ESG-krav.

I \ref{subsec:design-og-tilgang} beskrives forskningsdesign og analysestrategi. Datagrundlag og kilder uddybes i \ref{subsec:datagrundlag-og-kilder}, mens caseafgrænsning og relevans behandles i \ref{subsec:case-egen-virksomhed-som-pilot}. Etik, compliance og kvalitetskontrol er samlet i \ref{subsec:etik-compliance-og-kvalitet}, og de metodiske begrænsninger opsummeres i \ref{subsec:begraensninger}.

Metoden etablerer dermed grundlaget for, at softwarekapitlet kan fungere som operationalisering af kravene og som empirisk artefakt i analysen.
