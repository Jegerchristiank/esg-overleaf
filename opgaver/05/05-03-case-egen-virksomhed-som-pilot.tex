\subsection{Case: Egen virksomhed som pilot}
\label{subsec:case-egen-virksomhed-som-pilot}

Casen tager udgangspunkt i en egen virksomhed, der udvikler en ESG-as-a-Service-løsning til SMV'er i sundhedssektoren. Casen anvendes som pilot for at konkretisere, hvordan regulatoriske krav omsættes til datafelter, processer og rapportoutput i en praktisk kontekst. Dokumentationen er samlet i bilag \ref{app:teknisk-dokumentation}.

Datamaterialet er struktureret i en caseskabelon, der dækker produktbeskrivelse, målgruppe, leveranceprocesser og økonomiske antagelser. Casen beskriver modulbaseret dataindsamling inden for energi, affald og sociale KPI'er og viser, hvordan input valideres og omsættes til rapportoutput. Den giver dermed et konkret grundlag for at vurdere gennemførbarhed, ressourcebehov og organisatoriske implikationer.

Casen er valgt, fordi den repræsenterer en typisk SMV-kontekst med begrænsede ressourcer og samtidig markeds- og compliancepres. Samtidig indebærer valget en risiko for bias, da materialet er egenproduceret. Det håndteres ved at afgrænse generaliseringer og anvende casen som illustrativ evidens frem for statistisk repræsentativ dokumentation.

Da casen er egenproduceret, adresseres etik, compliance og kvalitet eksplicit i det næste afsnit (\ref{subsec:etik-compliance-og-kvalitet}).
