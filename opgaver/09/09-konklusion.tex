\section{Konklusion}
\label{sec:konklusion}

Analysen har belyst ESG-rapportering i sundhedssektoren og vurderet, hvordan et ESG-as-a-Service-koncept og en MVP kan omsætte regulatoriske krav til praksis. Arbejdet er forankret i standardiserings- og governanceperspektiver og suppleret af empiriske sektordata og casebaseret evidens.

Forskningsspørgsmålene kan besvares således:
\begin{enumerate}
  \item CSRD, ESRS, EU-taksonomien og GRI etablerer standardiserede krav til data, væsentlighed og dokumentation. For SMV'er sker påvirkningen primært indirekte via forsyningskæder og finansielle aktører, mens VSME giver et forenklet minimumsniveau.\parencite{EU2022CSRD,EU2023ESRS,EU2020Taxonomy,GRI2021,Virksomhedsguiden2025VSMEIntro,Virksomhedsguiden2025VSMEModules,Virksomhedsguiden2025NoDualMateriality}
  \item ESG-as-a-Service skaber værdiforslag ved at oversætte standarder til konkrete arbejdsgange, reducere transaktionsomkostninger og styrke sporbarhed og governance. Værdien afhænger af fokus på materielle temaer og af organisatorisk forankring, så rapportering ikke reduceres til symbolsk compliance.\parencite{BrunssonWorldOfStandards,RoevikTrenderTranslasjoner,TranslationTheoryKnowledgeTransfer,Khan2016Materiality}
  \item MVP'en kan omsætte krav til praksis gennem modulbaseret dataindsamling, validering, beregning og rapportoutput kombineret med auditspor. Den demonstrerer gennemførbarhed for SMV'er, men også behov for videre integration og udvidet dækning (bilag \ref{app:teknisk-dokumentation}).
\end{enumerate}

Samlet set viser analysen, at ESG-rapportering i sundhedssektoren kræver standardiseret datahåndtering og governance, og at ESG-as-a-Service kan fungere som en praktisk bro mellem regulering og drift. Konklusionerne er dog begrænset af case- og datagrundlag og bør valideres gennem flere empiriske studier.
