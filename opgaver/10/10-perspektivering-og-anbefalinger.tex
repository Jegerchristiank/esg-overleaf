\section{Perspektivering og anbefalinger}
\label{sec:perspektivering-og-anbefalinger}

Perspektiveringen peger på, at ESG-rapportering i sundhedssektoren vil udvikle sig fra ad hoc-projekter til en mere kontinuerlig driftsdisciplin. Fremtidig forskning bør derfor omfatte fler-case studier på tværs af delsektorer, longitudinelle analyser af implementering og test af, hvordan standardiserede data påvirker beslutningskvalitet og risikostyring.

\textbf{Empirisk baserede anbefalinger:}
\begin{itemize}
  \item SMV'er kan starte med VSME basismodul og gradvist udvide rapporteringen, efterhånden som datakvalitet og ressourcer forbedres.\parencite{Virksomhedsguiden2025VSMEIntro,Virksomhedsguiden2025VSMEModules,Virksomhedsguiden2025NoDualMateriality}
  \item Fokus bør ligge på materielle temaer (energi, affald, arbejdsmiljø), hvor evidensen peger på størst effekt og relevans.\parencite{Khan2016Materiality}
  \item ESG-as-a-Service-løsninger bør anvendes til at sikre auditspor og ensartet dataindsamling, så rapporteringen bliver sporbar og beslutningsrelevant (bilag \ref{app:teknisk-dokumentation}).
  \item Stop-the-clock bør ikke tolkes som en pause, men som en tidsbuffer til at etablere datagrundlag og governance-strukturer.\parencite{Erhvervsstyrelsen2025StopClock}
\end{itemize}

\textbf{Produktmæssige næste skridt:}
\begin{itemize}
  \item Udvid integrationer til centrale datakilder, så manuel indtastning reduceres.
  \item Standardisér mapping til ESRS/GRI og styrk rapportskabeloner, så output kan genbruges i flere sammenhænge.
  \item Forbered assurance-ready kontroller ved at uddybe auditspor og dokumentationsniveau.
  \item Tilføj analytiske funktioner (fx baseline-overblik og trendvisualisering), der understøtter løbende forbedringer.
\end{itemize}

\textbf{Normative vurderinger:}
\begin{itemize}
  \item Regulatorer bør prioritere proportionalitet og stabilitet i standarder, så SMV'er ikke presses til symbolsk rapportering uden kapacitet til reel implementering.
  \item Brancheaktører bør udvikle delte datapraksisser og fælles minimumsstandarder, der reducerer transaktionsomkostninger i forsyningskæderne.
  \item Serviceudbydere bør sikre transparens om antagelser og usikkerheder for at modvirke greenwashing og øge tillid til rapporteringen.
\end{itemize}
