\subsection{Formål og krav}
\label{subsec:formaal-og-krav}

Formålet med MVP'en er at reducere rapporteringsbyrden for SMV'er ved at samle ESG-data i en ensartet proces og skabe et auditabelt output. Løsningen skal gøre det muligt at indsamle kernedata, validere dem og generere dokumentation, som kan anvendes over for ledelse, revisorer og myndigheder.\parencite{InternalSoftwareSpec2026,InternalSoftwareDetails2026}

Kravene er opdelt i funktionelle og ikke-funktionelle krav og prioriteret efter, hvad der er nødvendigt i MVP'en versus en fuld løsning. Tabel \ref{tab:krav-prioritering} opsummerer den prioritering, der danner grundlag for evalueringen i \ref{subsec:demonstration-og-evaluering}.
\begin{table}[t]
  \caption{Krav til MVP og prioritering.}
  \label{tab:krav-prioritering}
  \begin{tabularx}{\textwidth}{l l l X}
    \toprule
    Krav & Type & Prioritet & Begrundelse \\
    \midrule
    Modulbaseret dataindsamling & Funktionel & MVP & Nødvendig for at strukturere ESG-data efter scope og tema. \\
    Validering af input & Funktionel & MVP & Sikrer datakvalitet og konsistens på tværs af moduler. \\
    Beregning af resultater & Funktionel & MVP & Omsætter input til indikatorer og skaber beslutningsgrundlag. \\
    Audit-log og versionsstyring & Ikke-funktionel & MVP & Understøtter sporbarhed og efterprøvbarhed. \\
    Rapporteksport (PDF/CSRD/XBRL) & Funktionel & MVP & Gør output anvendeligt for ledelse og compliance. \\
    Integration til ERP/HR-systemer & Funktionel & Fuld & Reducerer manuelt arbejde og øger skalerbarhed. \\
    Benchmark og avancerede alerts & Funktionel & Fuld & Tilføjer strategisk indsigt ud over compliance. \\
    Rollebaseret adgang og sikkerhed & Ikke-funktionel & MVP & Krævet for GDPR-kompatibel databehandling. \\
    \bottomrule
  \end{tabularx}
  \TableSource{\parencite{InternalSoftwareSpec2026,InternalSoftwareDetails2026}}
\end{table}
