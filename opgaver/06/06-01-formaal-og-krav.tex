\subsection{Formål og krav}
\label{subsec:formaal-og-krav}

Formålet med MVP'en er at reducere rapporteringsbyrden for SMV'er ved at samle ESG-data i en ensartet proces og skabe et auditabelt output. Løsningen skal gøre det muligt at indsamle kernedata, validere dem og generere dokumentation, som kan anvendes over for ledelse, revisorer og myndigheder. Kravene operationaliseres i case- og softwarematerialet i bilag \ref{app:teknisk-dokumentation}.

Kravene dækker både funktionelle og ikke-funktionelle behov og prioriteres efter, hvad der er nødvendigt i MVP'en versus en fuld løsning. Tabel \ref{tab:krav-prioritering} opsummerer den prioritering, der danner grundlag for evalueringen i \ref{subsec:demonstration-og-evaluering}.
\begin{table}[h]
  \caption{Krav og prioritering, der afgrænser hvad MVP'en skal kunne dokumentere.}
  \label{tab:krav-prioritering}
  \begin{tabularx}{\textwidth}{X l X}
    \toprule
    Krav & Prioritet & Begrundelse \\
    \midrule
    Modulbaseret dataindsamling & MVP & Nødvendig for at strukturere ESG-data efter scope og tema. \\
    Validering af input & MVP & Sikrer datakvalitet og konsistens på tværs af moduler. \\
    Beregning af indikatorer & MVP & Omsætter input til målbare resultater og beslutningsgrundlag. \\
    Sporbarhed og ændringshistorik & MVP & Understøtter efterprøvbarhed og audit. \\
    Rapporteksport (PDF og standardiserede formater) & MVP & Gør output anvendeligt for ledelse og compliance. \\
    Integration til eksisterende datakilder & Fuld & Reducerer manuelt arbejde og øger skalerbarhed. \\
    Benchmark og proaktive alarmer & Fuld & Tilføjer strategisk indsigt ud over compliance. \\
    Rollebaseret adgang og sikkerhed & MVP & Krævet for GDPR-kompatibel databehandling. \\
    \bottomrule
  \end{tabularx}
  \TableSource{Egen fremstilling baseret på case- og softwaremateriale (bilag \ref{app:teknisk-dokumentation}).}
\end{table}

Kravene omsættes til arkitektur og dataflows i næste afsnit (\ref{subsec:systemoversigt-og-arkitektur}).
