\subsection{Systemoversigt og arkitektur}
\label{subsec:systemoversigt-og-arkitektur}

MVP'en er en webbaseret løsning med en klar opdeling mellem brugergrænseflade, applikationslogik og datalagring. Løsningen håndterer brugerflow, dataindtastning, validering og lagring i et sammenhængende forløb. Den tekniske detaljering er samlet i bilag \ref{app:teknisk-dokumentation}, herunder software-specifikation (bilag \ref{app:software-spec}) og detaljeret specifikation (bilag \ref{app:detaljeret-software-spec}).

Dataflowet er kontrolleret: input indsamles, valideres og gemmes med historik, så ændringer kan spores og genskabes. Arkitekturen understøtter sporbarhed og gør det muligt at udvide komponenter uden at bryde den samlede proceslogik.

Figur \ref{fig:system-arkitektur} viser arkitekturens hovedkomponenter og deres relationer.
\begin{figure}[h]
  \centering
  \fbox{\begin{tabular}{c}
    Brugergrænseflade \\
    $\downarrow$ \\
    Applikationslogik \\
    $\downarrow$ \\
    Datalagring \\
  \end{tabular}}
  \caption{Systemoversigt, der viser hvor data valideres og gøres sporbare.}
  \label{fig:system-arkitektur}
  \FigureSource{Egen fremstilling.}
\end{figure}

For at gøre sporbarheden operationel beskrives datamodellen og de centrale ESG-indikatorer i det næste afsnit (\ref{subsec:datamodel-og-esg-indikatorer}).
