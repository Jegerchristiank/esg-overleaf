\subsection{Datamodel og ESG-indikatorer}
\label{subsec:datamodel-og-esg-indikatorer}

Datamodellen er designet til at understøtte sporbarhed gennem versionshistorik og auditspor. Den organiserer data i tre logiske lag: profil og afgrænsning, indikatorregistreringer og ændringshistorik.

Tabel \ref{tab:datamodel-kernetable} viser datamodellens hovedkomponenter og deres funktion i MVP'en.
\begin{table}[t]
  \caption{Hovedkomponenter i datamodellen.}
  \label{tab:datamodel-kernetable}
  \begin{tabularx}{\textwidth}{l X X}
    \toprule
    Dataobjekt & Formål & Eksempler på indhold \\
    \midrule
    Profil og afgrænsning & Fastlægger organisationens scope og relevante moduler. & Grundoplysninger, perioder, afgrænsninger. \\
    Indikatorregistrering & Samler data pr. modul til beregning og rapportering. & Forbrugstal, mængder, enheder, datakilder. \\
    Ændrings- og beregningsspor & Dokumenterer ændringer og udledte resultater. & Tidsstempel, ændringstype, beregningsgrundlag. \\
    \bottomrule
  \end{tabularx}
  \TableSource{Egen fremstilling; datadefinitioner fremgår af bilag \ref{app:teknisk-dokumentation}.}
\end{table}

MVP'en fokuserer på indikatorer inden for energi, affald og sociale forhold, i tråd med casebeskrivelsen og sektorens ESG-profil. Tabel \ref{tab:esg-indikatorer} viser eksempler på indikatorer og deres datakilder.
\begin{table}[t]
  \caption{Eksempler på ESG-indikatorer i MVP'en.}
  \label{tab:esg-indikatorer}
  \begin{tabularx}{\textwidth}{l X l X}
    \toprule
    Område & Eksempel på datafelt & Enhed & Typisk datakilde \\
    \midrule
    Energi & Varme- og elforbrug & kWh & Drifts- og energidata. \\
    Affald & Mængde affald pr. fraktion & kg/ton & Affaldsopgørelser og leverandørdata. \\
    Sociale forhold & Arbejdsmiljøindikatorer & antal/\% & HR-data og interne registreringer. \\
    \bottomrule
  \end{tabularx}
  \TableSource{Egen fremstilling baseret på case-materiale (bilag \ref{app:teknisk-dokumentation}).}
  \TableNote{Indikatorerne er illustrative og afspejler MVP'ens fokus.}
\end{table}

\subsubsection{Operationalisering af ESRS E1 til rapporteringsvariable}
\label{subsec:operationalisering-esrs-e1}

For at gøre oversættelsen fra standard til software eksplicit er udvalgte datapunkter fra ESRS E1 operationaliseret til konkrete input, beregningsregler og outputfelter i MVP'en. Tabel \ref{tab:esrs-e1-operationalisering} viser en forenklet mapping, som gør den tekniske logik sporbar.
\begin{table}[t]
  \caption{Eksempel på operationalisering af ESRS E1 i MVP'en.}
  \label{tab:esrs-e1-operationalisering}
  \begin{tabularx}{\textwidth}{l X X X}
    \toprule
    ESRS E1-datapunkt & Input & Beregningsregel & Output i MVP \\
    \midrule
    Energiforbrug & El- og varmeforbrug (kWh) pr. periode. & $E = \sum kWh$ & Energiforbrug pr. modul/periode. \\
    Scope 2-udledninger & Energiforbrug og emissionsfaktor. & $\mathrm{CO_2e}_{\text{el}} = kWh \cdot EF_{\text{el}}$ & Scope 2 CO$_2$e pr. periode. \\
    Scope 1-udledninger & Brændselsforbrug og emissionsfaktor. & $\mathrm{CO_2e}_{\text{br}} = m \cdot EF_{\text{br}}$ & Scope 1 CO$_2$e pr. periode. \\
    Scope 3 (udvalgt kategori) & Affaldsmængder og faktor pr. fraktion. & $\mathrm{CO_2e}_{\text{aff}} = m \cdot EF_{\text{aff}}$ & Scope 3 CO$_2$e pr. fraktion. \\
    \bottomrule
  \end{tabularx}
  \TableSource{\parencite{EU2023ESRS,GHGProtocol2004}}
  \TableNote{Mappingen er illustrativ og viser principper for oversættelse fra standard til datapunkter.}
\end{table}

De centrale beregninger kan udtrykkes med følgende forenklede relationer, der gør auditlogikken synlig:
\begin{align*}
  \mathrm{CO_2e}_{\text{total}} &= \sum_i \left( A_i \cdot EF_i \right) \\
  \mathrm{CO_2e}_{\text{el}} &= E_{\text{el}} \cdot EF_{\text{el}} \\
  \mathrm{CO_2e}_{\text{varme}} &= E_{\text{varme}} \cdot EF_{\text{varme}}
\end{align*}
hvor $A_i$ er aktivitetsdata (fx \si{\kilo\watt\hour} eller \si{\kilogram}) og $EF_i$ er emissionsfaktorer (kg CO$_2$e pr. enhed). Emissionsfaktorer antages at være scope- og periodespecifikke og hentes fra standardtabeller.\parencite{GHGProtocol2004}

Indikatorerne er organiseret efter ESG-domæner, hvilket gør det muligt at mappe data til de overordnede strukturer i ESRS og GRI uden at påstå fuld dækning af alle datapunkter.\parencite{EU2023ESRS,GRI2021}
En samlet moduloversigt for MVP'en er placeret i bilagene (tabel \ref{tab:app-moduloversigt}) for at give et supplerende overblik over scopes og moduler.
