\subsection{Dataindsamling og automatisering}
\label{subsec:dataindsamling-og-automatisering}

Dataindsamlingen er bygget op omkring et wizard-flow, hvor brugeren indtaster eller uploader data pr. modul. Input valideres gennem valideringsregler i data-skemaet, og data persisteres som snapshots, så input kan spores over tid.\parencite{InternalSoftwareSpec2026,InternalSoftwareDetails2026}

Figur \ref{fig:dataflow-pipeline} opsummerer dataflowet fra datakilder til rapportoutput og viser, hvor validering og beregning sker i processen.
\begin{figure}[t]
  \centering
  \fbox{\begin{tabular}{c}
    Datakilder (ERP/HR/facilities) \\
    $\downarrow$ CSV/API/manuel input \\
    Wizard-flow og validering \\
    $\downarrow$ Beregningsmotor \\
    $\downarrow$ Audit-log og versionsstyring \\
    $\downarrow$ Rapportoutput (PDF/CSRD/XBRL) \\
  \end{tabular}}
  \caption{Forenklet dataflow fra indsamling til rapportoutput.}
  \label{fig:dataflow-pipeline}
  \FigureSource{Egen fremstilling.}
\end{figure}

Automatisering sker primært gennem autosave og beregningsmotor. Autosave reducerer risikoen for datatab, mens beregningsmotoren omsætter input til indikatorer og samlede resultater. Audit-log registrerer ændringer og versioner og skaber en teknisk sporbarhed, som er nødvendig for efterprøvelse. Versioner opdateres kun ved reelle ændringer, så loggen bevarer relevans.

Dataindsamling kan ske via CSV/API/Excel fra ERP- og HR-systemer samt facilities, men MVP'en understøtter også manuel indtastning for at sikre, at rapporteringen kan gennemføres uden fulde integrationer. Kvalitetskontrol sker gennem valideringsregler, obligatoriske felter og konsistente enheder.
