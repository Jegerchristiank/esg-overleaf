\subsection{Dataindsamling og automatisering}
\label{subsec:dataindsamling-og-automatisering}

Dataindsamlingen er organiseret som et guidet modulforløb, hvor brugeren indtaster eller uploader data pr. modul. Input valideres gennem faste regler, og data gemmes med versionshistorik, så ændringer kan spores over tid. Detaljer om datadefinitioner og kontrolpunkter fremgår af bilag \ref{app:teknisk-dokumentation}.

Figur \ref{fig:dataflow-pipeline} opsummerer dataflowet fra datakilder til rapportoutput og viser, hvor validering og beregning sker i processen.
\begin{figure}[h]
  \centering
  \fbox{\begin{tabular}{c}
    Datakilder (drift, HR, økonomi) \\
    $\downarrow$ Strukturering og manuel input \\
    Guidet indsamling og validering \\
    $\downarrow$ Beregningsmotor \\
    $\downarrow$ Auditspor og versionshistorik \\
    $\downarrow$ Rapportoutput (PDF og standardiserede formater) \\
  \end{tabular}}
  \caption{Forenklet dataflow, der viser hvor validering, beregning og auditspor skaber sporbarhed.}
  \label{fig:dataflow-pipeline}
  \FigureSource{Egen fremstilling.}
\end{figure}

Automatisering sker gennem løbende gemmefunktion og beregning af indikatorer. Det reducerer risikoen for datatab og skaber et konsistent grundlag for rapportoutput. Auditsporet registrerer ændringer og versioner og gør det muligt at efterprøve resultater og antagelser.

Dataindsamling kan ske via API-integrationer eller CSV-udtræk fra drifts-, økonomi- og HR-data, men løsningen understøtter også manuel indtastning for at sikre, at rapporteringen kan gennemføres uden fulde integrationer. Kvalitetskontrol sker gennem valideringsregler, obligatoriske felter og konsistente enheder. IWA 48 understreger behovet for standardiserede KPI'er og rapporteringsprincipper med validering og dokumentation, hvilket understøtter krav om sporbarhed og datakvalitet i MVP'en.\parencite{ISO2024IWA48}

For at synliggøre governance i datakvalitet er det nyttigt at knytte typiske fejltyper til konkrete kontrolpunkter. Tabel \ref{tab:datakvalitet-kontrol} viser eksempler på, hvordan MVP'en teknisk reducerer risikoen for fejl og inkonsistens.
\begin{table}[h]
  \caption{Datakvalitetsfejl, der viser hvorfor tekniske kontrolpunkter er nødvendige.}
  \label{tab:datakvalitet-kontrol}
  \begin{tabularx}{\textwidth}{l X X}
    \toprule
    Fejltype & Konsekvens & MVP-kontrolpunkt \\
    \midrule
    Manglende scope-data & Under- eller fejlrapportering af udledninger. & Obligatoriske felter og datadækningstjek pr. modul. \\
    Inkonsekvente enheder & Misvisende beregninger og sammenligninger. & Enhedsvalidering og automatisk normalisering. \\
    Periodeafgrænsning & Uens rapporteringsår og manglende sammenlignelighed. & Låste rapporteringsperioder og versionshistorik. \\
    Dubletter i input & Overestimerede mængder og KPI'er. & Dubletkontrol ved import og ændringslog. \\
    \bottomrule
  \end{tabularx}
  \TableSource{Egen fremstilling.}
  \TableNote{Eksemplerne er illustrative og bygger på almindelige datakvalitetsproblemer i ESG-rapportering.}
\end{table}

Når dataindsamlingen er etableret, bliver det afgørende at forstå brugerflow og rapportoutput, som behandles i \ref{subsec:brugerflow-og-rapportoutput}.
