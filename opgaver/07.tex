\section{Analyse}
\label{sec:analyse}

\begin{quote}
ESG-rapportering i praksis risikerer at blive dokumentation uden styring.
\end{quote}

Standardiserings-, governance- og økonomiske perspektiver anvendes til at fortolke empirien og MVP'ens rolle i ESG-rapportering. Fokus er på, hvordan standarder og governance omsættes til praksis, hvordan værdiforslaget kan begrundes økonomisk, og hvilke organisatoriske konsekvenser servicekonceptet skaber for SMV'er i sundhedssektoren.

Fokus er at vurdere, om standardisering via ESG-as-a-Service reducerer risikoen for dekobling, og om rapporteringen bliver beslutningsrelevant frem for blot minimum compliance. Det gør det muligt at vurdere, hvor løftet om governance faktisk indfries i praksis.
\subsection{ESG som organisationsstandard}
\label{subsec:esg-som-organisationsstandard}

\textcite{BrunssonWorldOfStandards} beskriver standarder som regler, der skaber ensartethed og legitimitet, men som ikke automatisk følges i praksis. I sundhedssektoren fungerer ESG-krav som en sådan standardiseringsramme: de definerer, hvad der tæller som gyldig dokumentation, og hvilke indikatorer der skal måles. Samtidig skaber sektorens kompleksitet og datafragmentering en risiko for dekobling mellem rapportering og faktisk praksis.\parencite{BrunssonWorldOfStandards,HCWH2019ClimateFootprint,WHO2024HealthCareWaste} Ratingdivergens dokumenteret af \textcite{Berg2022AggregateConfusion} viser, at standarder alene ikke sikrer sammenlignelighed, hvilket øger betydningen af datakvalitet og sporbarhed.

Pensum om translasion understreger, at ideer og krav altid oversættes til lokale praksisser og dermed forandres i processen.\parencite{RoevikTrenderTranslasjoner,TranslationTheoryKnowledgeTransfer} Det betyder, at ESG-rapportering i SMV'er ikke bliver en direkte implementering af ESRS/GRI, men en lokal tilpasning, hvor ressourcer, kompetencer og datasystemer afgør, hvad der reelt kan rapporteres. Case-materialet viser en tilsvarende prioritering af energi, affald og sociale KPI'er som operationel kerne frem for fuld dækning af alle standardkrav (bilag \ref{app:teknisk-dokumentation}).

Sektorens empiriske profil understøtter behovet for standardisering: klimaaftrykets tyngde i scope~3 og de sociale risici i arbejdsmiljø kræver konsistente datakilder og klar dokumentation.\parencite{HCWH2019ClimateFootprint,WHO2021HealthWorkerDeaths} Samtidig viser litteraturen, at ESG-rammer overlapper med bredere sundheds- og miljøhensyn, hvilket gør standardisering til en koordinationsmekanisme på tværs af faglige domæner.\parencite{Vegro2025OneHealth}

Analytisk betyder det, at ESG som organisationsstandard skaber et pres for ensartethed, men også en systemisk risiko for dekobling, fordi rapportering kan blive et mål i sig selv. ESG-as-a-Service kan fungere som den infrastrukturelle oversætter, der binder standarder til konkrete datafelter og auditspor, men hvis løsningen primært optimerer rapporteringen frem for praksis, institutionaliserer den netop den dekobling, standarderne skulle modvirke. Derfor behandles dekobling som en hovedrisiko i vurderingen af ESG-as-a-Service.
\subsection{Værdiforslag og forretningsmodel}
\label{subsec:vaerdiforslag-og-forretningsmodel}

Værdiforslaget afhænger af, om ESG-arbejdet flyttes fra dokumentation til beslutningsgrundlag. Det kan forstås gennem en kombination af stakeholder- og shared value-perspektiver. Stakeholder-tilgangen betoner, at legitimitet og ansvarlighed skaber værdi for flere interessenter end aktionærer, mens shared value argumenterer for, at samfundsmæssige udfordringer kan omsættes til forretningsmuligheder.\parencite{Freeman1984Stakeholder,PorterKramer2011CSV} Dette støttes af pensum om ESG, som understreger behovet for at integrere ESG i kerneforretningen og bruge data som beslutningsgrundlag, ikke kun som compliance.\parencite{ESGBook} OECD peger samtidig på, at ESG-data ofte er inkonsistente og vanskelige at sammenligne, hvilket gør standardiseret indsamling og validering til en kerne i værdiforslaget.\parencite{OECD2020ESG}

MVP'ens værdiforslag består i at reducere transaktionsomkostninger ved dataindsamling, standardisere input og skabe sporbar dokumentation, der kan anvendes i dialog med banker, kunder og myndigheder (bilag \ref{app:teknisk-dokumentation}). For SMV'er er dette centralt, fordi de ofte mangler interne ESG-specialister og har begrænsede ressourcer til komplekse rapporteringsprocesser.\parencite{Virksomhedsguiden2025SMVDefinition}

Forretningsmodellen hviler på tre elementer: et digitalt produkt (dataindsamling og beregning), en compliance-service (validering og dokumentation) og et rapportoutput (PDF og standardiserede formater). Den reducerer måle- og rapporteringsfriktion, hvilket gør ESG til en operationaliserbar praksis snarere end et abstrakt krav. Det understøtter samtidig et strategisk narrativ om, at ESG-data kan bruges til risikostyring og værdiskabelse, i tråd med evidensen for positiv eller neutral sammenhæng mellem ESG og finansiel performance.\parencite{Eccles2014Impact,Friede2015Meta}

Alternativer til servicekonceptet er typisk konsulentprojekter, interne ESG-teams eller generiske rapporteringsværktøjer. Konsulenter kan levere specialviden, men skaber ofte højere transaktionsomkostninger og mindre løbende standardisering. Interne løsninger giver kontrol, men kræver kapacitet og datafaglighed, som mange SMV'er ikke har. Generiske værktøjer kan være billige, men matcher sjældent sektorspecifikke databehov.

Samtidig peger \textcite{Khan2016Materiality} på, at ESG kun skaber finansiel værdi, når indsatsen er rettet mod materielle temaer. Det betyder, at løsningen skal prioritere data, der er relevante for sundhedssektorens kernerisici (energi, affald, arbejdsmiljø) og ikke blot dække alle standarder mekanisk. Værdiforslaget afhænger derfor af evnen til at koble standardkrav til sektorrelevante indikatorer og til at omsætte dem til beslutningsrelevante outputs.

En bæredygtig forretningsmodel forudsætter, at kunderne oplever et klart compliance- og risikobenefit, at data kan indsamles med begrænset friktion, og at onboarding og support kan skaleres uden at øge omkostningsbasen proportionalt.

\subsubsection{Fra governance til forretning: hvor grænsen går}
\label{subsec:fra-governance-til-forretning}

Servicekonceptet bevæger sig fra governance til forretning, når data ikke længere primært fungerer som dokumentation, men som aktivt beslutningsgrundlag for drift, investeringer og markedspositionering. I governance-laget er værdien knyttet til compliance, sporbarhed og risikoreduktion; i forretningslaget opstår værdi, når ESG-data bruges til at optimere omkostninger (fx energi og affald), differentiere produkter eller understøtte pris- og segmentstrategier. Det skift kan operationaliseres ved, at output går fra minimumsrapporter til KPI'er, der indgår i ledelsesrapportering og kundedialog. Grænsen mellem governance og forretning ligger derfor ikke i selve rapportformatet, men i om data anvendes aktivt til strategiske beslutninger og kommercielle prioriteringer.\parencite{PorterKramer2011CSV,ESGBook}
\subsection{Prisfastsættelse og kundesegmentering}
\label{subsec:prisfastsaettelse-og-kundesegmentering}

Segmenteringen bør styres af datakompleksitet og compliancepres snarere end alene virksomhedstype. I sundhedssektoren kan der skelnes mellem (1) private klinikker med begrænset datagrundlag, (2) mindre hospitaler med mere kompleks drift, og (3) medtech-leverandører med compliancepres fra forsyningskæden. Segmenterne adskiller sig på rapporteringsbehov, integrationskrav og betalingsvillighed.

Prisfastsættelsen er derfor differentieret efter kompleksitet og behov. Casen peger på en abonnementsmodel for standardmoduler, et projektgebyr for integration og opsætning, samt usage-gebyrer for eksport af specifikke rapportpakker (bilag \ref{app:teknisk-dokumentation}). Denne struktur afspejler, at en stor del af omkostningen ligger i initial opsætning og datakortlægning, mens løbende drift kan standardiseres.

Et simpelt prisudtryk kan beskrives som:
\begin{equation*}
  P = \frac{F}{N} + V + m
\end{equation*}
hvor $F$ er faste omkostninger, $N$ er antal kunder, $V$ er variable omkostninger pr. kunde, og $m$ er en risikomargin. Modellen synliggør, at skalerbarhed i onboarding og support er afgørende for at holde $P$ konkurrencedygtig.

Betalingsvilligheden er tæt knyttet til compliancepres og interessentkrav. Selvom mange SMV'er ikke er direkte CSRD-pligtige, møder de dataanmodninger fra kunder og finansielle aktører og vælger derfor frivillig rapportering som risikoreduktion.\parencite{PwC2025GuideESGSMV,Erhvervsstyrelsen2025StopClock} Finanstilsynet beskriver ESG som en stigende risikofaktor i kreditvurdering, hvilket gør bankkrav til en strukturel driver for betalingsvillighed.\parencite{Finanstilsynet2025ESGRisk,GrantThorntonDK2025BankESG} EY's analyse af VSME-rapporter indikerer, at mange virksomheder rapporterer ud over basismodullets krav, hvilket peger på en villighed til at investere i bedre datakvalitet.\parencite{EY2025VSME}

Analytisk betyder det, at prisstrategien bør balancere lav adgangsbarriere for mindre aktører med mulighed for opgradering for kunder med højere compliancekrav. En modulopbygget prisstruktur understøtter denne balance, fordi den knytter omkostninger direkte til datakompleksitet og rapporteringsomfang.

\begin{table}[h]
  \caption{Illustrativ prisdifferentiering, der afspejler, at compliancepres og datakompleksitet driver betalingsvillighed.}
  \label{tab:pris-segmenter}
  \begin{tabularx}{\textwidth}{l X r}
    \toprule
    Segment & Karakteristika & Indikativ pris pr. måned (DKK) \\
    \midrule
    Private klinikker & Basismodul, få datakilder, lav integrationsgrad. & \numrange{1500}{3000} \\
    Mindre hospitaler & Flere moduler, flere datakilder, højere kompleksitet. & \numrange{4000}{7000} \\
    Medtech-leverandører & Høj compliance, eksportkrav, dokumentationskrav fra kunder. & \numrange{7000}{12000} \\
    \bottomrule
  \end{tabularx}
  \TableSource{Egen fremstilling.}
  \TableNote{Prisintervallerne er illustrative og afhænger af datakompleksitet og serviceomfang.}
\end{table}
\subsection{Organisatoriske og økonomiske implikationer}
\label{subsec:organisatoriske-og-oekonomiske-implikationer}

Implementering er i praksis en governance-opgave: hvem ejer data, hvem godkender dem, og hvordan sikres sporbarhed. Standarder og regulering etablerer forventninger til dokumentation, hvilket betyder, at SMV'er må definere ansvar for dataindsamling, validering og rapportering.\parencite{EU2022CSRD,EU2023ESRS} Det skaber behov for en intern funktion eller et eksternt serviceled, der kan oversætte krav til operative beslutninger.\parencite{RoevikTrenderTranslasjoner,TranslationTheoryKnowledgeTransfer}

Ressourcebehovet optræder på flere niveauer: tid til dataindsamling, kompetencer til fortolkning af standarder og tekniske ressourcer til at integrere data. Løsningen kan reducere noget af dette ved at tilbyde standardiserede arbejdsgange og auditspor, men kræver stadig organisatorisk forankring for at undgå dekobling mellem rapportering og praksis.\parencite{BrunssonWorldOfStandards}

Omkostningsstrukturen er typisk fronttung med udgifter til datakortlægning, integration og oplæring, mens den løbende drift domineres af datavedligehold, rapportering og kontrol. Return on investment afhænger af, om virksomheden kan reducere manuelle processer, mindske compliance-risici og anvende ESG-data i beslutninger. Manglende datakvalitet øger revisionsrisikoen og kan gøre rapporteringen mindre brugbar, hvilket svækker både governance og den forventede værdi.

Økonomisk set handler implikationerne om forholdet mellem omkostninger og potentiel værdi. Metastudier peger på, at ESG-arbejde ofte er forbundet med neutral eller positiv finansiel performance, men effekten er betinget af fokus på materielle temaer.\parencite{Friede2015Meta,Khan2016Materiality} For SMV'er betyder det, at ressourcer bør prioriteres til data og indikatorer, der er direkte relevante for sundhedssektorens risici og interessentkrav.

Det betyder, at servicekonceptet skal vurderes på sin evne til at reducere faste compliance-omkostninger og skabe beslutningsrelevante data, snarere end blot at levere rapporter. Governance-implikationen er, at organisationen må etablere en stabil rapporteringsrutine, hvor dataindsamling bliver en integreret del af driften.
\subsection{Implementering i sundhedssektoren}
\label{subsec:implementering-i-sundhedssektoren}

Implementering i sundhedssektoren bremses ikke primært af vilje, men af datafragmentering, kapacitetsmangel og kulturel kompleksitet. Sektoren er afhængig af mange datakilder (energi, affald, arbejdsmiljø), og data er ofte spredt på tværs af systemer og leverandører.\parencite{HCWH2019ClimateFootprint,WHO2024HealthCareWaste} Det gør standardiseret rapportering vanskelig uden en teknisk infrastruktur til indsamling og validering.

Samtidig er der tydelige incitamenter. Regulering og forsyningskædekrav driver rapporteringsbehov, og litteraturen peger på, at ESG og digital transformation kan styrke sektorens bæredygtighed og robusthed.\parencite{Sepetis2024Healthcare,Bosco2024ESGHealth} One Health-perspektivet underbygger, at sundhedssektoren har et særligt ansvar for at koble miljø og sociale hensyn, hvilket forstærker behovet for systematisk implementering.\parencite{Vegro2025OneHealth}

Modenhed og forandringsparathed varierer på tværs af delsektorer. Mindre klinikker har ofte lav datakapacitet og begrænsede ressourcer, mens hospitaler og medtech-leverandører typisk har stærkere governance-strukturer og større compliancepres. Det betyder, at implementeringen må tilpasses organisatorisk modenhed og datainfrastruktur.

En trinvist implementeringsstrategi er derfor mest realistisk for SMV'er: (1) kortlægning af datakilder og etablering af minimumsmoduler (energi, affald, sociale KPI'er), (2) standardiseret indsamling med sporbarhed, og (3) gradvis udvidelse til mere komplekse datakrav og integrationer. Case-materialet peger på samme trappelogik (bilag \ref{app:teknisk-dokumentation}). Dette harmonerer med pensum om translasion, hvor oversættelse og kontekstualisering er afgørende for, at nye standarder får organisatorisk gennemslag.\parencite{RoevikTrenderTranslasjoner,TranslationTheoryKnowledgeTransfer}

Implementeringsmodenhed afhænger derfor ikke kun af teknologi, men af organisatorisk læring, governance og prioritering. Spændingen opstår, når kravet om standardiseret dokumentation møder hverdagens driftslogik. Servicekonceptet kan reducere teknisk kompleksitet, men organisatorisk forankring er stadig nødvendig for at undgå symbolsk efterlevelse. Prioriteringen bør være at sikre datagrundlag og minimumsmoduler først, derefter standardiseret rapportering og til sidst integrationer, der muliggør løbende forbedringer.
