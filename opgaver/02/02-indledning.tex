\section{Indledning}
\label{sec:indledning}

\ESG-rapportering i sundhedssektoren er i stigende grad et styrings- og legitimitetskrav snarere end en frivillig kommunikationsopgave. Sektoren kombinerer kritiske ydelser, komplekse forsyningsk\ae der og h\o je krav til dokumentation, hvilket g\o r sporbar og konsistent rapportering s\ae rligt udfordrende for mange akt\o rer.

Rapporten analyserer, hvordan \CSRD, ESRS, EU-taksonomien og GRI oms\ae tter politiske m\aa ls\ae tninger til konkrete rapporteringskrav, og hvordan disse krav kan operationaliseres i praksis for \SMV'er i sundhedssektoren. Analysen kombinerer policy- og organisationsperspektiver med et \ESG-as-a-Service-koncept og en \MVP-baseret case, der viser dataindsamling, sporbarhed og rapportoutput.

Hovedproblemet er pr\ae ciseret i \ref{subsec:problemformulering} og afgr\ae nset i \ref{subsec:afgraensning}. Baggrund og motivation uddybes i \ref{subsec:baggrund-og-motivation}, mens kontekst, teori, metode og software behandles i \ref{sec:kontekst-og-rammer}--\ref{sec:software-og-mvp}. Analysen og diskussionen samler resultaterne i \ref{sec:analyse} og \ref{sec:diskussion}, og de endelige svar opsummeres i \ref{sec:konklusion}.
