\section{Indledning}
\label{sec:indledning}

ESG-rapportering (Environmental, Social og Governance) i sundhedssektoren er i stigende grad et styrings- og legitimitetskrav snarere end en frivillig kommunikationsopgave. Sektoren kombinerer kritiske ydelser, komplekse forsyningskæder og høje krav til dokumentation, hvilket gør sporbar og konsistent rapportering særligt udfordrende for mange aktører.

Rapporten analyserer, hvordan Corporate Sustainability Reporting Directive (CSRD), European Sustainability Reporting Standards (ESRS), EU-taksonomien og Global Reporting Initiative (GRI) omsætter politiske målsætninger til konkrete rapporteringskrav, og hvordan disse krav kan operationaliseres i praksis for små og mellemstore virksomheder (SMV'er) i sundhedssektoren. Analysen kombinerer policy- og organisationsperspektiver med et ESG-as-a-Service-koncept og en Minimum Viable Product (MVP)-baseret case, der viser dataindsamling, sporbarhed og rapportoutput.

Hovedproblemet er præciseret i \ref{subsec:problemformulering} og afgrænset i \ref{subsec:afgraensning}. Baggrund og motivation uddybes i \ref{subsec:baggrund-og-motivation}, mens formål og forskningsspørgsmål præsenteres i \ref{subsec:formaal-og-forskningssporgsmaal}. Kontekst, teori, metode og software behandles i \ref{sec:kontekst-og-rammer}--\ref{sec:software-og-mvp}, og rapportens samlede struktur er beskrevet i \ref{subsec:rapportens-opbygning}. Analysen og diskussionen samler resultaterne i \ref{sec:analyse} og \ref{sec:diskussion}, og de endelige svar opsummeres i \ref{sec:konklusion}.
