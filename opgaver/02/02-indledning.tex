\section{Indledning}
\label{sec:indledning}

ESG-rapportering (Environmental, Social og Governance) i sundhedssektoren er i stigende grad et styrings- og legitimitetskrav snarere end en frivillig kommunikationsopgave. Sektoren kombinerer kritiske ydelser, komplekse forsyningskæder og høje krav til dokumentation, hvilket gør sporbar og konsistent rapportering særligt udfordrende for mange aktører.
Det skaber et spændingsfelt mellem dokumentationskrav og den kapacitet, særligt SMV'er realistisk kan mobilisere.

Analysen undersøger, hvordan Corporate Sustainability Reporting Directive (CSRD), European Sustainability Reporting Standards (ESRS), EU-taksonomien og Global Reporting Initiative (GRI) omsætter politiske målsætninger til konkrete rapporteringskrav, og hvordan disse krav kan operationaliseres i praksis for små og mellemstore virksomheder (SMV'er) i sundhedssektoren. Tilgangen kombinerer policy- og organisationsperspektiver med et ESG-as-a-Service-koncept og en MVP-baseret case, der viser dataindsamling, sporbarhed og rapportoutput.

Den gennemgående argumentation er, at regulative krav skaber en kløft mellem standarder og operationel datapraksis i SMV'er, og at ESG-as-a-Service kan fungere som en oversættelsesmekanisme, der reducerer denne kløft. Rapporten følger derfor bevægelsen fra krav og kontekst over teori og metode til MVP og analyse, så konklusionen kan afklare, hvornår rapporteringen bliver beslutningsrelevant frem for symbolsk.
