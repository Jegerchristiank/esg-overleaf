\subsection{Struktur}
Strukturen er valgt for at gøre argumentkæden efterprøvbar og fastholde den gennemgående kløft mellem krav og datapraksis: Indledningen og konteksten etablerer problemets omfang og de regulatoriske rammer (\ref{sec:indledning}--\ref{sec:kontekst-og-rammer}), så det bliver klart, hvilke krav analysen faktisk skal forklare. Den teoretiske ramme og metodekapitlet (\ref{sec:teoretisk-ramme}--\ref{sec:metode}) fastlægger begreber, afgrænsninger og evidensgrundlag, som efterfølgende omsættes til praksis i softwarekapitlet (\ref{sec:software-og-mvp}). Analysen og diskussionen (\ref{sec:analyse}--\ref{sec:diskussion}) vurderer derefter, om ESG-as-a-Service reducerer eller reproducerer spændinger mellem standardisering og praksis. Konklusion og perspektivering prioriterer bidrag og implikationer (\ref{sec:konklusion} og \ref{sec:perspektivering-og-anbefalinger}), mens bilagene samler dokumentation og tekniske detaljer (\ref{app:bilag}).
