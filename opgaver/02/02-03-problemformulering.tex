\subsection{Problemformulering}
\label{subsec:problemformulering}

Rapporten tager udgangspunkt i \ESG som samlebetegnelse for milj\o-, sociale og governance-forhold, der forventes dokumenteret i virksomheders rapportering. \SMV'er forst\aa s som ikke-b\o rsnoterede virksomheder, der falder under EU's SMV-definition, og som typisk har begr\ae nsede ressourcer til compliance og datastyring.\parencite{Virksomhedsguiden2025SMVDefinition} \ESG-as-a-Service anvendes her som betegnelse for en kombineret software- og serviceleverance, der overs\ae tter standardkrav til operationelle datapunkter, kontroller og rapportoutput.

Reguleringen skaber et pres for ensartet og sporbar rapportering, men i sundhedssektoren er data ofte fragmenterede p\aa tv\ae rs af systemer, leverand\o rer og organisatoriske enheder. Dette g\o r der en kl\o ft mellem normative standarder og den praktiske mulighed for at levere valide og sammenlignelige \ESG-data.

\textbf{Problemformulering:} Hvordan kan \ESG-as-a-Service, underst\o ttet af en \MVP, reducere kl\o ften mellem regulatoriske krav og operationel datapraksis for \SMV'er i sundhedssektoren, s\aa rapportering bliver b\aa de sporbar og beslutningsrelevant?
