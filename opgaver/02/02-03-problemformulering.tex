\subsection{Problemformulering}
\label{subsec:problemformulering}

Analysen tager udgangspunkt i ESG som samlebetegnelse for miljø-, sociale og governance-forhold, der forventes dokumenteret i virksomheders rapportering. SMV'er forstås som ikke-børsnoterede virksomheder, der falder under EU's SMV-definition, og som typisk har begrænsede ressourcer til compliance og datastyring.\parencite{Virksomhedsguiden2025SMVDefinition} ESG-as-a-Service anvendes her som betegnelse for en kombineret software- og serviceleverance, der oversætter standardkrav til operationelle datapunkter, kontroller og rapportoutput.

Reguleringen skaber et pres for ensartet og sporbar rapportering, men i sundhedssektoren er data ofte fragmenterede på tværs af systemer, leverandører og organisatoriske enheder. Dette gør der en kløft mellem normative standarder og den praktiske mulighed for at levere valide og sammenlignelige ESG-data.

\textbf{Problemformulering:} Hvordan kan ESG-as-a-Service, understøttet af en MVP, reducere kløften mellem regulatoriske krav og operationel datapraksis for SMV'er i sundhedssektoren, så rapportering bliver både sporbar og beslutningsrelevant?

Problemformuleringen fungerer som gennemgående test i rapporten: Hvert kapitel bidrager med en del af forklaringen på, hvordan krav oversættes til praksis og hvor rapporteringens beslutningsrelevans opstår eller brydes.
