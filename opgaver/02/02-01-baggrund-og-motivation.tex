\subsection{Baggrund og motivation}
\label{subsec:baggrund-og-motivation}

Globalt bidrager sundhedssektoren med ca. 4,4~\% af de samlede netto-udledninger, svarende til omtrent 2~Gt CO$_2$e. Mere end halvdelen af klimaaftrykket kommer fra energiforbrug, og udledningerne fordeler sig omtrent som 17~\% scope~1, 12~\% scope~2 og 71~\% scope~3, hvilket understreger sektorens afh\ae ngighed af indirekte udledninger i forsyningsk\ae der og drift.\parencite{HCWH2019ClimateFootprint}

Affaldsprofilen er en anden central dimension. Omtrent 85~\% af sundhedssektorens affald klassificeres som ikke-farligt, mens ca. 15~\% er farligt (infekti\o st, kemisk eller radioaktivt), hvilket skaber s\ae rlige krav til bortskaffelse og dokumentation.\parencite{WHO2024HealthCareWaste}

Den sociale dimension er ligeledes betydningsfuld. Under COVID-19 estimerer WHO mindst 115.500 d\o dsfald blandt health and care workers over 18 m\aa neder, hvilket peger p\aa arbejdsvilk\aa r og sikkerhed som en integreret del af \ESG-arbejdet i sektoren.\parencite{WHO2021HealthWorkerDeaths}

Samtidig sk\ae rper de regulatoriske rammer kravene til rapportering. \CSRD og ESRS indf\o rer standardiserede krav og digital rapportering, EU-taksonomien kr\ae ver dokumentation for alignment, og GRI fungerer som et udbredt frivilligt rammev\ae rk for sammenlignelighed.\parencite{EU2022CSRD,EU2023ESRS,EU2020Taxonomy,GRI2021}

Kombinationen af et dokumenteret klima- og affaldsaftryk, sociale risici og sk\ae rpet regulering skaber et pres for systematisk og sporbar \ESG-rapportering. For \SMV'er i sundhedssektoren betyder det, at rapporteringsopgaven skal l\o ses under ressourcebegr\ae nsninger, hvilket motiverer en unders\o gelse af forenklede processer og \ESG-as-a-Service som et operationaliserende svar.
