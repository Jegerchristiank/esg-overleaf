\subsection{Baggrund og motivation}
\label{subsec:baggrund-og-motivation}

Globalt bidrager sundhedssektoren med ca. 4,4~\% af de samlede netto-udledninger, svarende til omtrent 2~Gt CO$_2$e. Mere end halvdelen af klimaaftrykket kommer fra energiforbrug, og udledningerne fordeler sig omtrent som 17~\% scope~1, 12~\% scope~2 og 71~\% scope~3, hvilket understreger sektorens afhængighed af indirekte udledninger i forsyningskæder og drift.\parencite{HCWH2019ClimateFootprint}

Affaldsprofilen er en anden central dimension. Omtrent 85~\% af sundhedssektorens affald klassificeres som ikke-farligt, mens ca. 15~\% er farligt (infektiøst, kemisk eller radioaktivt), hvilket skaber særlige krav til bortskaffelse og dokumentation.\parencite{WHO2024HealthCareWaste}

Den sociale dimension er ligeledes betydningsfuld. Under COVID-19 estimerer WHO mindst 115.500 dødsfald blandt sundheds- og omsorgspersonale over 18 måneder, hvilket peger på arbejdsvilkår og sikkerhed som en integreret del af ESG-arbejdet i sektoren.\parencite{WHO2021HealthWorkerDeaths}

Samtidig skærper de regulatoriske rammer kravene til rapportering. CSRD og ESRS indfører standardiserede krav og digital rapportering, EU-taksonomien kræver dokumentation for overensstemmelse med taksonomien, og GRI fungerer som et udbredt frivilligt rammeværk for sammenlignelighed.\parencite{EU2022CSRD,EU2023ESRS,EU2020Taxonomy,GRI2021}

Kombinationen af et dokumenteret klima- og affaldsaftryk, sociale risici og skærpet regulering skaber et pres for systematisk og sporbar ESG-rapportering. For SMV'er i sundhedssektoren betyder det, at rapporteringsopgaven skal løses under ressourcebegrænsninger, hvilket motiverer en undersøgelse af forenklede processer og ESG-as-a-Service som et operationaliserende svar.

Baggrunden afgrænser dermed den centrale kløft mellem krav og kapacitet, som formål, forskningsspørgsmål og problemformulering præciserer i de følgende afsnit.
