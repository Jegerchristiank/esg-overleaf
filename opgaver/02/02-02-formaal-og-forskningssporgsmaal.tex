\subsection{Formål og forskningsspørgsmål}
\label{subsec:formaal-og-forskningssporgsmaal}

Formålet med rapporten er at analysere ESG-rapportering i sundhedssektoren med særligt fokus på regulatoriske krav, organisatorisk implementering og økonomisk relevans for SMV'er. Rapporten undersøger samtidig, hvordan et ESG-as-a-Service-koncept og en MVP kan omsætte krav til konkret dataindsamling, sporbarhed og rapportoutput.

Rapporten besvarer følgende forskningsspørgsmål:
\begin{enumerate}
  \item Hvordan påvirker CSRD, ESRS, EU-taksonomien og GRI kravene til ESG-rapportering i sundhedssektoren, især for SMV'er?
  \item Hvilket værdiforslag skaber ESG-as-a-Service i forhold til standardisering, governance og compliance?
  \item Hvordan kan en MVP omsætte regulatoriske krav til dataindsamling, sporbarhed og rapportoutput?
\end{enumerate}

Spørgsmålene besvares gennem analysen i \ref{sec:analyse}, diskuteres i \ref{sec:diskussion} og sammenfattes i \ref{sec:konklusion}.
