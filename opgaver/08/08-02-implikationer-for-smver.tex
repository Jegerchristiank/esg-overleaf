\subsection{Implikationer for SMV'er}
\label{subsec:implikationer-for-smver}

Ifølge \textcite{PwC2025GuideESGSMV} bliver ESG-rapportering et vedvarende krav for SMV'er, selv når virksomhederne ikke er direkte omfattet af CSRD. Kravene kommer i stigende grad via kunder, banker og forsyningskæder, hvilket gør frivillig rapportering til en praktisk nødvendighed.\parencite{Virksomhedsguiden2025SMVDefinition}

VSME-standarden tilbyder en lavere adgangsbarriere og reducerer kompleksiteten ved at opstille et basismodul med begrænset datakrav og uden krav om dobbelt væsentlighed.\parencite{Virksomhedsguiden2025VSMEIntro,Virksomhedsguiden2025VSMEModules,Virksomhedsguiden2025NoDualMateriality} EY's evaluering viser samtidig, at mange virksomheder vælger at rapportere ud over minimumsniveauet, hvilket indikerer en strategisk anvendelse af ESG-data snarere end ren compliance.\parencite{EY2025VSME}

VSME kan være tilstrækkelig som minimumsniveau, men den kan også vise sig utilstrækkelig, når kunder, banker eller offentlige aktører kræver mere detaljeret dokumentation. Det kan skabe dobbeltarbejde, hvis SMV'er både skal opfylde VSME og mere avancerede krav, og dermed øge byrden frem for at reducere den.

Implikationen for SMV'er er derfor todelt. På kort sigt er der behov for forenklede arbejdsgange og teknisk støtte, som ESG-as-a-Service kan levere. På lang sigt er der behov for organisatorisk læring og datakapacitet, så ESG-arbejdet kan integreres i driften og skabe beslutningsrelevante indsigter. Stop-the-clock-aftalen giver ekstra tid, men nedsætter ikke markedets efterspørgsel efter dokumentation, hvilket betyder, at udsættelse ikke bør tolkes som en pause i implementeringen.\parencite{Erhvervsstyrelsen2025StopClock}

Samlet set understøtter diskussionen, at SMV'er bør anvende standarder og digitale løsninger som en gradvis overgangsstrategi: start med minimumsmoduler, dokumenter sporbarhed, og udvid derefter til mere komplekse krav, når datakvalitet og ressourcer tillader det.
