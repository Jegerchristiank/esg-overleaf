\subsection{Sammenhæng mellem teori, empiri og software}
\label{subsec:sammenhaeng-mellem-teori-empiri-og-software}

Analysen peger på en grundlæggende konsistens mellem teori, empiri og MVP. Brunssons standardiseringsperspektiv beskriver, hvordan standarder skaber legitimitet og sammenlignelighed, men også risiko for dekobling mellem rapportering og praksis.\parencite{BrunssonWorldOfStandards} MVP'en adresserer denne risiko ved at indbygge auditspor, validering og beregningsspor, som gør data efterprøvbare og reducerer symbolsk rapportering (bilag \ref{app:teknisk-dokumentation}). Det understøtter den teoretiske antagelse om, at standardisering kræver konkrete infrastrukturer for at få organisatorisk effekt.

Pensum om translasjon understreger, at ideer altid oversættes til lokale praksisser.\parencite{RoevikTrenderTranslasjoner,TranslationTheoryKnowledgeTransfer} Det genfindes i casen, hvor MVP'en prioriterer et begrænset sæt af indikatorer (energi, affald og sociale KPI'er) for at gøre rapporteringen gennemførbar for SMV'er. Dette bekræfter, at implementering ikke er en direkte kopi af ESRS/GRI, men en kontekstualiseret oversættelse.

Samtidig peger empirien på afvigelser, hvor implementering i praksis ofte drives af kundekrav og ressourcebegrænsninger snarere end af standardlogik alene. Det kan betyde, at rapporteringen bliver mere compliance-orienteret end den teoretiske ambition om strategisk værdiskabelse.

Empirien om sundhedssektorens klimaaftryk og arbejdsvilkår viser, at centrale ESG-dimensioner ligger i scope~3, affaldsstrømme og sociale risici.\parencite{HCWH2019ClimateFootprint,WHO2024HealthCareWaste,WHO2021HealthWorkerDeaths} MVP'ens fokus på sporbar dataindsamling og strukturerede outputs matcher disse behov og fungerer som et praktisk svar på sektorens dokumentationskrav.\parencite{Sepetis2024Healthcare,Bosco2024ESGHealth}

Samtidig viser pensum om ESG, at mangel på standardisering og datakvalitet er en tilbagevendende barriere for værdiskabelse.\parencite{ESGBook} ESG-ratinger divergerer og understreger, at standarder endnu ikke skaber fuld sammenlignelighed på tværs af aktører.\parencite{Berg2022AggregateConfusion} Det betyder, at MVP'en bør ses som et skridt mod standardisering, men ikke som en garanti for fuld konsensus eller endelig validitet. Diskussionen peger derfor på, at software kan reducere friktion og skabe bedre datakvalitet, men at organisatorisk forankring og governance stadig er afgørende for at undgå dekobling.

\subsubsection{Minimum compliance versus beslutningsrelevant ESG}
\label{subsec:minimum-compliance-vs-beslutningsrelevant-esg}

Den analytiske spændvidde i ESG-as-a-Service kan tydeliggøres ved at skelne mellem minimum compliance og beslutningsrelevant ESG. Distinktionen er central, fordi den markerer, hvornår rapportering bliver styringsinformation frem for dokumentation. Tabel \ref{tab:compliance-vs-beslutningsrelevant} opsummerer forskelle i formål, datakrav og organisatorisk anvendelse.
\begin{table}[t]
  \caption{Kontrast, der viser hvorfor beslutningsrelevant ESG kræver mere end minimum compliance.}
  \label{tab:compliance-vs-beslutningsrelevant}
  \begin{tabularx}{\textwidth}{l X X}
    \toprule
    Dimension & Minimum compliance & Beslutningsrelevant ESG \\
    \midrule
    Formål & Opfylde eksterne krav og dokumentation. & Understøtte strategiske beslutninger og værdiskabelse. \\
    Datakrav & Minimumsdatapunkter og lav granuleringsgrad. & Højere granularitet, sporbarhed og sammenhæng til KPI'er. \\
    Proces & Periodisk rapportering med fokus på kontrol. & Løbende styring og integration i driftsprocesser. \\
    Output & Rapport og dokumentation til myndigheder/kunder. & Ledelsesinformation, risikostyring og forbedringstiltag. \\
    \bottomrule
  \end{tabularx}
  \TableSource{Egen fremstilling baseret på teori og case.}
\end{table}

Normativt positioneres rapporten således: Minimum compliance er utilstrækkelig governance i sundhedssektoren, når ESG-data bruges til risikostyring, prioritering og ansvarlighed. ``God nok'' ESG er derfor dårlig governance, når den reducerer ESG til efterlevelse uden læring, prioritering eller beslutningsrelevans.

Troværdig ESG-rapportering kræver derfor både tekniske kontroller og organisatoriske rutiner, der sikrer, at data faktisk afspejler praksis og ikke kun formel efterlevelse.
