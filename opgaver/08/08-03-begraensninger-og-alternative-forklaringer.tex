\subsection{Begrænsninger og alternative forklaringer}
\label{subsec:begraensninger-og-alternative-forklaringer}

Diskussionens begrænsninger afspejler metodens karakter som casebaseret og konceptuel analyse. Casen er egenproduceret, og MVP'en er ikke afprøvet i flere organisationer. Det betyder, at konklusionerne primært er analytiske og bør valideres i fremtidige empiriske studier.

Der findes flere alternative forklaringer på de observerede sammenhænge. For det første kan motivationen for ESG-rapportering være drevet af generel digitaliseringsmodenhed snarere end af standardiseringslogikken i sig selv. For det andet kan implementering handle mere om efterlevelse af kundekrav i forsyningskæden end om strategisk værdiskabelse, hvilket reducerer muligheden for at udlede økonomiske effekter af ESG-arbejdet.\parencite{Berg2022AggregateConfusion}

En metodisk begrænsning er antagelserne i MVP'en om datatilgængelighed og standardiserede inputformater. Hvis data er mere fragmenterede end antaget, kan beregninger og rapportoutput blive mindre valide, hvilket påvirker konklusionernes styrke.

Regulatorisk usikkerhed er en tredje forklaring. CSRD, ESRS og EU-taksonomien er under løbende justering, og stop-the-clock-aftalen kan skabe midlertidige tilpasninger i virksomhedernes prioriteringer.\parencite{EU2022CSRD,EU2023ESRS,EU2020Taxonomy,Erhvervsstyrelsen2025StopClock} Effekter, der i analysen tolkes som implementeringsbarrierer, kan derfor delvist skyldes timing og regulatorisk bevægelse.

Endelig kan dataforskelle og rating-divergens betyde, at selv standardiseret rapportering ikke skaber entydige evalueringer på tværs af interessenter.\parencite{Berg2022AggregateConfusion} Dette understreger, at diskussionens konklusioner skal tolkes med forbehold for datakvalitet, fortolkningsrum og aktørers forskellige anvendelse af ESG-information. Yderligere empirisk test bør derfor omfatte flere cases, kvantitative data og brugeradoption over tid.
