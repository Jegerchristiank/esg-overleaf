\section{Diskussion}
\label{sec:diskussion}

\begin{quote}
Diskussionen afvejer evidens, begrænsninger og implikationer for, hvad der faktisk kan konkluderes.
\end{quote}

Diskussionen tager afsæt i tre hovedfund: (1) reguleringen skaber et vedvarende behov for standardiserede data og sporbarhed, (2) servicekonceptet og MVP'en kan operationalisere centrale krav, og (3) organisatorisk forankring er afgørende for at undgå symbolsk rapportering. På den baggrund vurderes, hvorvidt løsningen faktisk understøtter de teoretiske antagelser om standardisering, governance og værdiskabelse.

Evidensgrundlaget er stærkt for de regulatoriske krav og de teoretiske perspektiver, men mere begrænset empirisk, fordi sektordata er aggregerede og casen er single-case. Det betyder, at alternative forklaringer og implikationer for SMV'er må vurderes kritisk, og at konklusionerne bør forstås som analytiske snarere end generaliserbare.

I \ref{subsec:sammenhaeng-mellem-teori-empiri-og-software} diskuteres sammenhængen mellem teori, empiri og software. \ref{subsec:implikationer-for-smver} behandler konsekvenser og handlemuligheder for SMV'er, mens \ref{subsec:begraensninger-og-alternative-forklaringer} redegør for metodiske begrænsninger og mulige alternative forklaringer. Diskussionen nuancerer dermed svarene på forskningsspørgsmålene ved at afveje evidensstyrke, praktiske implikationer og usikkerheder.

Diskussionens afvejninger leder frem mod konklusionens endelige svar og prioritering af bidrag.
