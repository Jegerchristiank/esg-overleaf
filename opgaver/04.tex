\section{Teoretisk ramme}
\label{sec:teoretisk-ramme}

Den teoretiske ramme kombinerer tre perspektiver til at analysere ESG-rapportering i sundhedssektoren og ESG-as-a-Service som svar på regulatoriske krav. Valget af teori er styret af tre behov: at forstå ESG som standardiseringspraksis, at forklare hvordan policy omsættes til organisatorisk handling, og at vurdere økonomisk relevans og kommercialiseringsmuligheder.

Rammen integrerer (1) standardisering og organisationer med afsæt i Brunsson, (2) policy- og governanceperspektiver med fokus på translasion og institutionel styring, og (3) økonomiske og kommercielle perspektiver på værdiskabelse og performance. Tilsammen udgør de et analysekatalog, der bruges til at fortolke empirien og vurdere MVP'ens rolle.

Rammen er valgt for at kunne vurdere, om ESG-as-a-Service faktisk mindsker kløften mellem regulatoriske krav og praktisk datapraksis.
\subsection{Standardisering og organisationer (Brunsson)}
\label{subsec:standardisering-og-organisationer-brunsson}

Ifølge \textcite{BrunssonWorldOfStandards} er standarder organiserede regler og forventninger, der skaber ensartethed og sammenlignelighed på tværs af organisationer. Standardisering er ikke kun teknisk, men også institutionel: den etablerer, hvad der betragtes som legitim praksis, og skaber et fælles sprog for kontrol, måling og rapportering.
\begin{displayquote}
Standardization is a fundamental form for governance and co-ordination in societies.
\quoteattrib{\textcite{BrunssonWorldOfStandards}}
\end{displayquote}

Standarder kan samtidig fungere som styring på afstand. De gør det muligt at koordinere aktører, der ikke deler samme kontekst, men de skaber også risiko for, at organisationer fokuserer på formel overholdelse frem for substantiel forandring. Som \textcite{BrunssonWorldOfStandards} formulerer det: \enquote{The mere existence of a standard does not guarantee that it will be followed.} Det åbner for dekobling mellem det, der rapporteres, og det, der faktisk gøres i praksis. ESG-ratingdivergens dokumenteret af \textcite{Berg2022AggregateConfusion} understøtter, at standardisering ikke automatisk giver konsensus om kvalitet eller performance.

I ESG-rapportering betyder det, at standarder definerer, hvilke indikatorer og narrativer der opfattes som gyldig dokumentation. Det fremmer auditabilitet og sammenlignelighed, men kan også reducere rapportering til et compliance-projekt. For SMV'er er standardisering derfor både en forudsætning for legitim rapportering og en byrde, der kræver ressourcer, struktur og data. ESG-as-a-Service kan ses som en standardiseringsinfrastruktur, der omsætter krav til konkrete datafelter og arbejdsgange, men den kan også forstærke fokus på minimumskrav frem for strategisk læring. Analytisk peger dette på at vurdere graden af faktisk praksisændring versus symbolsk efterlevelse.

Implementering af standarder forudsætter governance og translasion, fordi krav omsættes lokalt gennem institutionelle mekanismer og organisatoriske fortolkninger.
\subsection{Policy- og governanceperspektiver}
\label{subsec:policy-og-governanceperspektiver}

Policy- og governanceperspektiver fokuserer på, hvordan regulering omsættes til praksis gennem institutionelle mekanismer, standarder og mellemled. ESG-rapportering i EU er et eksempel på flerniveau-styring, hvor politiske målsætninger realiseres gennem direktiver, standarder og vejledninger, som organisationer skal fortolke og implementere lokalt.

\textcite{RoevikTrenderTranslasjoner} beskriver, hvordan ideer og styringskoncepter vandrer mellem organisationer og bliver oversat til lokale praksisser. I hans translasionsteori betones, at implementering ikke er en ren kopiering, men en proces, hvor krav redigeres, omformuleres og tilpasses til organisatoriske betingelser.\parencite{TranslationTheoryKnowledgeTransfer} Det betyder, at ensartede standarder kan skabe variation i praksis, afhængigt af aktørers kapacitet, fortolkning og incitamenter.
\begin{displayquote}
Uthenting, overføring og mottak av organisasjonsideer kan forstås som en form for oversettelse.
\par Dansk oversættelse: Udtagning, overførsel og modtagelse af organisationsideer kan forstås som en form for oversættelse af idéer.
\quoteattrib{\textcite{RoevikTrenderTranslasjoner}}
\end{displayquote}

I ESG-rapportering fungerer revisorer, konsulenter og softwareleverandører som intermediære aktører, der oversætter policy til datafelter, processer og beslutningsregler. Disse aktører bidrager til governance ved at definere, hvad der opfattes som tilstrækkelig dokumentation, og ved at skabe rammer for sporbarhed. Ifølge \textcite{TranslationTheoryKnowledgeTransfer} er \enquote{Modtagere er ikke passive modtagere, men fortolkende systemer}, hvilket understreger, at governance altid filtreres gennem lokale fortolkninger.

Analytisk betyder perspektivet, at analysen undersøger, hvordan regulatoriske krav transformeres til operationelle krav i sundhedssektoren, og hvordan ESG-as-a-Service fungerer som en oversættelsesmekanisme mellem policy og praksis. Det giver et grundlag for at vurdere, hvorvidt governance styrker faktisk implementering eller primært producerer formel overholdelse.

Perspektivet omsættes til følgende analytiske spørgsmål i analysen:
\begin{itemize}
  \item Hvilke mellemled oversætter regulatoriske krav til konkrete datafelter og processer?
  \item Hvor opstår de centrale fortolkninger og redigeringer af standarder i praksis?
  \item Hvilke governance-mekanismer skaber sporbarhed og reducerer risikoen for formel efterlevelse uden praksisændring?
\end{itemize}

Økonomisk værdi og bæredygtige forretningsmodeller vurderes gennem perspektiver på værdiskabelse, performance og betalingsvillighed.
\subsection{Økonomiske og kommercielle perspektiver}
\label{subsec:oekonomiske-og-kommercielle-perspektiver}

Det økonomiske perspektiv starter med spørgsmålet om virksomhedens formål. \textcite{Friedman1970Profits} argumenterer for, at virksomhedens primære ansvar er at maksimere profit inden for lovens rammer, mens stakeholder-tilgangen betoner ansvar over for flere interessenter end aktionærerne.\parencite{Freeman1984Stakeholder} \textcite{Carroll1991Pyramid} uddyber dette gennem en CSR-pyramide, hvor økonomiske, juridiske, etiske og filantropiske hensyn kombineres i virksomhedens ansvar. Business Roundtable's 2019-erklæring signalerer et skifte mod stakeholder-ansvar, men kritik peger på, at sådanne forpligtelser kan være symbolske uden stærk governance og målelige mål.\parencite{BusinessRoundtable2019,Economist2022Broken}

I et bredere værdiskabelsesperspektiv fremhæver triple bottom line, at bæredygtighed skal vurderes på tværs af økonomi, miljø og sociale forhold.\parencite{Elkington1998TripleBottomLine} \textcite{PorterKramer2011CSV} argumenterer for, at virksomheder kan skabe shared value ved at koble samfundsmæssige udfordringer til forretningsstrategi og innovation. Disse perspektiver tilbyder et begrebsligt fundament for at vurdere, om ESG-indsatser kan skabe varig værdi frem for at være rene omkostninger.

Empirisk peger studier på en overvejende positiv eller neutral sammenhæng mellem ESG og finansiel performance, men effekten varierer med kontekst og tidsperiode.\parencite{Eccles2014Impact,Friede2015Meta} \textcite{Khan2016Materiality} viser, at finansielt materielle ESG-temaer er forbundet med bedre performance, mens immaterielle temaer ikke har samme effekt. Det peger på, at værdien af ESG-arbejde afhænger af strategisk fokus og relevans.

\begin{displayquote}
ESG is a means, not an end.
\quoteattrib{\textcite{ESGBook}}
\end{displayquote}

Det betyder, at ESG-as-a-Service ikke kun skal vurderes på compliance, men også på om løsningen reducerer risici, forbedrer beslutningsgrundlag og skaber økonomisk nytte for SMV'er. Det informerer vurderingen af betalingsvillighed, prisstruktur og den kommercielle bæredygtighed af servicekonceptet.

Perspektiverne definerer tilsammen de centrale kriterier for analysen.
\subsection{Opsamling af den teoretiske ramme}
\label{subsec:opsamling-af-den-teoretiske-ramme}

Den teoretiske ramme samler tre komplementære perspektiver, der tilsammen forklarer, hvorfor ESG-rapportering opfattes som nødvendig, hvordan den implementeres, og hvilke økonomiske konsekvenser den kan have. Standardisering bidrager med begreber om legitimitet og sammenlignelighed, governance forklarer oversættelse af policy til praksis, og økonomiske perspektiver adresserer værdiskabelse og betalingsvillighed.

De centrale begreber og deres analytiske anvendelse fremgår af tabel \ref{tab:teoretisk-ramme-overblik}.
\begin{table}[h]
  \caption{Teoretiske linser, der definerer analysens vurderingskriterier.}
  \label{tab:teoretisk-ramme-overblik}
  \begin{tabularx}{\textwidth}{l X X}
    \toprule
    Perspektiv & Kernebegreber & Analytisk anvendelse \\
    \midrule
    Standardisering & Legitimitet, sammenlignelighed, dekobling. & Vurderer hvordan ESG-krav omsættes til standardiserede data og om rapportering bliver substans eller symbol. \\
    Policy og governance & Translasion, intermediære aktører, flerniveau-styring. & Forklarer hvordan regulering og standarder oversættes til lokale processer og tekniske arbejdsgange. \\
    Økonomi og kommercialisering & Stakeholder/shareholder, shared value, finansiel materialitet. & Vurderer økonomisk relevans, forretningsmodel og betalingsvillighed for ESG-as-a-Service. \\
    \bottomrule
  \end{tabularx}
  \TableSource{Egen fremstilling.}
\end{table}

Analysen tager især afsæt i to spændinger: (1) standardiseringens løfte om legitimitet versus risikoen for dekobling, og (2) compliance-logik versus beslutningsrelevant ESG som styringsgrundlag. Det skaber en eksplicit ramme for at vurdere, om ESG-as-a-Service omsætter krav til praksis eller primært producerer dokumentation.

Opsamlingen fungerer som analytisk ramme for vurderingen af ESG-rapporteringens praksis og MVP'ens kommercielle og organisatoriske implikationer.
