\section{Konklusion}
\label{sec:konklusion}

Analysen peger på, at ESG-rapportering i sundhedssektoren er et styringsproblem under \textit{standardiseringspres}, hvor data og ansvar er spredt i værdikæden. ESG-as-a-Service opstår primært som svar på et EU-retligt flerniveau-regime og sekundært som et strategisk bæredygtighedsværktøj, hvilket gør beslutningsrelevans vigtigere end \textit{minimum compliance}. MVP'en indikerer gennemførbarhed og \textit{sporbarhed}, men peger samtidig på \textit{governance} og datakvalitet som de afgørende flaskehalse.

Konklusionen gør to ting: den besvarer forskningsspørgsmålene og udpeger rapportens hovedbidrag og afgrænsninger.

\subsection{Svar på forskningsspørgsmål}
\label{subsec:svar-forskningssporgsmaal}

\subsubsection{RQ1: Hvordan påvirker CSRD, ESRS, EU-taksonomien og GRI kravene til ESG-rapportering i sundhedssektoren?}
CSRD og ESRS fastlægger standardiserede krav til data, væsentlighed og dokumentation, mens EU-taksonomien og GRI supplerer med klassifikationer og indikatorlogik.\parenciteshort{EU2022CSRD,EU2023ESRS,EU2020Taxonomy,GRI2021} For SMV'er sker påvirkningen primært indirekte via forsyningskæder og finansielle aktører, mens VSME giver et forenklet minimumsniveau.\parencite{Virksomhedsguiden2025VSMEIntro,Virksomhedsguiden2025VSMEModules,Virksomhedsguiden2025NoDualMateriality} Evidens-anker: EU-hierarkiet og danske kontrolpunkter (tabel \ref{tab:eu-retligt-hierarki}, \ref{tab:danske-kontrolpunkter}).

Evaluativt betyder det, at kravene ikke blot er juridiske, men bliver operative gennem danske kontrolgrænseflader og finansielle aktører, hvilket gør reguleringen til et \textit{governance}-spørgsmål i driften. Nødvendig evidens: sammenhæng mellem retskilder, danske kontrolpunkter og konkrete data-/auditkrav (tabel \ref{tab:juridisk-til-data}). Begrænsning: graden af indirekte pres varierer på tværs af brancher og kreditprofiler og kræver flercase-dokumentation.

\subsubsection{RQ2: Hvilket værdiforslag skaber ESG-as-a-Service i forhold til standardisering, governance og compliance?}
ESG-as-a-Service skaber værdiforslag ved at oversætte standarder til konkrete arbejdsgange, reducere transaktionsomkostninger og styrke \textit{sporbarhed} og \textit{governance}. Værdien afhænger af fokus på \textit{materielle} temaer og organisatorisk forankring, så rapportering ikke reduceres til \textit{symbolsk compliance}.\parencite{Khan2016Materiality} \textit{Standardisering} og \textit{translasjon} viser, at oversættelse til praksis er afgørende for governance.\parencite{BrunssonWorldOfStandards,RoevikTrenderTranslasjoner,TranslationTheoryKnowledgeTransfer} Evidens-anker: governance- og værdiforslagsanalysen (tabel \ref{tab:kontrolfordeling}, \ref{tab:compliance-vs-beslutningsrelevant}).

Evaluativt betyder det, at værdiskabelse først opstår, når \textit{auditspor} og rollefordeling gør data beslutningsrelevante, ikke kun compliance-gyldige. Nødvendig evidens: dokumenteret rollefordeling og \textit{sporbarhed} i dataflow (tabel \ref{tab:kontrolfordeling}, \ref{tab:traceability-evaluering}). Begrænsning: værdiskabelse afhænger af implementeringsmodenhed og kan ikke generaliseres uden flercase-data.

\subsubsection{RQ3: Hvordan kan en MVP omsætte regulatoriske krav til dataindsamling, sporbarhed og rapportoutput?}
MVP'en kan omsætte krav til praksis gennem modulbaseret dataindsamling, validering, beregning og rapportoutput kombineret med \textit{auditspor}. Den indikerer gennemførbarhed for SMV'er, men også behov for videre integration og udvidet dækning (bilag \ref{app:teknisk-dokumentation}). Evidens-anker: MVP-krav og \textit{sporbarhedstests} (tabel \ref{tab:krav-prioritering}, \ref{tab:traceability-evaluering}, \ref{tab:supplerende-tests}).

Evaluativt betyder det, at MVP'en demonstrerer sporbarhed for de afgrænsede moduler, men ikke dokumenterer skalerbarhed eller iXBRL/XBRL-compliance. Nødvendig evidens: fuld traceability-matrix for MVP-scope (bilagstabel \ref{tab:traceability-matrix-full}) og testruns med auditfelter. Begrænsning: større datamængder og fuld digital indberetning kan ændre kravene til validering og auditlogik.

\textbf{Det centrale bidrag er at afgrænse, hvornår ESG-as-a-Service skaber beslutningsrelevant styring frem for symbolsk rapportering, og at tydeliggøre de organisatoriske betingelser, der skal være opfyldt for at undgå \textit{dekobling}.} Afslutningsvis er konklusionerne begrænset af case- og datagrundlag og bør valideres gennem flere empiriske studier.
