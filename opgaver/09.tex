\section{Konklusion}
\label{sec:konklusion}

Analysen peger på, at ESG-rapportering i sundhedssektoren er et styringsproblem under standardiseringspres, hvor data og ansvar er spredt i værdikæden. ESG-as-a-Service kan fungere som den nødvendige infrastruktur til at omsætte krav til praksis, men kun hvis beslutningsrelevans prioriteres over minimum compliance. MVP'en indikerer gennemførbarhed og sporbarhed, men peger samtidig på governance og datakvalitet som de afgørende flaskehalse.

Forskningsspørgsmålene kan besvares således:
\begin{enumerate}
  \item I \textcite{EU2022CSRD} og \textcite{EU2023ESRS} fastlægges standardiserede krav til data, væsentlighed og dokumentation. \textcite{EU2020Taxonomy} og \textcite{GRI2021} supplerer med klassifikationer og indikatorlogik. For SMV'er sker påvirkningen primært indirekte via forsyningskæder og finansielle aktører, mens \textcite{Virksomhedsguiden2025VSMEIntro}, \textcite{Virksomhedsguiden2025VSMEModules} og \textcite{Virksomhedsguiden2025NoDualMateriality} viser, at VSME giver et forenklet minimumsniveau.
  \item ESG-as-a-Service skaber værdiforslag ved at oversætte standarder til konkrete arbejdsgange, reducere transaktionsomkostninger og styrke sporbarhed og governance. \textcite{Khan2016Materiality} viser, at værdien afhænger af fokus på materielle temaer og af organisatorisk forankring, så rapportering ikke reduceres til symbolsk compliance. \textcite{BrunssonWorldOfStandards}, \textcite{RoevikTrenderTranslasjoner} og \textcite{TranslationTheoryKnowledgeTransfer} underbygger, at oversættelse til praksis er afgørende for governance.
  \item MVP'en kan omsætte krav til praksis gennem modulbaseret dataindsamling, validering, beregning og rapportoutput kombineret med auditspor. Den indikerer gennemførbarhed for SMV'er, men også behov for videre integration og udvidet dækning (bilag \ref{app:teknisk-dokumentation}).
\end{enumerate}

Det vigtigste bidrag er derfor at afgrænse, hvornår ESG-as-a-Service skaber beslutningsrelevant styring frem for symbolsk rapportering, og at tydeliggøre de organisatoriske betingelser, der skal være opfyldt for at undgå dekobling. Afslutningsvis er konklusionerne begrænset af case- og datagrundlag og bør valideres gennem flere empiriske studier.
