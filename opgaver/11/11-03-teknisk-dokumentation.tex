\subsection{Teknisk dokumentation}
\label{app:teknisk-dokumentation}

Dette bilag sammenfatter det interne case- og softwaremateriale, der ligger til grund for beskrivelsen af MVP'en i kapitel \ref{sec:software-og-mvp}. Materialet er ikke offentligt tilgængeligt og er derfor gengivet her for sporbarhed.

\paragraph{Systemstruktur} MVP'en er en webbaseret løsning med adskilt brugergrænseflade, applikationslogik og datalagring. Strukturen understøtter et kontrolleret flow fra input til rapportoutput og gør det muligt at udvide komponenter uden at bryde den samlede proceslogik.

\paragraph{Datamodel og sporbarhed} Data organiseres omkring profilafgrænsning, modulregistreringer og et auditspor. Modellen sikrer, at ændringer, perioder og beregningsgrundlag kan efterprøves, og at data kan følges tilbage til deres kilde.

\paragraph{Datakvalitet og kontrol} Input valideres mod faste regler, enheder og obligatoriske felter. Versionshistorik gør det muligt at dokumentere udvikling over tid og at skelne mellem primære input og beregnede resultater.

\paragraph{Rapportoutput} Output omfatter et læsbart dokument og strukturerede data, så resultater kan genbruges i compliance- og beslutningsprocesser på tværs af interessenter.

\paragraph{Antagelser og begrænsninger} Bilaget beskriver MVP'ens aktuelle scope og forudsætter tilgængelige datakilder i konsistente formater. Det dækker ikke fuld drifts- eller sikkerhedsarkitektur, som typisk specificeres i en senere produktionsfase.

\subsubsection{Software-specifikation (intern)}
\label{app:software-spec}

\VerbatimInput[fontsize=\footnotesize,breaklines=true]{softwarebeskrivelser/software-spec.md}

\subsubsection{Detaljeret software-specifikation (intern)}
\label{app:detaljeret-software-spec}

\VerbatimInput[fontsize=\footnotesize,breaklines=true]{softwarebeskrivelser/detaljeret-software-spec.txt}
