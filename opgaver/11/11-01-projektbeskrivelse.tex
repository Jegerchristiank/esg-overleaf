\subsection{Projektbeskrivelse}
\label{app:projektbeskrivelse}

Bilaget relaterer til problemformuleringen i \ref{subsec:problemformulering} ved at præcisere kursens fokus på ESG-rapportering som serviceydelse og de organisatoriske og økonomiske implikationer, som rapporten analyserer.

\textbf{Kursets titel:} ESG-rapportering i sundhedssektoren: Policy, organisering, økonomisk relevans og kommercialisering.

\textbf{English title:} ESG reporting in the healthcare sector: Policy, organization, economic relevance, and commercialization.

\textbf{Kursets faglige indhold:} Kurset undersøger ESG-rapporteringens rolle og betydning i sundhedssektoren, og hvordan rapporteringen kan udvikles til en selvstændig serviceydelse (ESG-as-a-Service) i en startup-kontekst. Der lægges vægt på:
\begin{itemize}
  \item Relevans, lovkrav og strategiske gevinster ved ESG-rapportering for sundhedsorganisationer og deres leverandørkæder.
  \item Nationale og EU-rammer (bl.a. CSRD, EU-taksonomien, GRI-standarderne).
  \item Udvikling af forretningsmodeller for ESG-rapportering som service, herunder prisfastsættelse, værdiforslag og kundesegmentering.
  \item Digitalisering og automatisering af ESG-dataindsamling og -rapportering (softwareplatforme og data-API'er).
  \item Casestudie: Den studerendes egen virksomhed som pilot for udvikling af en ESG-rapporteringstjeneste til eksterne kunder.
\end{itemize}

Kurset kulminerer i en skriftlig rapport, der kombinerer teori, praksis, analyse og konkrete anbefalinger til både sundhedssektoren og den studerendes virksomhed.

\textbf{Kursets formål:} At give den studerende en sundhedsvidenskabeligt forankret forståelse af ESG-rapportering og at kunne analysere, vurdere og kommercialisere dens organisatoriske, politiske og økonomiske implikationer. Dette gælder både internt i sundhedsorganisationer og som ekstern serviceydelse leveret af den studerendes egen startup.

\textbf{Kursets omfang i ECTS:} 15 ECTS. Cirka 375 timer total (25 timer pr. ECTS), cirka 275 timers forberedelse/læsning, cirka 100 timer på skriftlig aflevering.

\textbf{Supplerende målbeskrivelse:}

\textbf{Viden:}
\begin{itemize}
  \item Redegøre for ESG-rapporteringens betydning i en sundhedssektor-kontekst.
  \item Forklare centrale lovrammer (CSRD, GRI, EU-taksonomien) og markedsforventninger.
  \item Beskrive forretningsmæssige aspekter ved at tilbyde ESG-rapportering som service.
  \item Anvende Brunsson og andre teoretikere til at forstå ESG som eksempel på en organisationsstandard.
\end{itemize}

\textbf{Færdigheder:}
\begin{itemize}
  \item Analysere sundhedspolitiske, organisatoriske og økonomiske implikationer af ESG-rapportering.
  \item Udarbejde et koncept og en MVP for en ESG-rapporteringstjeneste målrettet sundhedssektorens aktører.
  \item Redegøre for digitale værktøjer til dataindsamling, analyse og visualisering af ESG-nøgletal.
\end{itemize}

\textbf{Kompetencer:}
\begin{itemize}
  \item Gennemføre en tværfaglig analyse i spændingsfeltet mellem sundhedsvidenskab, bæredygtighed og forretningsudvikling.
  \item Skabe en selvstændig, markedsorienteret service baseret på ESG-rapportering og implementere denne i egen virksomhed.
  \item Reflektere kritisk over etiske, compliance-mæssige og økonomiske aspekter ved ESG-data og -rapportering.
  \item Reflektere kritisk over ESG som organisationsstandard, herunder styringsmæssige og institutionelle implikationer.
\end{itemize}

\textbf{Form for produkt:} Skriftlig rapport (maks. 40 sider).

\textbf{Supplerende karakterbeskrivelse:} Et bestået projekt demonstrerer:
\begin{itemize}
  \item Klar struktur, metodisk stringens og faglig dybde.
  \item Velunderbygget forretningsmodel for ESG-rapportering som service.
  \item Selvstændig vurdering af ESG's betydning for sundhedsorganisationer samt startup-markedsmuligheder.
  \item Refleksion over økonomisk, politisk og etisk kontekst.
\end{itemize}

\textbf{Sprog:} Dansk.

\textbf{Prøveform:} Skriftlig rapport.

\textbf{Beståelsesform:} Bestået / Ikke bestået.

\par{\small\textit{Kilde: biblen-projektbeskrivelse.txt}}
