\section{Bilag}
\label{app:bilag}

Bilagene understøtter sporbarhed og dokumentation for undersøgelsens metode, software og projektkontekst. Bilagene gør tre ting: de dokumenterer projektgrundlaget, samler supplerende figurer og tabeller og uddyber den tekniske dokumentation. Bilag \ref{app:projektbeskrivelse} indeholder den formelle projektbeskrivelse. Bilag \ref{app:supplerende-figurer-og-tabeller} samler supplerende figurer og tabeller, som refereres fra hovedteksten, og bilag \ref{app:teknisk-dokumentation} uddyber den tekniske dokumentation. Bilagsdokumentet er et lukket evidenssystem med indholdsfortegnelse og stabile labels, så alle henvisninger kan følges uden eksterne filer. Hvis programkode eller detaljeret software-specifikation vedlægges som separate filer, er det supplerende materiale; bilaget giver den offentlige oversigt, som er en del af bedømmelsesgrundlaget.
\subsection{Projektbeskrivelse}
\label{app:projektbeskrivelse}

Bilaget relaterer til problemformuleringen i \ref{subsec:problemformulering} ved at præcisere kursens fokus på ESG-rapportering som serviceydelse og de organisatoriske og økonomiske implikationer, som analysen behandler.

\textbf{Kursets titel:} ESG-rapportering i sundhedssektoren: Policy, organisering, økonomisk relevans og kommercialisering.

\textbf{English title:} ESG reporting in the healthcare sector: Policy, organization, economic relevance, and commercialization.

\textbf{Kursets faglige indhold:} Kurset undersøger ESG-rapporteringens rolle og betydning i sundhedssektoren, og hvordan rapporteringen kan udvikles til en selvstændig serviceydelse (ESG-as-a-Service) i en startup-kontekst. Der lægges vægt på:
\begin{itemize}
  \item Relevans, lovkrav og strategiske gevinster ved ESG-rapportering for sundhedsorganisationer og deres leverandørkæder.
  \item Nationale og EU-rammer (bl.a. CSRD, EU-taksonomien, GRI-standarderne).
  \item Udvikling af forretningsmodeller for ESG-rapportering som service, herunder prisfastsættelse, værdiforslag og kundesegmentering.
  \item Digitalisering og automatisering af ESG-dataindsamling og -rapportering (softwareplatforme og data-API'er).
  \item Casestudie: Den studerendes egen virksomhed som pilot for udvikling af en ESG-rapporteringstjeneste til eksterne kunder.
\end{itemize}

Kurset kulminerer i en skriftlig rapport, der kombinerer teori, praksis, analyse og konkrete anbefalinger til både sundhedssektoren og den studerendes virksomhed.

\textbf{Kursets formål:} At give den studerende en sundhedsvidenskabeligt forankret forståelse af ESG-rapportering og at kunne analysere, vurdere og kommercialisere dens organisatoriske, politiske og økonomiske implikationer. Dette gælder både internt i sundhedsorganisationer og som ekstern serviceydelse leveret af den studerendes egen startup.

\textbf{Kursets omfang i ECTS:} 15 ECTS. Cirka 375 timer total (25 timer pr. ECTS), cirka 275 timers forberedelse/læsning, cirka 100 timer på skriftlig aflevering.

\textbf{Supplerende målbeskrivelse:}

\textbf{Viden:}
\begin{itemize}
  \item Redegøre for ESG-rapporteringens betydning i en sundhedssektor-kontekst.
  \item Forklare centrale lovrammer (CSRD, GRI, EU-taksonomien) og markedsforventninger.
  \item Beskrive forretningsmæssige aspekter ved at tilbyde ESG-rapportering som service.
  \item Anvende Brunsson og andre teoretikere til at forstå ESG som eksempel på en organisationsstandard.
\end{itemize}

\textbf{Færdigheder:}
\begin{itemize}
  \item Analysere sundhedspolitiske, organisatoriske og økonomiske implikationer af ESG-rapportering.
  \item Udarbejde et koncept og en MVP for en ESG-rapporteringstjeneste målrettet sundhedssektorens aktører.
  \item Redegøre for digitale værktøjer til dataindsamling, analyse og visualisering af ESG-nøgletal.
\end{itemize}

\textbf{Kompetencer:}
\begin{itemize}
  \item Gennemføre en tværfaglig analyse i spændingsfeltet mellem sundhedsvidenskab, bæredygtighed og forretningsudvikling.
  \item Skabe en selvstændig, markedsorienteret service baseret på ESG-rapportering og implementere denne i egen virksomhed.
  \item Reflektere kritisk over etiske, compliance-mæssige og økonomiske aspekter ved ESG-data og -rapportering.
  \item Reflektere kritisk over ESG som organisationsstandard, herunder styringsmæssige og institutionelle implikationer.
\end{itemize}

\textbf{Form for produkt:} Skriftlig rapport (maks. 40 sider).

\textbf{Supplerende karakterbeskrivelse:} Et bestået projekt demonstrerer:
\begin{itemize}
  \item Klar struktur, metodisk stringens og faglig dybde.
  \item Velunderbygget forretningsmodel for ESG-rapportering som service.
  \item Selvstændig vurdering af ESG's betydning for sundhedsorganisationer samt startup-markedsmuligheder.
  \item Refleksion over økonomisk, politisk og etisk kontekst.
\end{itemize}

\textbf{Sprog:} Dansk.

\textbf{Prøveform:} Skriftlig rapport.

\textbf{Beståelsesform:} Bestået / Ikke bestået.

\par{\small\textit{Kilde: biblen-projektbeskrivelse.txt}}
\subsection{Supplerende figurer og tabeller}
\label{app:supplerende-figurer-og-tabeller}

Dette bilag samler supplerende materiale, der refereres fra hovedteksten, men som ikke er nødvendigt at placere i de centrale kapitler. Moduloversigten i tabel \ref{tab:app-moduloversigt} og skærmbillederne dokumenterer den fulde brugerrejse fra profilafgrænsning til rapportoutput.

\begin{table}[h]
  \caption{Moduloversigt pr. scope i MVP'en.}
  \label{tab:app-moduloversigt}
  \rowcolors{2}{black!5}{white}
  \begin{tabularx}{\textwidth}{l X}
    \toprule
    Område & Moduler (oversigt) \\
    \midrule
    Scope 1 & A1--A4 (direkte emissioner og processer). \\
    Scope 2 & B1--B11 (energi og varme). \\
    Scope 3 & C1--C15 (værdikæde og affald). \\
    Socialt & S1--S4 (arbejdsmiljø og sociale forhold). \\
    Governance og strategisk & G1, D1--D2, SBM, IRO, MR. \\
    \bottomrule
  \end{tabularx}
  \TableSource{Egen fremstilling baseret på case-materiale.}
\end{table}

\begin{table}[h]
  \caption{Fuld traceability-matrice for MVP-scope (B2, C5, S1).}
  \label{tab:traceability-matrix-full}
  \begingroup
  \renewcommand{\arraystretch}{1.0}
  \footnotesize
  \setlength{\tabcolsep}{4pt}
  \rowcolors{2}{black!5}{white}
  \begin{tabularx}{\textwidth}{l >{\raggedright\arraybackslash}X >{\raggedright\arraybackslash}p{0.12\textwidth} >{\raggedright\arraybackslash}p{0.12\textwidth} >{\raggedright\arraybackslash}p{0.11\textwidth} >{\raggedright\arraybackslash}p{0.11\textwidth} >{\raggedright\arraybackslash}p{0.11\textwidth} >{\raggedright\arraybackslash}X}
    \toprule
    ID & Retskilde + reference & ESRS-datapunkt & Inputfelter & Valideringsregel & Beregningsregel & Outputfelt & Audit-evidens \\
    \midrule
    M1 & CSRD \lawart{19a}, \lawart{29a} (Dir. EU 2022/2464) & ESRS E1-5/E1-6 (energi og GHG) & El/varme (kWh), brændsel (L/kg), emissionsfaktorer & Enheder, periodelås, kilde angivet & CO$_2$e = aktivitet $\times$ EF; energisum pr. periode & Scope 1/2 CO$_2$e; energiforbrug pr. periode & Kilde-ID, EF-version, ændringslog, godkendelse af dataejer \\
    M2 & CSRD \lawart{19a} (bæredygtighedsoplysninger) & ESRS E5 (affald og ressourcebrug) & Affald pr. fraktion (kg/ton), behandlingsform & Ikke-negative mængder, fraktionsmatch & Summering pr. fraktion; evt. CO$_2$e = m $\times$ EF & Affald pr. fraktion; andel farligt affald & Vejesedler, leverandørdokumentation, godkendt af drift \\
    M3 & CSRD \lawart{19a} (sociale forhold) & ESRS S1 (arbejdsmiljø) & Antal hændelser, arbejdstimer, medarbejderantal & Konsistens med HR-data, periodelås & Frekvens = hændelser/arbejdstimer & Arbejdsmiljø-KPI'er & HR-log, hændelsesjournal, sign-off af HR \\
    \bottomrule
  \end{tabularx}
  \endgroup
  \TableSource{\parenciteshort{EU2022CSRD,EU2023ESRS,GHGProtocol2004}}
\end{table}

\begin{table}[h]
  \caption{Indikatoreksempler, der operationaliserer sektorens kernerisici i konkrete datapunkter.}
  \label{tab:esg-indikatorer}
  \rowcolors{2}{black!5}{white}
  \begin{tabularx}{\textwidth}{l X l X}
    \toprule
    Område & Eksempel på datafelt & Enhed & Typisk datakilde \\
    \midrule
    Energi & Varme- og elforbrug (ESRS E1; direkte CSRD/ESRS) & kWh & Drifts- og energidata. \\
    Affald & Mængde affald pr. fraktion (ESRS E5; direkte CSRD/ESRS; indirekte EU-taksonomi) & kg/ton & Affaldsopgørelser og leverandørdata. \\
    Sociale forhold & Arbejdsmiljøindikatorer (ESRS S1; direkte CSRD/ESRS; indirekte bankkrav) & antal/\% & HR-data og interne registreringer. \\
    \bottomrule
  \end{tabularx}
  \TableSource{Egen fremstilling baseret på case-materiale (bilag \ref{app:teknisk-dokumentation}).}
  \TableNote{Indikatorerne er illustrative og afspejler MVP'ens fokus.}
\end{table}

\begin{table}[h]
  \caption{Datakvalitetsfejl, der viser hvorfor tekniske kontrolpunkter er nødvendige.}
  \label{tab:datakvalitet-kontrol}
  \rowcolors{2}{black!5}{white}
  \begin{tabularx}{\textwidth}{l X X}
    \toprule
    Fejltype & Konsekvens & MVP-kontrolpunkt \\
    \midrule
    Manglende scope-data & Under- eller fejlrapportering af udledninger. & Obligatoriske felter og datadækningstjek pr. modul. \\
    Inkonsekvente enheder & Misvisende beregninger og sammenligninger. & Enhedsvalidering og automatisk normalisering. \\
    Periodeafgrænsning & Uens rapporteringsår og manglende sammenlignelighed. & Låste rapporteringsperioder og versionshistorik. \\
    Dubletter i input & Overestimerede mængder og KPI'er. & Dubletkontrol ved import og ændringslog. \\
    \bottomrule
  \end{tabularx}
  \TableSource{Egen fremstilling.}
  \TableNote{Eksemplerne er illustrative og bygger på almindelige datakvalitetsproblemer i ESG-rapportering.}
\end{table}

\begin{table}[h]
  \caption{Testscenarier, der viser hvordan kravene afprøves mod konkrete outputs.}
  \label{tab:testscenarier}
  \rowcolors{2}{black!5}{white}
  \begin{tabularx}{\textwidth}{l X X}
    \toprule
    Scenarie & Forventet output & Relaterede krav \\
    \midrule
    Energi (B2) & Valideret input, beregnet resultat og auditspor. & Dataindsamling, validering, beregning, auditspor. \\
    Affald & Registrering af mængder og konsistent rapportering. & Dataindsamling, validering, rapporteksport. \\
    Sociale indikatorer & Samlet KPI-oversigt til rapport. & Dataindsamling, rapporteksport. \\
    \bottomrule
  \end{tabularx}
  \TableSource{Egen fremstilling.}
  \TableNote{Scenarierne er illustrative og afspejler MVP'ens scope.}
\end{table}

\clearpage

\paragraph{Profil og modulvalg} Profiloverblik og profil-flow vises i figurer \ref{fig:app-landing-profil-overblik}, \ref{fig:app-profil-flow-start}, \ref{fig:app-profil-flow-ja}, \ref{fig:app-profil-flow-nej} og \ref{fig:app-profil-flow-progression}. Moduloverblik og navigation er dokumenteret i figurer \ref{fig:app-modul-overblik} og \ref{fig:app-modul-navigation}.

\begin{figure}[p]
  \centering
  \includegraphics[width=0.9\textwidth]{\detokenize{billeder af software/Landing – profil‑overblik.png}}
  \caption{Profiloverblik som startpunkt for brugerrejsen.}
  \label{fig:app-landing-profil-overblik}
  \FigureSource{Egen fremstilling (skærmbillede fra MVP).}
\end{figure}

\begin{figure}[p]
  \centering
  \includegraphics[width=0.9\textwidth]{\detokenize{billeder af software/Profil‑flow – startskærm.png}}
  \caption{Profil-flowets startskærm med formål og progression.}
  \label{fig:app-profil-flow-start}
  \FigureSource{Egen fremstilling (skærmbillede fra MVP).}
\end{figure}

\begin{figure}[p]
  \centering
  \includegraphics[width=0.9\textwidth]{\detokenize{billeder af software/Profil‑flow – svar “Ja” .png}}
  \caption{Profil-flow med svar ``Ja'' og opdateret progression.}
  \label{fig:app-profil-flow-ja}
  \FigureSource{Egen fremstilling (skærmbillede fra MVP).}
\end{figure}

\begin{figure}[p]
  \centering
  \includegraphics[width=0.9\textwidth]{\detokenize{billeder af software/Profil‑flow – svar “Nej” .png}}
  \caption{Profil-flow med svar ``Nej'' og opdateret progression.}
  \label{fig:app-profil-flow-nej}
  \FigureSource{Egen fremstilling (skærmbillede fra MVP).}
\end{figure}

\begin{figure}[p]
  \centering
  \includegraphics[width=0.9\textwidth]{\detokenize{billeder af software/Profil‑flow – progression.png}}
  \caption{Profil-flow efter flere svar med progression.}
  \label{fig:app-profil-flow-progression}
  \FigureSource{Egen fremstilling (skærmbillede fra MVP).}
\end{figure}

\begin{figure}[p]
  \centering
  \includegraphics[width=0.9\textwidth]{\detokenize{billeder af software/Wizard – overblik før moduler.png}}
  \caption{Moduloverblik med adgang til moduler.}
  \label{fig:app-modul-overblik}
  \FigureSource{Egen fremstilling (skærmbillede fra MVP).}
\end{figure}

\begin{figure}[p]
  \centering
  \includegraphics[width=0.9\textwidth]{\detokenize{billeder af software/Modul‑navigation – overblik .png}}
  \caption{Modulnavigation opdelt efter scope.}
  \label{fig:app-modul-navigation}
  \FigureSource{Egen fremstilling (skærmbillede fra MVP).}
\end{figure}

\clearpage
\paragraph{Dataindtastning og beregning} B2-modulet illustrerer dataindsamling, beregning og sporbarhed i figurer \ref{fig:app-b2-tomt}, \ref{fig:app-b2-udfyldt}, \ref{fig:app-b2-resultat} og \ref{fig:app-b2-trace}.

\begin{figure}[p]
  \centering
  \includegraphics[width=0.9\textwidth]{\detokenize{billeder af software/B2 – tomt:ikke udfyldt.png}}
  \caption{B2-modul uden udfyldte felter.}
  \label{fig:app-b2-tomt}
  \FigureSource{Egen fremstilling (skærmbillede fra MVP).}
\end{figure}

\begin{figure}[p]
  \centering
  \includegraphics[width=0.9\textwidth]{\detokenize{billeder af software/B2 – udfyldt formular.png}}
  \caption{B2-modul med udfyldt energi-input.}
  \label{fig:app-b2-udfyldt}
  \FigureSource{Egen fremstilling (skærmbillede fra MVP).}
\end{figure}

\begin{figure}[p]
  \centering
  \includegraphics[width=0.9\textwidth]{\detokenize{billeder af software/B2 – CO₂‑estimat i UI.png}}
  \caption{B2-modul med beregnet CO$_2$-estimat.}
  \label{fig:app-b2-resultat}
  \FigureSource{Egen fremstilling (skærmbillede fra MVP).}
\end{figure}

\begin{figure}[p]
  \centering
  \includegraphics[width=0.9\textwidth]{\detokenize{billeder af software/B2 – teknisk beregningstrace.png}}
  \caption{Beregningstrace for CO$_2$-estimat.}
  \label{fig:app-b2-trace}
  \FigureSource{Egen fremstilling (skærmbillede fra MVP).}
\end{figure}

\clearpage
\paragraph{Review og eksport} Review og eksport vises i figurer \ref{fig:app-review-topoverblik}, \ref{fig:app-reportoutput-preview} og \ref{fig:app-review-download}.

\begin{figure}[p]
  \centering
  \includegraphics[width=0.9\textwidth]{\detokenize{billeder af software/Review – topoverblik.png}}
  \caption{Review-side med status og eksportmuligheder.}
  \label{fig:app-review-topoverblik}
  \FigureSource{Egen fremstilling (skærmbillede fra MVP).}
\end{figure}

\clearpage

\begin{figure}[p]
  \centering
  \includegraphics[width=0.9\textwidth]{\detokenize{billeder af software/Review – PDF‑preview:sektion.png}}
  \caption{PDF-preview som del af rapportoutput.}
  \label{fig:app-reportoutput-preview}
  \FigureSource{Egen fremstilling (skærmbillede fra MVP).}
\end{figure}

\begin{figure}[p]
  \centering
  \includegraphics[width=0.9\textwidth]{\detokenize{billeder af software/Review – download‑knapper .png}}
  \caption{Download-knapper til rapportoutput.}
  \label{fig:app-review-download}
  \FigureSource{Egen fremstilling (skærmbillede fra MVP).}
\end{figure}

\clearpage

\subsection{Teknisk dokumentation}
\label{app:teknisk-dokumentation}

Dette bilag sammenfatter det interne case- og softwaremateriale, der ligger til grund for beskrivelsen af MVP'en i kapitel \ref{sec:software-og-mvp}. Materialet er internt, så nedenfor gives en kort offentlig oversigt af hensyn til fortrolighed og sporbarhed.

\paragraph{Eksterne bilag} Programkoden til MVP'en vedlægges som separat fil. Den detaljerede software-specifikation er udeladt af hensyn til fortrolighed og opsummeres derfor kun i offentlig form nedenfor.

\paragraph{Systemstruktur} MVP'en er en webbaseret løsning med adskilt brugergrænseflade, applikationslogik og datalagring. Strukturen understøtter et kontrolleret flow fra input til rapportoutput og gør det muligt at udvide komponenter uden at bryde den samlede proceslogik.

\paragraph{Datamodel og sporbarhed} Data organiseres omkring profilafgrænsning, modulregistreringer og et auditspor. Modellen sikrer, at ændringer, perioder og beregningsgrundlag kan efterprøves, og at data kan følges tilbage til deres kilde.

\paragraph{Datakvalitet og kontrol} Input valideres mod faste regler, enheder og obligatoriske felter. Versionshistorik gør det muligt at dokumentere udvikling over tid og at skelne mellem primære input og beregnede resultater.

\paragraph{Rapportoutput} Output omfatter et læsbart dokument og strukturerede data, så resultater kan genbruges i compliance- og beslutningsprocesser på tværs af interessenter.

\paragraph{Antagelser og begrænsninger} Bilaget beskriver MVP'ens aktuelle scope og forudsætter tilgængelige datakilder i konsistente formater. Det dækker ikke fuld drifts- eller sikkerhedsarkitektur, som typisk specificeres i en senere produktionsfase.

\subsubsection{Software-specifikation (offentlig oversigt)}
\label{app:software-spec}

Oversigten opsummerer de centrale tekniske og funktionelle principper uden interne detaljer:
\begin{itemize}
  \item Webbaseret løsning med adskilt brugergrænseflade, applikationslogik og relationel datalagring.
  \item Modulbaseret dataindsamling med valideringsregler og auditspor for ændringer.
  \item Beregningslogik for centrale ESG-indikatorer og eksport til PDF samt strukturerede dataformater.
  \item Sporbarhed via versionshistorik og dokumenterede antagelser.
  \item Udelader fuld drifts- og sikkerhedsarkitektur, som specificeres i en senere produktionsfase.
\end{itemize}

Den detaljerede software-specifikation og programkode vedlægges som separate filer.
