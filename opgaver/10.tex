\section{Perspektivering og anbefalinger}
\label{sec:perspektivering-og-anbefalinger}

Perspektiveringen peger på, at ESG-rapportering i sundhedssektoren vil udvikle sig fra ad hoc-projekter til en mere kontinuerlig driftsdisciplin. Det sætter ramme for anbefalingerne og for behovet for fler-case studier på tværs af delsektorer, longitudinelle analyser af implementering og test af, hvordan standardiserede data påvirker beslutningskvalitet og risikostyring.
Afsnittet organiserer anbefalingerne i tre niveauer: empirisk baserede, produktmæssige og normative.

\textbf{Empirisk baserede anbefalinger:}
\begin{itemize}
  \item VSME-standarden sænker startniveauet, så SMV'er kan begynde med basismodul og gradvist udvide rapporteringen i takt med bedre datakvalitet og ressourcer.\parencite{Virksomhedsguiden2025VSMEIntro,Virksomhedsguiden2025VSMEModules,Virksomhedsguiden2025NoDualMateriality}
  \item \textcite{Khan2016Materiality} viser størst effekt for materielle temaer, hvilket peger på prioritering af energi, affald og arbejdsmiljø.
  \item ESG-as-a-Service-løsninger bør anvendes til at sikre \textit{auditspor} og ensartet dataindsamling, så rapporteringen bliver sporbar og beslutningsrelevant (bilag \ref{app:teknisk-dokumentation}).
  \item Stop-the-clock bør ikke tolkes som en pause, men som en tidsbuffer til at etablere datagrundlag og \textit{governance}-strukturer, i tråd med \textcite{Erhvervsstyrelsen2025StopClock}.
\end{itemize}

\textbf{Gradvis overgang kan operationaliseres som et beslutningskriterium.} Tabel \ref{tab:gearshift-kriterier} angiver, hvornår en SMV bør skifte fra basismodul til udvidet scope.

\begin{table}[h]
  \caption{Skifte-kriterier for overgang fra basismodul til udvidet scope.}
  \label{tab:gearshift-kriterier}
  \begin{tabularx}{\textwidth}{l X}
    \toprule
    Kriterietype & Indikator og trigger \\
    \midrule
    Eksternt pres & Bankkrav, udbudskrav eller kundekrav om udvidede datapunkter. \\
    Datamodenhed & Min. 50\% af kernedata via automatiske kilder og valideringsfejlrate under 5\%. \\
    Sporbarhed & $C_{\text{trace}} \geq 90\%$ for kernemoduler i mindst to rapporteringscykler. \\
    Governance & Navngiven dataejer og sign-off-proces med fast responstid. \\
    \bottomrule
  \end{tabularx}
  \TableSource{Egen fremstilling.}
\end{table}

\textbf{Produktmæssige næste skridt:}
\begin{itemize}
  \item Udvid integrationer til centrale datakilder, så manuel indtastning reduceres.
  \item Standardisér mapping til ESRS/GRI og styrk rapportskabeloner, så output kan genbruges i flere sammenhænge.
  \item Forbered assurance-ready kontroller ved at uddybe auditspor og dokumentationsniveau.
  \item Tilføj analytiske funktioner (fx baseline-overblik og trendvisualisering), der understøtter løbende forbedringer.
\end{itemize}

\textbf{Normative vurderinger:}
\begin{itemize}
  \item Regulatorer bør prioritere proportionalitet og stabilitet i standarder, så SMV'er ikke presses til symbolsk rapportering uden kapacitet til reel implementering.
  \item Brancheaktører bør udvikle delte datapraksisser og fælles minimumsstandarder, der reducerer transaktionsomkostninger i forsyningskæderne.
  \item Serviceudbydere bør sikre transparens om antagelser og usikkerheder for at modvirke greenwashing og øge tillid til rapporteringen.
\end{itemize}
