\subsection{Standardisering og organisationer (Brunsson)}
\label{subsec:standardisering-og-organisationer-brunsson}

Ifølge \textcite{BrunssonWorldOfStandards} er standarder organiserede regler og forventninger, der skaber ensartethed og sammenlignelighed på tværs af organisationer. Standardisering er ikke kun teknisk, men også institutionel: den etablerer, hvad der betragtes som legitim praksis, og skaber et fælles sprog for kontrol, måling og rapportering.
\begin{displayquote}
Standardization is a fundamental form for governance and co-ordination in societies.
\quoteattrib{\textcite{BrunssonWorldOfStandards}}
\end{displayquote}

Standarder kan samtidig fungere som styring på afstand. De gør det muligt at koordinere aktører, der ikke deler samme kontekst, men de skaber også risiko for, at organisationer fokuserer på formel overholdelse frem for substantiel forandring. Som \textcite{BrunssonWorldOfStandards} formulerer det: \enquote{The mere existence of a standard does not guarantee that it will be followed.} Det åbner for dekobling mellem det, der rapporteres, og det, der faktisk gøres i praksis.

I ESG-rapportering betyder det, at standarder definerer, hvilke indikatorer og narrativer der opfattes som gyldig dokumentation. Det fremmer auditabilitet og sammenlignelighed, men kan også reducere rapportering til et compliance-projekt. For SMV'er er standardisering derfor både en forudsætning for legitim rapportering og en byrde, der kræver ressourcer, struktur og data. ESG-as-a-Service kan ses som en standardiseringsinfrastruktur, der omsætter krav til konkrete datafelter og arbejdsgange, men den kan også forstærke fokus på minimumskrav frem for strategisk læring. Analytisk peger dette på at vurdere graden af faktisk praksisændring versus symbolsk efterlevelse.
