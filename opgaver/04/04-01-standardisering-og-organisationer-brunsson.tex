\subsection{Standardisering og organisationer (Brunsson)}
\label{subsec:standardisering-og-organisationer-brunsson}

\textcite{BrunssonWorldOfStandards} beskriver standarder som organiserede regler og forventninger, der skaber ensartethed og sammenlignelighed p\aa tv\ae rs af organisationer. Standardisering er ikke kun teknisk, men ogs\aa institutionel: den etablerer, hvad der betragtes som legitim praksis, og skaber et f\ae lles sprog for kontrol, m\aa ling og rapportering.

Standarder kan samtidig fungere som styring p\aa afstand. De g\o r det muligt at koordinere akt\o rer, der ikke deler samme kontekst, men de skaber ogs\aa risiko for, at organisationer fokuserer p\aa formel overholdelse frem for substantiel forandring. Brunsson peger p\aa, at standardisering kan give anledning til dekobling mellem det, der rapporteres, og det, der faktisk g\o res i praksis.\parencite{BrunssonWorldOfStandards}

I \ESG-rapportering betyder det, at standarder definerer, hvilke indikatorer og narrativer der opfattes som gyldig dokumentation. Det fremmer auditabilitet og sammenlignelighed, men kan ogs\aa reducere rapportering til et compliance-projekt. For \SMV'er er standardisering derfor b\aa de en foruds\ae tning for legitim rapportering og en byrde, der kr\ae ver ressourcer, struktur og data. \ESG-as-a-Service kan ses som en standardiseringsinfrastruktur, der oms\ae tter krav til konkrete datafelter og workflows, men den kan ogs\aa forst\ae rke fokus p\aa minimumskrav frem for strategisk l\ae ring. Analytisk peger dette p\aa at vurdere graden af faktisk praksis\ae ndring versus symbolsk efterlevelse.
