\subsection{\O konomiske og kommercielle perspektiver}
\label{subsec:oekonomiske-og-kommercielle-perspektiver}

Det \o konomiske perspektiv starter med sp\o rgsm\aa let om virksomhedens form\aa l. \textcite{Friedman1970Profits} argumenterer for, at virksomhedens prim\ae re ansvar er at maksimere profit inden for lovens rammer, mens stakeholder-tilgangen betoner ansvar over for flere interessenter end aktion\ae rerne.\parencite{Freeman1984Stakeholder} \textcite{Carroll1991Pyramid} uddyber dette gennem en CSR-pyramide, hvor \o konomiske, juridiske, etiske og filantropiske hensyn kombineres i virksomhedens ansvar.

I et bredere v\ae rdiskabelsesperspektiv fremh\ae ver triple bottom line, at b\ae redygtighed skal vurderes p\aa tv\ae rs af \o konomi, milj\o og sociale forhold.\parencite{Elkington1998TripleBottomLine} \textcite{PorterKramer2011CSV} argumenterer for, at virksomheder kan skabe shared value ved at koble samfundsm\ae ssige udfordringer til forretningsstrategi og innovation. Disse perspektiver tilbyder et begrebsligt fundament for at vurdere, om \ESG-indsatser kan skabe varig v\ae rdi frem for at v\ae re rene omkostninger.

Empirisk peger studier p\aa en overvejende positiv eller neutral sammenh\ae ng mellem \ESG og finansiel performance, men effekten varierer med kontekst og tidsperiode.\parencite{Eccles2014Impact,Friede2015Meta} \textcite{Khan2016Materiality} viser, at finansielt materielle \ESG-temaer er forbundet med bedre performance, mens immaterielle temaer ikke har samme effekt. Det peger p\aa, at v\ae rdien af \ESG-arbejde afh\ae nger af strategisk fokus og relevans.

For denne rapport betyder det, at \ESG-as-a-Service ikke kun skal vurderes p\aa compliance, men ogs\aa p\aa om l\o sningen reducerer risici, forbedrer beslutningsgrundlag og skaber \o konomisk nytte for \SMV'er. Det informerer vurderingen af betalingsvillighed, prisstruktur og den kommercielle b\ae redygtighed af servicekonceptet.
