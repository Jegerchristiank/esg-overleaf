\subsection{Økonomiske og kommercielle perspektiver}
\label{subsec:oekonomiske-og-kommercielle-perspektiver}

Det økonomiske perspektiv starter med spørgsmålet om virksomhedens formål. \textcite{Friedman1970Profits} argumenterer for, at virksomhedens primære ansvar er at maksimere profit inden for lovens rammer, mens stakeholder-tilgangen betoner ansvar over for flere interessenter end aktionærerne.\parencite{Freeman1984Stakeholder} \textcite{Carroll1991Pyramid} uddyber dette gennem en CSR-pyramide, hvor økonomiske, juridiske, etiske og filantropiske hensyn kombineres i virksomhedens ansvar.

I et bredere værdiskabelsesperspektiv fremhæver triple bottom line, at bæredygtighed skal vurderes på tværs af økonomi, miljø og sociale forhold.\parencite{Elkington1998TripleBottomLine} \textcite{PorterKramer2011CSV} argumenterer for, at virksomheder kan skabe shared value ved at koble samfundsmæssige udfordringer til forretningsstrategi og innovation. Disse perspektiver tilbyder et begrebsligt fundament for at vurdere, om ESG-indsatser kan skabe varig værdi frem for at være rene omkostninger.

Empirisk peger studier på en overvejende positiv eller neutral sammenhæng mellem ESG og finansiel performance, men effekten varierer med kontekst og tidsperiode.\parencite{Eccles2014Impact,Friede2015Meta} \textcite{Khan2016Materiality} viser, at finansielt materielle ESG-temaer er forbundet med bedre performance, mens immaterielle temaer ikke har samme effekt. Det peger på, at værdien af ESG-arbejde afhænger af strategisk fokus og relevans.

\begin{displayquote}
ESG is a means, not an end.
\quoteattrib{\textcite{ESGBook}}
\end{displayquote}

Det betyder, at ESG-as-a-Service ikke kun skal vurderes på compliance, men også på om løsningen reducerer risici, forbedrer beslutningsgrundlag og skaber økonomisk nytte for SMV'er. Det informerer vurderingen af betalingsvillighed, prisstruktur og den kommercielle bæredygtighed af servicekonceptet.

Perspektiverne samles i den efterfølgende opsamling, som definerer de centrale kriterier for analysen (\ref{subsec:opsamling-af-den-teoretiske-ramme}).
