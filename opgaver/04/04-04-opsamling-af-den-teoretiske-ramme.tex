\subsection{Opsamling af den teoretiske ramme}
\label{subsec:opsamling-af-den-teoretiske-ramme}

Den teoretiske ramme samler tre komplement\ae re perspektiver, der tilsammen forklarer, hvorfor \ESG-rapportering opfattes som n\o dvendig, hvordan den implementeres, og hvilke \o konomiske konsekvenser den kan have. Standardisering bidrager med begreber om legitimitet og sammenlignelighed, governance forklarer oversaettelse af policy til praksis, og \o konomiske perspektiver adresserer v\ae rdiskabelse og betalingsvillighed.

Tabel \ref{tab:teoretisk-ramme-overblik} opsummerer de centrale begreber og viser, hvordan de anvendes i analysen.
\begin{table}[t]
  \caption{Teoretiske linser og analytiske anvendelser i rapporten.}
  \label{tab:teoretisk-ramme-overblik}
  \begin{tabularx}{\textwidth}{l X X}
    \toprule
    Perspektiv & Kernebegreber & Analytisk anvendelse \\
    \midrule
    Standardisering & Legitimitet, sammenlignelighed, dekobling. & Vurderer hvordan \ESG-krav oms\ae ttes til standardiserede data og om rapportering bliver substans eller symbol. \\
    Policy og governance & Translasjon, intermediaere akt\o rer, flerniveau-styring. & Forklarer hvordan regulering og standarder overs\ae ttes til lokale processer og tekniske workflows. \\
    \O konomi og kommercialisering & Stakeholder/shareholder, shared value, finansiel materialitet. & Vurderer \o konomisk relevans, forretningsmodel og betalingsvillighed for \ESG-as-a-Service. \\
    \bottomrule
  \end{tabularx}
  \TableNote{Egen syntese.}
\end{table}

Opsamlingen fungerer som analytisk ramme for kapitel \ref{sec:analyse}, hvor teorierne anvendes til at vurdere \ESG-rapporteringens praksis og \MVP'ens kommercielle og organisatoriske implikationer.
