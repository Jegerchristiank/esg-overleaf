\subsection{Opsamling af den teoretiske ramme}
\label{subsec:opsamling-af-den-teoretiske-ramme}

Den teoretiske ramme samler tre komplementære perspektiver, der tilsammen forklarer, hvorfor ESG-rapportering opfattes som nødvendig, hvordan den implementeres, og hvilke økonomiske konsekvenser den kan have. Standardisering bidrager med begreber om legitimitet og sammenlignelighed, governance forklarer oversættelse af policy til praksis, og økonomiske perspektiver adresserer værdiskabelse og betalingsvillighed.

Tabel \ref{tab:teoretisk-ramme-overblik} opsummerer de centrale begreber og viser, hvordan de anvendes i analysen.
\begin{table}[h]
  \caption{Teoretiske linser, der definerer analysens vurderingskriterier.}
  \label{tab:teoretisk-ramme-overblik}
  \begin{tabularx}{\textwidth}{l X X}
    \toprule
    Perspektiv & Kernebegreber & Analytisk anvendelse \\
    \midrule
    Standardisering & Legitimitet, sammenlignelighed, dekobling. & Vurderer hvordan ESG-krav omsættes til standardiserede data og om rapportering bliver substans eller symbol. \\
    Policy og governance & Translasion, intermediære aktører, flerniveau-styring. & Forklarer hvordan regulering og standarder oversættes til lokale processer og tekniske arbejdsgange. \\
    Økonomi og kommercialisering & Stakeholder/shareholder, shared value, finansiel materialitet. & Vurderer økonomisk relevans, forretningsmodel og betalingsvillighed for ESG-as-a-Service. \\
    \bottomrule
  \end{tabularx}
  \TableSource{Egen fremstilling.}
\end{table}

Analysen tager især afsæt i to spændinger: (1) standardiseringens løfte om legitimitet versus risikoen for dekobling, og (2) compliance-logik versus beslutningsrelevant ESG som styringsgrundlag. Det skaber en eksplicit ramme for at vurdere, om ESG-as-a-Service omsætter krav til praksis eller primært producerer dokumentation.

Opsamlingen fungerer dermed som analytisk ramme for kapitel \ref{sec:analyse}, hvor teorierne anvendes til at vurdere ESG-rapporteringens praksis og MVP'ens kommercielle og organisatoriske implikationer.
