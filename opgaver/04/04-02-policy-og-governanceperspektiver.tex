\subsection{Policy- og governanceperspektiver}
\label{subsec:policy-og-governanceperspektiver}

Policy- og governanceperspektiver fokuserer p\aa, hvordan regulering oms\ae ttes til praksis gennem institutionelle mekanismer, standarder og mellemled. \ESG-rapportering i EU er et eksempel p\aa flerniveau-styring, hvor politiske m\aa ls\ae tninger realiseres gennem direktiver, standarder og vejledninger, som organisationer skal fortolke og implementere lokalt.

\textcite{RoevikTrenderTranslasjoner} beskriver, hvordan ideer og styringskoncepter vandrer mellem organisationer og bliver oversat til lokale praksisser. I hans translationsteori betones, at implementering ikke er en ren kopiering, men en proces, hvor krav redigeres, omformuleres og tilpasses til organisatoriske betingelser.\parencite{TranslationTheoryKnowledgeTransfer} Det betyder, at ensartede standarder kan skabe variation i praksis, afh\ae ngigt af akt\o rers kapacitet, fortolkning og incitamenter.

I \ESG-rapportering fungerer revisorer, konsulenter og softwareleverand\o rer som intermediaere akt\o rer, der overs\ae tter policy til datafelter, processer og beslutningsregler. Disse akt\o rer bidrager til governance ved at definere, hvad der opfattes som tilstr\ae kkelig dokumentation, og ved at skabe tekniske rammer for sporbarhed.

Analytisk betyder perspektivet, at rapporten unders\o ger, hvordan regulatoriske krav transformeres til operationelle krav i sundhedssektoren, og hvordan \ESG-as-a-Service fungerer som en oversaettelsesmekanisme mellem policy og praksis. Det giver et grundlag for at vurdere, hvorvidt governance styrker faktisk implementering eller prim\ae rt producerer formel overholdelse.
