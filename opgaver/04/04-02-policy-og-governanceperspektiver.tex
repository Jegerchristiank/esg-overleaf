\subsection{Policy- og governanceperspektiver}
\label{subsec:policy-og-governanceperspektiver}

Policy- og governanceperspektiver fokuserer på, hvordan regulering omsættes til praksis gennem institutionelle mekanismer, standarder og mellemled. ESG-rapportering i EU er et eksempel på flerniveau-styring, hvor politiske målsætninger realiseres gennem direktiver, standarder og vejledninger, som organisationer skal fortolke og implementere lokalt.

\textcite{RoevikTrenderTranslasjoner} beskriver, hvordan ideer og styringskoncepter vandrer mellem organisationer og bliver oversat til lokale praksisser. I hans translasionsteori betones, at implementering ikke er en ren kopiering, men en proces, hvor krav redigeres, omformuleres og tilpasses til organisatoriske betingelser.\parencite{TranslationTheoryKnowledgeTransfer} Det betyder, at ensartede standarder kan skabe variation i praksis, afhængigt af aktørers kapacitet, fortolkning og incitamenter.
\begin{displayquote}
Uthenting, overføring og mottak av organisasjonsideer kan forstås som en form for oversettelse.
\par Dansk oversættelse: Udtagning, overførsel og modtagelse af organisationsideer kan forstås som en form for oversættelse af idéer.
\quoteattrib{\textcite{RoevikTrenderTranslasjoner}}
\end{displayquote}

I ESG-rapportering fungerer revisorer, konsulenter og softwareleverandører som intermediære aktører, der oversætter policy til datafelter, processer og beslutningsregler. Disse aktører bidrager til governance ved at definere, hvad der opfattes som tilstrækkelig dokumentation, og ved at skabe rammer for sporbarhed. Ifølge \textcite{TranslationTheoryKnowledgeTransfer} er \enquote{Modtagere er ikke passive modtagere, men fortolkende systemer}, hvilket understreger, at governance altid filtreres gennem lokale fortolkninger.

Analytisk betyder perspektivet, at analysen undersøger, hvordan regulatoriske krav transformeres til operationelle krav i sundhedssektoren, og hvordan ESG-as-a-Service fungerer som en oversættelsesmekanisme mellem policy og praksis. Det giver et grundlag for at vurdere, hvorvidt governance styrker faktisk implementering eller primært producerer formel overholdelse.

Perspektivet omsættes til følgende analytiske spørgsmål i analysen:
\begin{itemize}
  \item Hvilke mellemled oversætter regulatoriske krav til konkrete datafelter og processer?
  \item Hvor opstår de centrale fortolkninger og redigeringer af standarder i praksis?
  \item Hvilke governance-mekanismer skaber sporbarhed og reducerer risikoen for formel efterlevelse uden praksisændring?
\end{itemize}
