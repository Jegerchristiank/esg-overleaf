\subsection{ESG-as-a-Service som servicekoncept}
\label{subsec:esg-as-a-service-som-servicekoncept}

\ESG-as-a-Service betegner en kombineret software- og serviceleverance, der oms\ae tter regulatoriske krav til operationelle datapunkter, kontroller og rapportoutput. Konceptet adskiller sig fra klassisk SaaS ved at inkludere faglig sparring, konfigurerede standarder og l\o bende datakvalitetssikring, men adskiller sig ogs\aa fra traditionel konsulentbistand ved at bygge p\aa en fast digital infrastruktur.

V\ae rdiforslaget kan forankres i stakeholder- og shared value-perspektiver, hvor dokumenteret \ESG-indsats er en foruds\ae tning for legitimitet og langsigtet v\ae rdiskabelse.\parencite{Freeman1984Stakeholder,PorterKramer2011CSV,Elkington1998TripleBottomLine} Samtidig eksisterer en modposition, hvor virksomhedens prim\ae re ansvar er over for aktion\ae rerne, hvilket understreger behovet for at g\o re compliance og \o konomisk relevans eksplicit i servicekonceptet.\parencite{Friedman1970Profits}

I en reguleret sundhedssektor fungerer \ESG-as-a-Service som et organisatorisk mellemled, der kan standardisere dataindsamling, reducere transaktionsomkostninger og skabe en audit trail, som g\o r rapportering beslutningsrelevant for ledelse, revisor og myndigheder.
