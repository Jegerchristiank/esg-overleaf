\subsection{ESG-as-a-Service som servicekoncept}
\label{subsec:esg-as-a-service-som-servicekoncept}

ESG-as-a-Service betegner en kombineret software- og serviceleverance, der omsætter regulatoriske krav til operationelle datapunkter, kontroller og rapportoutput. Konceptet adskiller sig fra klassisk SaaS (Software as a Service) ved at inkludere faglig sparring, konfigurerede standarder og løbende datakvalitetssikring, men adskiller sig også fra traditionel konsulentbistand ved at bygge på en fast digital infrastruktur.

Værdiforslaget kan forankres i stakeholder- og shared value-perspektiver, hvor dokumenteret ESG-indsats er en forudsætning for legitimitet og langsigtet værdiskabelse.\parencite{Freeman1984Stakeholder,PorterKramer2011CSV,Elkington1998TripleBottomLine} Samtidig eksisterer en modposition, hvor virksomhedens primære ansvar er over for aktionærerne, hvilket understreger behovet for at gøre compliance og økonomisk relevans eksplicit i servicekonceptet.\parencite{Friedman1970Profits}

I en reguleret sundhedssektor fungerer ESG-as-a-Service som et organisatorisk mellemled, der kan standardisere dataindsamling, reducere transaktionsomkostninger og skabe et auditspor, som gør rapportering beslutningsrelevant for ledelse, revisor og myndigheder. IWA 48 fremhæver, at ESG-implementering kræver standardiserede KPI'er, datakvalitet og klare rapporteringsprincipper, hvilket styrker behovet for en struktureret infrastruktur.\parencite{ISO2024IWA48} Sektoren er et relevant startmarked, fordi compliance-krav, datakompleksitet og forsyningskædepres gør behovet for struktureret ESG-rapportering særligt tydeligt.

Konceptet fungerer dermed som det praktiske svar på den identificerede kløft mellem krav og datapraksis og motiverer den teoretiske ramme i \ref{sec:teoretisk-ramme}, hvor standardisering, governance og værdiskabelse analyseres.
