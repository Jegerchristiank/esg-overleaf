\subsection{SMV'er og behovet for forenkling}
\label{subsec:smver-og-behovet-for-forenkling}

SMV'er defineres efter EU's størrelsesgrænser og vil ofte ligge uden for den direkte CSRD-pligt, men påvirkes indirekte gennem kundekrav, banker og leverandørrelationer.\parencite{Virksomhedsguiden2025SMVDefinition,PwC2025GuideESGSMV,GrantThorntonDK2025Omnibus} Dette skaber et rapporteringspres, selv hvor rapporteringen formelt er frivillig.

EU har derfor udviklet en frivillig VSME-standard (Voluntary Sustainability Reporting Standard for SMEs), der skal sikre ensartet dataudveksling og forhindre uforholdsmæssige datakrav fra større virksomheder.\parencite{Virksomhedsguiden2025VSMEIntro} VSME er opbygget af et basismodul og et udvidet modul, hvor basismodullets 11 datapunkter kan anvendes som minimumsniveau.\parencite{Virksomhedsguiden2025VSMEModules} Standarden kræver ikke en dobbelt væsentlighedsanalyse, hvilket reducerer kompleksitet og ressourceforbrug.\parencite{Virksomhedsguiden2025NoDualMateriality} Erhvervsstyrelsen har samtidig udviklet en skabelon, der samler datapunkterne og understøtter ensartet rapportering.\parencite{Erhvervsstyrelsen2025ESGTemplate}

Indfasningen af CSRD er trinvist implementeret, og den seneste stop-the-clock-aftale udskyder rapporteringskrav for flere virksomhedstyper, hvilket giver SMV'er mere tid, men også skaber usikkerhed om krav og timing.\parencite{Erhvervsstyrelsen2025StopClock} Erfaringer fra de første VSME-rapporter viser, at mange virksomheder rapporterer ud over basismodullets krav, hvilket indikerer både ambition og behov for klar prioritering.\parencite{EY2025VSME}

Samlet peger udviklingen på et behov for forenklede processer, klare minimumskrav og teknisk støtte, så SMV'er kan levere sporbar ESG-dokumentation uden at belaste kerneopgaven.
