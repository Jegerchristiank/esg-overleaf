\subsection{SMV'er og behovet for forenkling}
\label{subsec:smver-og-behovet-for-forenkling}

\SMV'er defineres efter EU's st\o rrelsesgr\ae nser og vil ofte ligge uden for den direkte \CSRD-pligt, men p\aa virkes indirekte gennem kundekrav, banker og leverand\o rrelationer.\parencite{Virksomhedsguiden2025SMVDefinition,PwC2025GuideESGSMV,GrantThorntonDK2025Omnibus} Dette skaber et rapporteringspres, selv hvor rapporteringen formelt er frivillig.

EU har derfor udviklet en frivillig VSME-standard, der skal sikre ensartet dataudveksling og forhindre uforholdsm\ae ssige datakrav fra st\o rre virksomheder.\parencite{Virksomhedsguiden2025VSMEIntro} VSME er opbygget af et basismodul og et udvidet modul, hvor basismodullets 11 datapunkter kan anvendes som minimumsniveau.\parencite{Virksomhedsguiden2025VSMEModules} Standarden kr\ae ver ikke en dobbelt v\ae sentlighedsanalyse, hvilket reducerer kompleksitet og ressourceforbrug.\parencite{Virksomhedsguiden2025NoDualMateriality} Erhvervsstyrelsen har samtidig udviklet en skabelon, der samler datapunkterne og underst\o tter ensartet rapportering.\parencite{Erhvervsstyrelsen2025ESGTemplate}

Indfasningen af \CSRD er trinvist implementeret, og den seneste stop-the-clock-aftale udskyder rapporteringskrav for flere virksomhedstyper, hvilket giver \SMV'er mere tid, men ogs\aa skaber usikkerhed om krav og timing.\parencite{Erhvervsstyrelsen2025StopClock} Erfaringer fra de f\o rste VSME-rapporter viser, at mange virksomheder rapporterer ud over basismodullets krav, hvilket indikerer b\aa de ambition og behov for klar prioritering.\parencite{EY2025VSME}

Samlet peger udviklingen p\aa et behov for forenklede processer, klare minimumskrav og teknisk st\o tte, s\aa \SMV'er kan levere sporbar \ESG-dokumentation uden at belaste kerneopgaven.
