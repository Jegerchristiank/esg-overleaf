\subsection{Regulatoriske rammer: CSRD, EU-taksonomien og GRI}
\label{subsec:regulatoriske-rammer-csrd-eu-taksonomien-og-gri}

De regulatoriske rammer for ESG-rapportering i EU består af flere sammenhængende elementer. CSRD udvider rapporteringspligten og gør bæredygtighedsrapportering til en integreret del af virksomhedernes officielle rapportering. Direktivet fastsætter, at rapporteringen skal ske efter ESRS, som er vedtaget som delegerede standarder, og at oplysningerne leveres i et standardiseret, digitalt format med XBRL-tagging. Implementeringen er trinvist indfaset og politisk justeret gennem den såkaldte stop-the-clock-aftale.\parencite{EU2022CSRD,EU2023ESRS,Erhvervsstyrelsen2025StopClock}

Indfasningen og de politiske justeringer betyder, at virksomheder skal planlægge rapportering under bevægelige rammer, hvor tidsplaner og præciseringer ændres over tid. Det gør fortolkning og prioritering til en reel del af compliance-arbejdet, særligt for SMV'er med begrænset kapacitet.\parencite{Erhvervsstyrelsen2025StopClock}

ESRS konkretiserer de oplysninger, virksomhederne skal levere, og giver struktur til sammenhængen mellem strategi, risici, mål og resultater. Standarderne operationaliserer kravet om dobbelt væsentlighed og etablerer en fælles logik for datagrundlag og rapportering.\parencite{EU2023ESRS}

Dobbelt væsentlighed betyder, at virksomheder skal rapportere både deres påvirkning af mennesker og miljø og hvordan bæredygtighedsforhold påvirker virksomheden finansielt. Det udvider datakravet til værdikæde, governance og sociale forhold og gør materialitetsprocessen til en central del af rapporteringen.\parencite{EU2023ESRS}

EU-taksonomien supplerer rapporteringen ved at definere, hvilke økonomiske aktiviteter der kan betragtes som miljømæssigt bæredygtige. Rammeværket kræver, at virksomheder kan dokumentere overensstemmelse med taksonomien og derved knytte finansielle aktiviteter til konkrete miljømæssige mål.\parencite{EU2020Taxonomy}

GRI er et globalt, frivilligt rammeværk, som anvendes bredt til sammenlignelig ESG-rapportering. Det tilbyder en struktureret tilgang til indikatorer og narrativer og bruges ofte som supplement til regulatoriske krav.\parencite{GRI2021}

\subsubsection{EU-retligt overblik og retskildernes funktion}
\label{subsec:eu-retligt-overblik}

ESG-regimet fungerer som et EU-retligt flerniveau-system, hvor retskilderne har forskellige funktioner og bindende karakter. Det gør det muligt at forstå, hvorfor rapporteringen både kræver juridisk pligt, tekniske standarder og klassifikationslogik:
\begin{enumerate}
  \item \textbf{Direktiv (CSRD).} Etablerer rapporteringspligt og overordnede krav, som skal implementeres i national ret.\parencite{EU2022CSRD}
  \item \textbf{Delegerede standarder (ESRS).} Konkretiserer indhold, datapunkter og struktur, så rapportering bliver sammenlignelig og auditerbar.\parencite{EU2023ESRS}
  \item \textbf{Forordning (EU-taksonomien).} Fastlægger en klassifikation af bæredygtige aktiviteter og kobler rapportering til miljømål og finansielle nøgletal.\parencite{EU2020Taxonomy}
  \item \textbf{Frivillige standarder (GRI).} Udgør et udbredt supplement, især hvor globale interessenter efterspørger sammenlignelige ESG-oplysninger.\parencite{GRI2021}
\end{enumerate}

Det hierarkiske overblik er samlet i tabel \ref{tab:eu-retligt-hierarki}.
\begin{table}[t]
  \caption{EU-retligt hierarki for ESG-rapportering og dets funktionelle rolle i styringskæden.}
  \label{tab:eu-retligt-hierarki}
  \begin{tabularx}{\textwidth}{l X X}
    \toprule
    Niveau & Retskilde og bindende karakter & Funktion i ESG-regimet \\
    \midrule
    Direktiv & CSRD; bindende mål, implementeres i national ret. & Etablerer rapporteringspligt, indfasning og krav til digital rapportering. \\
    Delegerede standarder & ESRS; bindende teknisk konkretisering. & Definerer datapunkter, struktur og auditbarhed. \\
    Forordning & EU-taksonomien; direkte gældende. & Klassificerer aktiviteter og forbinder rapportering med miljømål og nøgletal. \\
    Frivillige standarder & GRI; ikke-bindende, globalt udbredt. & Supplerer sammenlignelighed, især i globale værdikæder. \\
    \bottomrule
  \end{tabularx}
  \TableSource{\parencite{EU2022CSRD,EU2023ESRS,EU2020Taxonomy,GRI2021}}
\end{table}

\begin{displayquote}
ESG does not currently benefit from a universally accepted common set of standards.
\quoteattrib{\textcite{ESGBook}}
\end{displayquote}

Det gør reguleringen analytisk interessant: på den ene side øger standarderne sammenligneligheden, på den anden side efterlader de et fortolkningsrum, hvor rapportering kan glide over i minimum compliance frem for beslutningsrelevant styring.

\subsubsection{Gennemgående regulatoriske principper}
\label{subsec:gennemgaaende-regulatoriske-principper}

På tværs af retskilderne går en række gentagende principper, der forklarer, hvorfor ESG-rapportering kræver systematik. Proportionalitet og faseindføring betyder, at kravene skaleres og udskydes, hvilket skaber et gradvist, men vedvarende implementeringspres.\parencite{EU2022CSRD,Erhvervsstyrelsen2025StopClock} Dobbelt væsentlighed gør materialitetsprocessen til et kernekrav og udvider datagrundlaget til både påvirkning og finansiel risiko.\parencite{EU2023ESRS} Dokumentations- og sporbarhedskrav følger af standardernes struktur og taksonomiens klassifikationslogik, hvilket nødvendiggør auditspor og konsistente datakilder.\parencite{EU2023ESRS,EU2020Taxonomy} Endelig peger udviklingen i finansielt tilsyn på en risikobaseret forventning om ESG-data, som forstærker kravene uden for den direkte rapporteringspligt.\parencite{Finanstilsynet2025ESGRisk}

\subsubsection{Implementering i dansk ret og praksis}
\label{subsec:implementering-i-dansk-ret}

Som direktiv forudsætter CSRD national implementering. I Danmark sker implementeringen via lov nr 480 af 22/05/2024, som ændrer årsregnskabsloven og relaterede love, og som operationaliseres gennem Erhvervsstyrelsens vejledninger og værktøjer til SMV'er.\parencite{EU2022CSRD,Lov2024CSRD,Aarsregnskabsloven2022,Erhvervsstyrelsen2025ESGTemplate,Virksomhedsguiden2025VSMEIntro} I praksis mødes virksomhederne ofte gennem nationale kanaler som myndigheder, revisorer og finansielle institutioner, hvor bankernes ESG-risikostyring gør data til en forudsætning for kredit, også når rapportering formelt er frivillig.\parencite{DanskErhvervFSR2025CSRDTimeline,GrantThorntonDK2025BankESG,Finanstilsynet2025ESGRisk}

Denne juridiske struktur forankrer problemformuleringens fokus på at reducere kløften mellem krav og datapraksis (jf. \ref{subsec:problemformulering}): ESG-as-a-Service er ikke blot et teknisk valg, men et regulatorisk mellemled i et flerniveau-regime.

Tabel \ref{tab:rammer-sammenligning} opsummerer de centrale forskelle mellem rammerne og deres implikationer for datakrav og rapporteringslogik.
\begin{table}[t]
  \caption{Sammenligning, der viser hvordan retskilderne udfylder forskellige styringsfunktioner i ESG-rapportering.}
  \label{tab:rammer-sammenligning}
  \begin{tabularx}{\textwidth}{l X X}
    \toprule
    Rammeværk & Status og formål & Implikation for rapportering \\
    \midrule
    CSRD (direktiv) & Obligatorisk ramme for rapporteringspligt og indfasning. & Forankrer rapportering i års- og koncernrapport med krav om digital tagging. \\
    ESRS (delegerede standarder) & Tekniske standarder for datapunkter og struktur. & Operationaliserer dobbelt væsentlighed og sikrer sammenlignelighed. \\
    EU-taksonomien & Obligatorisk supplement; klassificerer miljømæssigt bæredygtige aktiviteter. & Kræver dokumentation for aktiviteters bidrag og overensstemmelse med minimumsgarantier. \\
    GRI & Frivilligt rammeværk, anvendes globalt. & Indikatorbaseret rapportering med fokus på væsentlige forhold og sammenlignelighed. \\
    \bottomrule
  \end{tabularx}
  \TableSource{\parencite{EU2022CSRD,EU2023ESRS,EU2020Taxonomy,GRI2021}}
\end{table}

\subsubsection{Lovhenvisninger og paragrafformater (eksempler)}
\label{subsec:lovhenvisninger-og-paragrafformater}

For at gøre reguleringen operationel i rapporteringen kan centrale retskilder bindes til konkrete artikelhenvisninger. EU-lovgivning angives typisk med artikelnumre, mens danske love bruger paragraftegn. Tabel \ref{tab:lovhenvisninger-eksempler} viser udvalgte eksempler på formater og relevans.
\begin{table}[t]
  \caption{Eksempler på lovhenvisninger, der gør juridiske krav operationelle i rapporteringen.}
  \label{tab:lovhenvisninger-eksempler}
  \begin{tabularx}{\textwidth}{l l X}
    \toprule
    Retskilde & Henvisning & Relevans for ESG-rapportering \\
    \midrule
    CSRD (Dir. EU 2022/2464) & \lawart{19a}, \lawart{29a} & Krav til bæredygtighedsrapportering i års- og koncernrapport. \\
    EU-taksonomien (Reg. EU 2020/852) & \lawart{8} & Oplysning om taksonomiforenelige aktiviteter og nøgletal. \\
    GDPR (Reg. EU 2016/679) & \lawart{5} & Principper for dataminimering og lovlig behandling af persondata i ESG-data. \\
    \bottomrule
  \end{tabularx}
  \TableSource{\parencite{EU2022CSRD,EU2020Taxonomy,EU2016GDPR}}
  \TableNote{Paragraffer i dansk lovgivning angives typisk med \lawpar{...}.}
\end{table}
