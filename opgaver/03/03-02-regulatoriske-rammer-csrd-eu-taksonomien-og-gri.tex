\subsection{Regulatoriske rammer: CSRD, EU-taksonomien og GRI}
\label{subsec:regulatoriske-rammer-csrd-eu-taksonomien-og-gri}

De regulatoriske rammer for ESG-rapportering i EU består af flere sammenhængende elementer. CSRD udvider rapporteringspligten og gør bæredygtighedsrapportering til en integreret del af virksomhedernes officielle rapportering. Direktivet fastsætter, at rapporteringen skal ske efter ESRS og i et standardiseret, digitalt format med XBRL-tagging, og implementeringen er trinvist indfaset og politisk justeret gennem den såkaldte stop-the-clock-aftale.\parencite{EU2022CSRD,EU2023ESRS,Erhvervsstyrelsen2025StopClock}

ESRS konkretiserer de oplysninger, virksomhederne skal levere, og giver struktur til sammenhængen mellem strategi, risici, mål og resultater. Standarderne operationaliserer kravet om dobbelt væsentlighed og etablerer en fælles logik for datagrundlag og rapportering.\parencite{EU2023ESRS}

Dobbelt væsentlighed betyder, at virksomheder skal rapportere både deres påvirkning af mennesker og miljø og hvordan bæredygtighedsforhold påvirker virksomheden finansielt. Det udvider datakravet til værdikæde, governance og sociale forhold og gør materialitetsprocessen til en central del af rapporteringen.\parencite{EU2023ESRS}

EU-taksonomien supplerer rapporteringen ved at definere, hvilke økonomiske aktiviteter der kan betragtes som miljømæssigt bæredygtige. Rammeværket kræver, at virksomheder kan dokumentere overensstemmelse med taksonomien og derved knytte finansielle aktiviteter til konkrete miljømæssige mål.\parencite{EU2020Taxonomy}

GRI er et globalt, frivilligt rammeværk, som anvendes bredt til sammenlignelig ESG-rapportering. Det tilbyder en struktureret tilgang til indikatorer og narrativer og bruges ofte som supplement til regulatoriske krav.\parencite{GRI2021}
\begin{displayquote}
ESG does not currently benefit from a universally accepted common set of standards.
\quoteattrib{\textcite{ESGBook}}
\end{displayquote}

\subsubsection{Regulatorisk udvikling, indskærpning og tidshorisonter}
\label{subsec:regulatorisk-udvikling-og-horisonter}

Udviklingen i EU kan læses som en gradvis indskærpning fra principbaserede oplysningskrav til standardiseret, digital og revisorpåtegnet rapportering. NFRD introducerede de første obligatoriske ikke-finansielle disclosures for store virksomheder, mens CSRD udvider omfanget markant og knytter rapporteringen til ESRS og digitale tags.\parencite{EU2014NFRD,EU2022CSRD,EU2023ESRS} Samtidig har EU-taksonomien defineret, hvad der tæller som bæredygtig aktivitet, og SFDR har gjort finanssektoren til en drivkraft for ESG-data i værdikæden.\parencite{EU2020Taxonomy,EU2019SFDR}

Næste lag er due diligence: CSDDD flytter fokus fra rapportering til aktiv risikostyring og ansvar i værdikæder, hvilket udvider ESG fra disclosure til løbende governance.\parencite{EU2024CSDDD} Den politiske horisont viser samtidig en kalibrering af tempoet gennem stop-the-clock og Omnibus-tiltag, hvor krav forenkles og udsættes uden at ændre den overordnede retning mod sporbarhed og dokumenteret væsentlighed.\parencite{Erhvervsstyrelsen2025StopClock,GibsonDunn2025SimplifiedESRS} På længere sigt peger ambitionen om interoperabilitet med internationale standarder som ISSB på, at ESG-rapportering i stigende grad integreres med finansiel rapportering.\parencite{IFRS2023S1}

Reguleringen virker i et samspil med marked og civilsamfund: standardiserede ESG-data bliver et socialt sammenligningsgrundlag for investorer, kunder og offentlige indkøb, og legitimitet forhandles i et felt, hvor målinger, ratinger og narrativer bruges strategisk.\parencite{OECD2020ESG,BusinessRoundtable2019,Economist2022Broken} Det betyder, at juridiske krav ikke alene handler om compliance, men også om adgang til kapital, kontrakter og omdømme, hvilket forklarer, hvorfor virksomheder reagerer på krav også før de bliver formelt pligtige.

Tabel \ref{tab:rammer-sammenligning} opsummerer de centrale forskelle mellem rammerne og deres implikationer for datakrav og rapporteringslogik.
\begin{table}[t]
  \caption{Sammenligning af centrale rammer for ESG-rapportering.}
  \label{tab:rammer-sammenligning}
  \begin{tabularx}{\textwidth}{l X X}
    \toprule
    Rammeværk & Status og formål & Implikation for rapportering \\
    \midrule
    CSRD/ESRS & Obligatorisk og indfaset i bølger; etablerer standardiserede ESG-krav. & Dobbelt væsentlighed og sammenhæng mellem strategi, risici og mål med krav om digital rapportering. \\
    EU-taksonomien & Obligatorisk supplement; klassificerer miljømæssigt bæredygtige aktiviteter. & Kræver dokumentation for aktiviteters bidrag og overensstemmelse med minimumsgarantier. \\
    GRI & Frivilligt rammeværk, anvendes globalt. & Indikatorbaseret rapportering med fokus på væsentlige forhold og sammenlignelighed. \\
    \bottomrule
  \end{tabularx}
  \TableSource{\parencite{EU2022CSRD,EU2023ESRS,EU2020Taxonomy,GRI2021}}
\end{table}

\subsubsection{Lovhenvisninger og paragrafformater (eksempler)}
\label{subsec:lovhenvisninger-og-paragrafformater}

For at gøre reguleringen operationel i rapporteringen kan centrale retskilder bindes til konkrete artikelhenvisninger. EU-lovgivning angives typisk med artikelnumre, mens danske love bruger paragraftegn. Tabel \ref{tab:lovhenvisninger-eksempler} viser udvalgte eksempler på formater og relevans.
\begin{table}[t]
  \caption{Eksempler på lovhenvisninger og format.}
  \label{tab:lovhenvisninger-eksempler}
  \begin{tabularx}{\textwidth}{l l X}
    \toprule
    Retskilde & Henvisning & Relevans for ESG-rapportering \\
    \midrule
    CSRD (Dir. EU 2022/2464) & \lawart{19a}, \lawart{29a} & Krav til bæredygtighedsrapportering i års- og koncernrapport. \\
    EU-taksonomien (Reg. EU 2020/852) & \lawart{8} & Oplysning om taksonomiforenelige aktiviteter og nøgletal. \\
    GDPR (Reg. EU 2016/679) & \lawart{5} & Principper for dataminimering og lovlig behandling af persondata i ESG-data. \\
    \bottomrule
  \end{tabularx}
  \TableSource{\parencite{EU2022CSRD,EU2020Taxonomy,EU2016GDPR}}
  \TableNote{Paragraffer i dansk lovgivning angives typisk med \lawpar{...}.}
\end{table}
