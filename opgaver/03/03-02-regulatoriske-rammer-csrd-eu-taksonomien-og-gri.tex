\subsection{Regulatoriske rammer: CSRD, EU-taksonomien og GRI}
\label{subsec:regulatoriske-rammer-csrd-eu-taksonomien-og-gri}

De regulatoriske rammer for \ESG-rapportering i EU best\aa r af flere sammenh\ae ngende elementer. \CSRD udvider rapporteringspligten og g\o r b\ae redygtighedsrapportering til en integreret del af virksomhedernes officielle rapportering. Direktivet fasts\ae tter, at rapporteringen skal ske efter ESRS og i et standardiseret, digitalt format, og implementeringen er trinvist indfaset og politisk justeret gennem den s\aa kaldte stop-the-clock-aftale.\parencite{EU2022CSRD,EU2023ESRS,Erhvervsstyrelsen2025StopClock}

ESRS konkretiserer de oplysninger, virksomhederne skal levere, og giver struktur til sammenh\ae ngen mellem strategi, risici, m\aa l og resultater. Standarderne operationaliserer kravet om dobbelt v\ae sentlighed og etablerer en f\ae lles logik for datagrundlag og rapportering.\parencite{EU2023ESRS}

EU-taksonomien supplerer rapporteringen ved at definere, hvilke \o konomiske aktiviteter der kan betragtes som milj\o m\ae ssigt b\ae redygtige. Rammev\ae rket kr\ae ver, at virksomheder kan dokumentere alignment og derved knytte finansielle aktiviteter til konkrete milj\o m\ae ssige m\aa l.\parencite{EU2020Taxonomy}

GRI er et globalt, frivilligt rammev\ae rk, som anvendes bredt til sammenlignelig \ESG-rapportering. Det tilbyder en struktureret tilgang til indikatorer og narrativer og bruges ofte som supplement til regulatoriske krav.\parencite{GRI2021}

Tabel \ref{tab:rammer-sammenligning} opsummerer de centrale forskelle mellem rammerne og deres implikationer for datakrav og rapporteringslogik.
\begin{table}[t]
  \caption{Sammenligning af centrale rammer for \ESG-rapportering.}
  \label{tab:rammer-sammenligning}
  \begin{tabularx}{\textwidth}{l X X X}
    \toprule
    Rammev\ae rk & Status og m\aa lgruppe & Form\aa l og datakrav & Rapporteringslogik \\
    \midrule
    \CSRD/ESRS & Obligatorisk for omfattede virksomheder, indfaset i b\o lger. & Standardiseret b\ae redygtighedsrapportering og digital rapportering. & Dobbelt v\ae sentlighed og sammenh\ae ng mellem strategi, risici og m\aa l. \\
    EU-taksonomien & Obligatorisk supplement for omfattede virksomheder og finansielle akt\o rer. & Klassifikation af milj\o m\ae ssigt b\ae redygtige aktiviteter og rapportering om alignment. & Fokus p\aa aktiviteters milj\o m\ae ssige bidrag og minimumsgarantier. \\
    GRI & Frivilligt rammev\ae rk, ofte anvendt globalt. & Retningslinjer for sammenlignelig og transparent \ESG-rapportering. & Indikatorbaseret rapportering med fokus p\aa v\ae sentlige forhold. \\
    \bottomrule
  \end{tabularx}
  \TableSource{\parencite{EU2022CSRD,EU2023ESRS,EU2020Taxonomy,GRI2021}}
\end{table}
