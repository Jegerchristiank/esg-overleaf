\subsection{ESG-rapportering i sundhedssektoren}
\label{subsec:esg-rapportering-i-sundhedssektoren}

\ESG-rapportering i sundhedssektoren omfatter milj\o-, sociale og governance-forhold, der er t\ae t forbundet med drift, forsyningsk\ae der og patientsikkerhed. Litteraturen peger p\aa, at sektoren m\aa koble b\ae redygtighedsarbejde med digital transformation og klare processer for at kunne dokumentere resultater og ansvar.\parencite{Sepetis2024Healthcare,Bosco2024ESGHealth} Samtidig viser One Health-litteraturen, at ESG-rammer overlapper med bredere sundheds- og milj\o -hensyn, hvilket understreger behovet for tv\ae rg\aa ende data og governance.\parencite{Vegro2025OneHealth}

Empirisk er sektoren ansvarlig for et betydeligt klimaaftryk. Globale vurderinger estimerer, at sundhedssektoren st\aa r for ca. 4,4~\% af netto-udledningerne (omtrent 2~Gt CO$_2$e), og at mere end halvdelen af aftrykket kommer fra energiforbrug. Udledningerne fordeler sig omtrent som 17~\% scope~1, 12~\% scope~2 og 71~\% scope~3, hvilket indikerer, at indirekte udledninger i forsyningsk\ae den er dominerende.\parencite{HCWH2019ClimateFootprint}

Der er ogs\aa markante \ESG-forhold uden for klimaomr\aa det. WHO angiver, at ca. 85~\% af healthcare-affald er ikke-farligt, mens ca. 15~\% klassificeres som farligt, hvilket stiller s\ae rlige krav til dokumentation og behandling.\parencite{WHO2024HealthCareWaste} Den sociale dimension er ligeledes tydelig, idet WHO estimerer mindst 115.500 d\o dsfald blandt health and care workers under COVID-19 over 18 m\aa neder.\parencite{WHO2021HealthWorkerDeaths}

Tabel \ref{tab:esg-sundhedssektor-noegletal} sammenfatter de centrale n\o gletal og viser, at \ESG-rapportering i sektoren kr\ae ver data p\aa tv\ae rs af energi, forsyningsk\ae de, affald og arbejdsmilj\o.
\begin{table}[t]
  \caption{Empiriske noegletal for \ESG i sundhedssektoren (uddrag).}
  \label{tab:esg-sundhedssektor-noegletal}
  \begin{tabularx}{\textwidth}{l X X}
    \toprule
    Tema & N\o gletal & Betydning for rapportering \\
    \midrule
    Klimaaftryk & Ca. 4,4~\% af globale netto-udledninger (omtrent 2~Gt CO$_2$e). & Kr\ae ver opg\o relse af emissioner og reduktionsplaner p\aa tv\ae rs af drift og forsyningsk\ae de. \\
    Energi og scope & Over halvdelen af aftrykket kommer fra energi; ca. 17~\% scope~1, 12~\% scope~2 og 71~\% scope~3. & Understreger behovet for energidata og systematisk indsamling af leverand\o r- og indk\o bsdata. \\
    Affald & Ca. 85~\% ikke-farligt affald og 15~\% farligt affald. & Kr\ae ver sporbarhed for affaldsstr\o mme og dokumentation for behandling og compliance. \\
    Arbejdsvilk\aa r & Mindst 115.500 d\o dsfald blandt health and care workers under COVID-19 (18 m\aa neder). & Peger p\aa behov for robuste sociale indikatorer om arbejdsmilj\o og sikkerhed. \\
    \bottomrule
  \end{tabularx}
  \TableSource{\parencite{HCWH2019ClimateFootprint,WHO2024HealthCareWaste,WHO2021HealthWorkerDeaths}}
\end{table}

Samlet peger empiri og litteratur p\aa, at \ESG-rapportering i sundhedssektoren m\aa d\ae kke b\aa de direkte driftstal og indirekte forhold i forsyningsk\ae den, samt sociale og organisatoriske risici. Det skaber et behov for standardiserede processer, der kan oms\ae tte kvalitative krav til sammenlignelige data og sporbar dokumentation.
