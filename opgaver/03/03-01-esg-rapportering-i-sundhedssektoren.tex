\subsection{ESG-rapportering i sundhedssektoren}
\label{subsec:esg-rapportering-i-sundhedssektoren}

Analytisk set er ESG-rapportering i sundhedssektoren en styringsopgave under høj kompleksitet, fordi data og ansvar er fordelt på tværs af drift, forsyningskæder og arbejdsmiljø. ESG-rapportering i sundhedssektoren omfatter miljø-, sociale og governance-forhold, der er tæt forbundet med drift, forsyningskæder og patientsikkerhed. Ifølge \textcite{Sepetis2024Healthcare} må sektoren koble bæredygtighedsarbejde med digital transformation og klare processer for at kunne dokumentere resultater og ansvar, hvilket også understreges af sektorspecifikke processkort.\parencite{Bosco2024ESGHealth} Ifølge \textcite{Vegro2025OneHealth} overlapper ESG-rammer med bredere sundheds- og miljøhensyn, hvilket understreger behovet for tværgående data og governance.

Empirisk er sektoren ansvarlig for et betydeligt klimaaftryk. Globale vurderinger estimerer, at sundhedssektoren står for ca. 4,4~\% af netto-udledningerne (omtrent 2~Gt CO$_2$e), og at mere end halvdelen af aftrykket kommer fra energiforbrug. Udledningerne fordeler sig omtrent som 17~\% scope~1, 12~\% scope~2 og 71~\% scope~3, hvilket indikerer, at indirekte udledninger i forsyningskæden er dominerende.\parencite{HCWH2019ClimateFootprint}

Der er også markante ESG-forhold uden for klimaområdet. WHO angiver, at ca. 85~\% af affald fra sundhedssektoren er ikke-farligt, mens ca. 15~\% klassificeres som farligt, hvilket stiller særlige krav til dokumentation og behandling.\parencite{WHO2024HealthCareWaste} Den sociale dimension er ligeledes tydelig, idet WHO estimerer mindst 115.500 dødsfald blandt sundheds- og omsorgspersonale under COVID-19 over 18 måneder.\parencite{WHO2021HealthWorkerDeaths}

Tabel \ref{tab:esg-sundhedssektor-noegletal} sammenfatter de centrale nøgletal og viser, at ESG-rapportering i sektoren kræver data på tværs af energi, forsyningskæde, affald og arbejdsmiljø.
\begin{table}[h]
  \caption{Empiriske nøgletal, der viser hvorfor ESG-rapportering kræver data på tværs af drift og forsyningskæde.}
  \label{tab:esg-sundhedssektor-noegletal}
  \begin{tabularx}{\textwidth}{l X X}
    \toprule
    Tema & Nøgletal & Betydning for rapportering \\
    \midrule
    Klimaaftryk & Ca. 4,4~\% af globale netto-udledninger (omtrent 2~Gt CO$_2$e). & Kræver opgørelse af emissioner og reduktionsplaner på tværs af drift og forsyningskæde. \\
    Energi og scope & Over halvdelen af aftrykket kommer fra energi; ca. 17~\% scope~1, 12~\% scope~2 og 71~\% scope~3. & Understreger behovet for energidata og systematisk indsamling af leverandør- og indkøbsdata. \\
    Affald & Ca. 85~\% ikke-farligt affald og 15~\% farligt affald. & Kræver sporbarhed for affaldsstrømme og dokumentation for behandling og compliance. \\
    Arbejdsvilkår & Mindst 115.500 dødsfald blandt sundheds- og omsorgspersonale under COVID-19 (18 måneder). & Peger på behov for robuste sociale indikatorer om arbejdsmiljø og sikkerhed. \\
    \bottomrule
  \end{tabularx}
  \TableSource{\parencite{HCWH2019ClimateFootprint,WHO2024HealthCareWaste,WHO2021HealthWorkerDeaths}}
\end{table}

Samlet peger empiri og litteratur på, at ESG-rapportering i sundhedssektoren må dække både direkte driftstal og indirekte forhold i forsyningskæden samt sociale og organisatoriske risici. Det skaber et behov for standardiserede processer, der kan omsætte kvalitative krav til sammenlignelige data og sporbar dokumentation.

Dette behov sætter scenen for at forstå de regulatoriske rammer, som definerer, hvilke data der faktisk forventes rapporteret (jf. \ref{subsec:regulatoriske-rammer-csrd-eu-taksonomien-og-gri}).
