\section*{Abstract}
\addcontentsline{toc}{section}{Abstract}

\begin{otherlanguage}{english}
This report looks closely at environmental, social, and governance (ESG) reporting in the healthcare sector, with a focus on regulatory frameworks, organizational implementation, and economic relevance for small and medium-sized enterprises (SMEs). CSRD (the EU directive on sustainability reporting) and ESRS (the accompanying reporting standards) push reporting away from narrative and toward traceable data through digital tagging; the report examines how an ESG-as-a-Service model (software plus service delivery) and an MVP (a first functional version) turn those requirements into data capture, traceability, and report outputs.\parenciteshort{EU2022CSRD,EU2023ESRS}

The study combines document analysis of CSRD, ESRS, the EU Taxonomy (a classification of sustainable activities), and GRI (a voluntary reporting standard) with sector sources on climate footprint, waste, and working conditions. The sector evidence shows fragmented data across operations and supply chains; therefore the study includes a case-based pilot and an artefact analysis of the MVP's data flow and functionality.\parenciteshort{HCWH2019ClimateFootprint,WHO2024HealthCareWaste,WHO2021HealthWorkerDeaths}

The analysis suggests that ESG operates as an organizational standard that strengthens legitimacy and comparability but also introduces a risk of decoupling between reporting and practice.\parencite{BrunssonWorldOfStandards,Berg2022AggregateConfusion} ESG-as-a-Service can reduce that risk by providing standardised workflows, validation, and audit trails. The value proposition depends on prioritising material topics and embedding data practices in the organization.

The conclusion is that the MVP seems practically feasible for SMEs, but scaling requires gradual implementation, stronger integrations, and a steady focus on data quality. The report therefore recommends a phased reporting approach, starting with minimum modules and expanding as capacity and maturity grow.
\end{otherlanguage}
\clearpage
% Antagelse: begrebsliste placeres i fortekst før indledning for tidlig afklaring.
\section*{Begrebsliste}
\addcontentsline{toc}{section}{Begrebsliste}

\begin{description}
  \item[API] Programmeringsgrænseflade, der gør automatiseret dataudveksling mellem systemer mulig.
  \item[Audit-log (auditspor)] Log over ændringer i data og beregninger, som gør sporbarhed og efterprøvning mulig.
  \item[CSRD] EU-direktiv, der udvider og standardiserer krav til bæredygtighedsrapportering.\parenciteshort{EU2022CSRD}
  \item[CSR] Corporate Social Responsibility; ramme for virksomheders sociale og etiske ansvar.\parencite{Carroll1991Pyramid}
  \item[CSV] Kommasepareret filformat til strukturerede dataudtræk.
  \item[Dobbelt væsentlighed] Krav om at rapportere både påvirkning af mennesker og miljø og finansiel påvirkning på virksomheden.\parenciteshort{EU2023ESRS}
  \item[ESG] Samlebetegnelse for miljø-, sociale- og governanceforhold, der rapporteres og styres i organisationer.
  \item[ESG-as-a-Service] Kombineret software- og serviceleverance, der omsætter standardkrav til datafelter, kontroller og rapportoutput.
  \item[ESRS] Europæiske standarder, som operationaliserer CSRD-kravene til rapportering.\parenciteshort{EU2023ESRS}
  \item[EU-taksonomien] EU's klassifikationssystem for, hvilke økonomiske aktiviteter der kan anses som miljømæssigt bæredygtige.\parenciteshort{EU2020Taxonomy}
  \item[GDPR] EU's databeskyttelsesforordning, der regulerer behandling af personoplysninger.\parenciteshort{EU2016GDPR}
  \item[GRI] Globalt rammeværk med standarder for bæredygtighedsrapportering.\parenciteshort{GRI2021}
  \item[KPI] Key Performance Indicator; måltal, der følger udviklingen i centrale resultater.
  \item[LCA] Life Cycle Assessment; metode til at vurdere miljøpåvirkning på tværs af en livscyklus.
  \item[Materialitet (væsentlighed)] Vurdering af hvilke temaer der er væsentlige for virksomhedens påvirkning og/eller økonomi og derfor prioriteres i rapporteringen.\parenciteshort{EU2023ESRS}
  \item[MVP] Minimum Viable Product; første funktionsdygtige version af en løsning, der demonstrerer centrale processer og output.
  \item[SaaS] Software as a Service; software leveret som løbende tjeneste.
  \item[Scope 1/2/3] Kategorier for drivhusgasudledninger: direkte emissioner (Scope 1), indkøbt energi (Scope 2) og øvrige værdikædeemissioner (Scope 3).\parencite{GHGProtocol2004}
  \item[Shared value] Perspektiv om at skabe forretningsværdi ved at løse samfundsmæssige udfordringer.\parencite{PorterKramer2011CSV}
  \item[SMV] Små og mellemstore virksomheder efter EU's størrelseskriterier.\parencite{Virksomhedsguiden2025SMVDefinition}
  \item[Stakeholder-/shareholder-perspektiv] Stakeholder fokuserer på flere interessenters værdi; shareholder prioriterer aktionærværdi.\parencite{Freeman1984Stakeholder,Friedman1970Profits}
  \item[Stop-the-clock-aftale] Politisk aftale om at udskyde CSRD's rapporteringskrav i indfasningen.\parencite{Erhvervsstyrelsen2025StopClock}
  \item[Translasjon] Proces hvor ideer og standarder oversættes til lokale praksisser og ændres i implementeringen.\parencite{RoevikTrenderTranslasjoner,TranslationTheoryKnowledgeTransfer}
  \item[Triple bottom line] Værdiramme, der vurderer performance på økonomi, miljø og sociale forhold.\parencite{Elkington1998TripleBottomLine}
  \item[VSME] Frivillig EU-standard for SMV'ers bæredygtighedsrapportering med basis- og udvidet modul.\parencite{Virksomhedsguiden2025VSMEIntro,Erhvervsstyrelsen2025ESGTemplate}
  \item[XBRL] Standardiseret digitalt rapporteringsformat med data-tags til finansielle og ESG-oplysninger.\parenciteshort{EU2022CSRD}
\end{description}
