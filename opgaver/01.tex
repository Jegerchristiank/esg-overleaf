\section*{Abstract}
\addcontentsline{toc}{section}{Abstract}

\begin{otherlanguage}{english}
This report examines environmental, social, and governance (ESG) reporting in the healthcare sector with a focus on regulatory frameworks, organizational implementation, and economic relevance for small and medium-sized enterprises (SMEs). As \textcite{EU2022CSRD} and \textcite{EU2023ESRS} set out, reporting depends on traceable data and standardised outputs; the goal is to assess how an ESG-as-a-Service concept and an MVP translate those requirements into data collection, traceability, and report outputs.

The study combines document analysis of CSRD, ESRS, the EU Taxonomy, and GRI with empirical sector sources on climate footprint, waste, and working conditions. As \textcite{HCWH2019ClimateFootprint} and \textcite{WHO2024HealthCareWaste} show, healthcare data are fragmented across operations and supply chains; the analysis therefore includes a case-based pilot of the author's company and an artefact analysis of the MVP's data flow and functionality.

The analysis shows that ESG operates as an organizational standard that enhances legitimacy and comparability but also introduces a risk of decoupling between reporting and practice. ESG-as-a-Service can mitigate this risk by providing standardized workflows, validation, and audit trails. The value proposition depends on prioritizing material topics and embedding data practices in the organization.

The conclusion is that the MVP demonstrates practical feasibility for SMEs, but scaling requires gradual implementation, stronger integrations, and sustained focus on data quality. The report therefore recommends a phased reporting approach, starting with minimum modules and expanding as capacity and maturity grow.
\end{otherlanguage}
\clearpage
% Antagelse: begrebsliste placeres i fortekst før indledning for tidlig afklaring.
\section*{Begrebsliste}
\addcontentsline{toc}{section}{Begrebsliste}

\begin{description}
  \item[API] Programmeringsgrænseflade, der muliggør automatiseret udveksling af data mellem systemer.
  \item[Audit-log (auditspor)] Log over ændringer i data og beregninger, der understøtter sporbarhed og efterprøvning.
  \item[CSRD] EU-direktiv, der udvider og standardiserer krav til bæredygtighedsrapportering.\parencite{EU2022CSRD}
  \item[CSR] Corporate Social Responsibility; ramme for virksomheders sociale og etiske ansvar.\parencite{Carroll1991Pyramid}
  \item[CSV] Komma-separeret filformat til strukturerede dataudtræk.
  \item[Dobbelt væsentlighed] Krav om at rapportere både påvirkning af mennesker og miljø og finansiel påvirkning på virksomheden.\parencite{EU2023ESRS}
  \item[ESG] Samlebetegnelse for miljø-, sociale- og governanceforhold, der rapporteres og styres i organisationer.
  \item[ESG-as-a-Service] Kombineret software- og serviceleverance, der omsætter standardkrav til datafelter, kontroller og rapportoutput.
  \item[ESRS] Europæiske standarder, som operationaliserer CSRD-kravene til rapportering.\parencite{EU2023ESRS}
  \item[EU-taksonomien] EU's klassifikationssystem for, hvilke økonomiske aktiviteter der kan anses som miljømæssigt bæredygtige.\parencite{EU2020Taxonomy}
  \item[GDPR] EU's databeskyttelsesforordning, der regulerer behandling af personoplysninger.\parencite{EU2016GDPR}
  \item[GRI] Globalt rammeværk med standarder for bæredygtighedsrapportering.\parencite{GRI2021}
  \item[KPI] Key Performance Indicator; måltal, der følger udviklingen i centrale resultater.
  \item[LCA] Life Cycle Assessment; metode til at vurdere miljøpåvirkning på tværs af en livscyklus.
  \item[Materialitet (væsentlighed)] Vurdering af hvilke temaer der er væsentlige for virksomhedens påvirkning og/eller økonomi og derfor prioriteres i rapporteringen.\parencite{EU2023ESRS}
  \item[MVP] Minimum Viable Product; første funktionsdygtige version af softwareløsningen, der demonstrerer centrale processer og output.
  \item[SaaS] Software as a Service; software leveret som løbende tjeneste.
  \item[Scope 1/2/3] Kategorier for drivhusgasudledninger: direkte emissioner (Scope 1), indkøbt energi (Scope 2) og øvrige værdikædeemissioner (Scope 3).\parencite{GHGProtocol2004}
  \item[Shared value] Perspektiv om at skabe forretningsværdi ved at løse samfundsmæssige udfordringer.\parencite{PorterKramer2011CSV}
  \item[SMV] Små og mellemstore virksomheder efter EU's størrelseskriterier.\parencite{Virksomhedsguiden2025SMVDefinition}
  \item[Stakeholder-/shareholder-perspektiv] Stakeholder fokuserer på flere interessenters værdi; shareholder prioriterer aktionærværdi.\parencite{Freeman1984Stakeholder,Friedman1970Profits}
  \item[Stop-the-clock-aftale] Politisk aftale om at udskyde CSRD's rapporteringskrav i indfasningen.\parencite{Erhvervsstyrelsen2025StopClock}
  \item[Translasjon] Proces, hvor ideer og standarder oversættes til lokale praksisser og ændres i implementeringen.\parencite{RoevikTrenderTranslasjoner,TranslationTheoryKnowledgeTransfer}
  \item[Triple bottom line] Værdiramme, der vurderer performance på økonomi, miljø og sociale forhold.\parencite{Elkington1998TripleBottomLine}
  \item[VSME] Frivillig EU-standard for SMV'ers bæredygtighedsrapportering med basis- og udvidet modul.\parencite{Virksomhedsguiden2025VSMEIntro,Virksomhedsguiden2025VSMEModules}
  \item[XBRL] Standardiseret digitalt rapporteringsformat med data-tags til finansielle og ESG-oplysninger.\parencite{EU2022CSRD}
\end{description}
