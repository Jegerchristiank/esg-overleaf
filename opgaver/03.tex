\section{Kontekst og rammer}
\label{sec:kontekst-og-rammer}

Den sektorielle og regulatoriske ramme former ESG-rapportering i sundhedssektoren og synliggør, hvor krav, data og organisatoriske kapaciteter ikke passer sammen. Rammen omfatter sektorens ESG-profil og nøgletal, de regulatoriske krav, SMV'ers ressourcevilkår og ESG-as-a-Service som respons på identificerede barrierer. Det giver et empirisk og regulatorisk grundlag for at forstå kløften mellem krav og praksis.
\subsection{ESG-rapportering i sundhedssektoren}
\label{subsec:esg-rapportering-i-sundhedssektoren}

Analytisk set er ESG-rapportering i sundhedssektoren en styringsopgave under høj kompleksitet, fordi data og ansvar er fordelt på tværs af drift, forsyningskæder og arbejdsmiljø. ESG-rapportering i sundhedssektoren omfatter miljø-, sociale og governance-forhold, der er tæt forbundet med drift, forsyningskæder og patientsikkerhed. Ifølge \textcite{Sepetis2024Healthcare} må sektoren koble bæredygtighedsarbejde med digital transformation og klare processer for at kunne dokumentere resultater og ansvar, hvilket også understreges af sektorspecifikke processkort.\parencite{Bosco2024ESGHealth} Det peger på, at governance i praksis bliver en data- og procesdisciplin, ikke blot en compliance-opgave. Ifølge \textcite{Vegro2025OneHealth} overlapper ESG-rammer med bredere sundheds- og miljøhensyn, hvilket understreger behovet for tværgående data og governance.

Som beskrevet i baggrund og motivation (afsnit \ref{subsec:baggrund-og-motivation}) er sektorens klimaaftryk domineret af scope~3, affaldsprofilen kræver fraktionsspecifik dokumentation, og arbejdsmiljørisici er empirisk veldokumenterede.\parencite{HCWH2019ClimateFootprint,WHO2024HealthCareWaste,WHO2021HealthWorkerDeaths} Det betyder, at rapporteringen skal kunne samle leverandør-, affalds- og arbejdsmiljødata i konsistente formater, også for SMV'er med begrænset datakapacitet.

De centrale nøgletal opsummeres i tabel \ref{tab:esg-sundhedssektor-noegletal} og underbygger, at ESG-rapportering i sektoren kræver data på tværs af energi, forsyningskæde, affald og arbejdsmiljø.
\begin{table}[h]
  \caption{Empiriske nøgletal, der viser hvorfor ESG-rapportering kræver data på tværs af drift og forsyningskæde.}
  \label{tab:esg-sundhedssektor-noegletal}
  \begin{tabularx}{\textwidth}{l X X}
    \toprule
    Tema & Nøgletal & Betydning for rapportering \\
    \midrule
    Klimaaftryk & Ca. 4,4~\% af globale netto-udledninger (omtrent 2~Gt CO$_2$e). & Kræver opgørelse af emissioner og reduktionsplaner på tværs af drift og forsyningskæde. \\
    Energi og scope & Over halvdelen af aftrykket kommer fra energi; ca. 17~\% scope~1, 12~\% scope~2 og 71~\% scope~3. & Understreger behovet for energidata og systematisk indsamling af leverandør- og indkøbsdata. \\
    Affald & Ca. 85~\% ikke-farligt affald og 15~\% farligt affald. & Kræver sporbarhed for affaldsstrømme og dokumentation for behandling og compliance. \\
    Arbejdsvilkår & Mindst 115.500 dødsfald blandt sundheds- og omsorgspersonale under COVID-19 (18 måneder). & Peger på behov for robuste sociale indikatorer om arbejdsmiljø og sikkerhed. \\
    \bottomrule
  \end{tabularx}
  \TableSource{\parencite{HCWH2019ClimateFootprint,WHO2024HealthCareWaste,WHO2021HealthWorkerDeaths}}
\end{table}

Samlet peger empiri og litteratur på, at ESG-rapportering i sundhedssektoren må dække både direkte driftstal og indirekte forhold i forsyningskæden samt sociale og organisatoriske risici. Det skaber et behov for standardiserede processer, der kan omsætte kvalitative krav til sammenlignelige data og sporbar dokumentation.

Behovet for standardiserede processer gør de regulatoriske rammer til en central drivkraft for, hvilke data der forventes rapporteret.
\subsection{Regulatoriske rammer: CSRD, EU-taksonomien og GRI}
\label{subsec:regulatoriske-rammer-csrd-eu-taksonomien-og-gri}

De regulatoriske rammer for ESG-rapportering i EU består af flere sammenhængende elementer. CSRD udvider rapporteringspligten og gør bæredygtighedsrapportering til en integreret del af virksomhedernes officielle rapportering. Direktivet fastsætter, at rapporteringen skal ske efter ESRS, som er vedtaget som delegerede standarder, og at oplysningerne leveres i et standardiseret, digitalt format med XBRL-tagging. Implementeringen er trinvist indfaset og politisk justeret gennem den såkaldte stop-the-clock-aftale.\parencite{EU2022CSRD,EU2023ESRS,Erhvervsstyrelsen2025StopClock}

Indfasningen og de politiske justeringer betyder, at virksomheder skal planlægge rapportering under bevægelige rammer, hvor tidsplaner og præciseringer ændres over tid. Det gør fortolkning og prioritering til en reel del af compliance-arbejdet, særligt for SMV'er med begrænset kapacitet.\parencite{Erhvervsstyrelsen2025StopClock}

ESRS konkretiserer de oplysninger, virksomhederne skal levere, og giver struktur til sammenhængen mellem strategi, risici, mål og resultater. Standarderne operationaliserer kravet om dobbelt væsentlighed og etablerer en fælles logik for datagrundlag og rapportering.\parencite{EU2023ESRS} IFRS S1 etablerer samtidig en investorrettet global baseline, men uden dobbelt væsentlighed, hvilket øger spændet mellem globale standarder og EU's krav og skærper kompleksiteten for SMV'er.\parencite{IFRS2023S1}

Dobbelt væsentlighed betyder, at virksomheder skal rapportere både deres påvirkning af mennesker og miljø og hvordan bæredygtighedsforhold påvirker virksomheden finansielt. Det udvider datakravet til værdikæde, governance og sociale forhold og gør materialitetsprocessen til en central del af rapporteringen.\parencite{EU2023ESRS}

EU-taksonomien supplerer rapporteringen ved at definere, hvilke økonomiske aktiviteter der kan betragtes som miljømæssigt bæredygtige. Rammeværket kræver, at virksomheder kan dokumentere overensstemmelse med taksonomien og derved knytte finansielle aktiviteter til konkrete miljømæssige mål.\parencite{EU2020Taxonomy}

GRI er et globalt, frivilligt rammeværk, som anvendes bredt til sammenlignelig ESG-rapportering. Det tilbyder en struktureret tilgang til indikatorer og narrativer og bruges ofte som supplement til regulatoriske krav.\parencite{GRI2021} OECD peger på, at ESG-data stadig er fragmenterede og vanskelige at sammenligne, hvilket gør frivillige rammer utilstrækkelige uden en stærk datainfrastruktur.\parencite{OECD2020ESG}

ISO's IWA 48 er en international workshop-aftale, der samler principper for ESG-implementering og lægger vægt på standardiserede KPI'er, datakvalitet og rapporteringsprincipper.\parencite{ISO2024IWA48} Den er ikke bindende i EU-retlig forstand, men fungerer som et globalt referencepunkt for, hvordan organisationer kan operationalisere ESG i praksis. Implikationen er, at der opstår en forventning om konsistente målinger, sporbarhed og systematisk dokumentation på tværs af sektorer, hvilket styrker behovet for robuste dataflows og governance. For SMV'er betyder det, at selv frivillige rammer kan drive professionalisering af ESG-data og gøre implementering til en løbende kapacitetsopgave.

\subsubsection{EU-retligt overblik og retskildernes funktion}
\label{subsec:eu-retligt-overblik}

ESG-regimet fungerer som et EU-retligt flerniveau-system, hvor retskilderne har forskellige funktioner og bindende karakter. Det gør det muligt at forstå, hvorfor rapporteringen både kræver juridisk pligt, tekniske standarder og klassifikationslogik:
\begin{enumerate}
  \item \textbf{Direktiv (CSRD).} Etablerer rapporteringspligt og overordnede krav, som skal implementeres i national ret.\parencite{EU2022CSRD}
  \item \textbf{Delegerede standarder (ESRS).} Konkretiserer indhold, datapunkter og struktur, så rapportering bliver sammenlignelig og auditerbar.\parencite{EU2023ESRS}
  \item \textbf{Forordning (EU-taksonomien).} Fastlægger en klassifikation af bæredygtige aktiviteter og kobler rapportering til miljømål og finansielle nøgletal.\parencite{EU2020Taxonomy}
  \item \textbf{Frivillige standarder (GRI).} Udgør et udbredt supplement, især hvor globale interessenter efterspørger sammenlignelige ESG-oplysninger.\parencite{GRI2021}
\end{enumerate}

Det EU-retlige hierarki fremgår af tabel \ref{tab:eu-retligt-hierarki}.
\begin{table}[h]
  \caption{EU-retligt hierarki for ESG-rapportering og dets funktionelle rolle i styringskæden.}
  \label{tab:eu-retligt-hierarki}
  \begin{tabularx}{\textwidth}{l X X}
    \toprule
    Niveau & Retskilde og bindende karakter & Funktion i ESG-regimet \\
    \midrule
    Direktiv & CSRD; bindende mål, implementeres i national ret. & Etablerer rapporteringspligt, indfasning og krav til digital rapportering. \\
    Delegerede standarder & ESRS; bindende teknisk konkretisering. & Definerer datapunkter, struktur og auditbarhed. \\
    Forordning & EU-taksonomien; direkte gældende. & Klassificerer aktiviteter og forbinder rapportering med miljømål og nøgletal. \\
    Frivillige standarder & GRI; ikke-bindende, globalt udbredt. & Supplerer sammenlignelighed, især i globale værdikæder. \\
    \bottomrule
  \end{tabularx}
  \TableSource{\parencite{EU2022CSRD,EU2023ESRS,EU2020Taxonomy,GRI2021}}
\end{table}

\begin{displayquote}
ESG does not currently benefit from a universally accepted common set of standards.
\quoteattrib{\textcite{ESGBook}}
\end{displayquote}

Det gør reguleringen analytisk interessant: på den ene side øger standarderne sammenligneligheden, på den anden side efterlader de et fortolkningsrum, hvor rapportering kan glide over i minimum compliance frem for beslutningsrelevant styring.

\subsubsection{Gennemgående regulatoriske principper}
\label{subsec:gennemgaaende-regulatoriske-principper}

På tværs af retskilderne går en række gentagende principper, der forklarer, hvorfor ESG-rapportering kræver systematik. Proportionalitet og faseindføring betyder, at kravene skaleres og udskydes, hvilket skaber et gradvist, men vedvarende implementeringspres.\parencite{EU2022CSRD,Erhvervsstyrelsen2025StopClock} Dobbelt væsentlighed gør materialitetsprocessen til et kernekrav og udvider datagrundlaget til både påvirkning og finansiel risiko.\parencite{EU2023ESRS} Dokumentations- og sporbarhedskrav følger af standardernes struktur og taksonomiens klassifikationslogik, hvilket nødvendiggør auditspor og konsistente datakilder.\parencite{EU2023ESRS,EU2020Taxonomy} Endelig peger udviklingen i finansielt tilsyn på en risikobaseret forventning om ESG-data, som forstærker kravene uden for den direkte rapporteringspligt.\parencite{Finanstilsynet2025ESGRisk}

\subsubsection{Implementering i dansk ret og praksis}
\label{subsec:implementering-i-dansk-ret}

Som direktiv forudsætter CSRD national implementering. I Danmark sker implementeringen via lov nr 480 af 22/05/2024, som ændrer årsregnskabsloven og relaterede love, og som operationaliseres gennem Erhvervsstyrelsens vejledninger og værktøjer til SMV'er.\parencite{EU2022CSRD,Lov2024CSRD,Aarsregnskabsloven2022,Erhvervsstyrelsen2025ESGTemplate,Virksomhedsguiden2025VSMEIntro} I praksis mødes virksomhederne ofte gennem nationale kanaler som myndigheder, revisorer og finansielle institutioner, hvor bankernes ESG-risikostyring gør data til en forudsætning for kredit, også når rapportering formelt er frivillig.\parencite{DanskErhvervFSR2025CSRDTimeline,GrantThorntonDK2025BankESG,Finanstilsynet2025ESGRisk}

Den juridiske struktur forstærker fokus på at reducere kløften mellem krav og datapraksis: ESG-as-a-Service er ikke blot et teknisk valg, men et regulatorisk mellemled i et flerniveau-regime.

De centrale forskelle mellem rammerne og deres implikationer for datakrav og rapporteringslogik fremgår af tabel \ref{tab:rammer-sammenligning}.
\begin{table}[h]
  \caption{Sammenligning, der viser hvordan retskilderne udfylder forskellige styringsfunktioner i ESG-rapportering.}
  \label{tab:rammer-sammenligning}
  \begin{tabularx}{\textwidth}{l X X}
    \toprule
    Rammeværk & Status og formål & Implikation for rapportering \\
    \midrule
    CSRD (direktiv) & Obligatorisk ramme for rapporteringspligt og indfasning. & Forankrer rapportering i års- og koncernrapport med krav om digital tagging. \\
    ESRS (delegerede standarder) & Tekniske standarder for datapunkter og struktur. & Operationaliserer dobbelt væsentlighed og sikrer sammenlignelighed. \\
    EU-taksonomien & Obligatorisk supplement; klassificerer miljømæssigt bæredygtige aktiviteter. & Kræver dokumentation for aktiviteters bidrag og overensstemmelse med minimumsgarantier. \\
    GRI & Frivilligt rammeværk, anvendes globalt. & Indikatorbaseret rapportering med fokus på væsentlige forhold og sammenlignelighed. \\
    \bottomrule
  \end{tabularx}
  \TableSource{\parencite{EU2022CSRD,EU2023ESRS,EU2020Taxonomy,GRI2021}}
\end{table}

\subsubsection{Lovhenvisninger og paragrafformater (eksempler)}
\label{subsec:lovhenvisninger-og-paragrafformater}

For at gøre reguleringen operationel i rapporteringen kan centrale retskilder bindes til konkrete artikelhenvisninger. EU-lovgivning angives typisk med artikelnumre, mens danske love bruger paragraftegn. Udvalgte eksempler på formater og relevans fremgår af tabel \ref{tab:lovhenvisninger-eksempler}.
\begin{table}[h]
  \caption{Eksempler på lovhenvisninger, der gør juridiske krav operationelle i rapporteringen.}
  \label{tab:lovhenvisninger-eksempler}
  \begin{tabularx}{\textwidth}{l l X}
    \toprule
    Retskilde & Henvisning & Relevans for ESG-rapportering \\
    \midrule
    CSRD (Dir. EU 2022/2464) & \lawart{19a}, \lawart{29a} & Krav til bæredygtighedsrapportering i års- og koncernrapport. \\
    EU-taksonomien (Reg. EU 2020/852) & \lawart{8} & Oplysning om taksonomiforenelige aktiviteter og nøgletal. \\
    GDPR (Reg. EU 2016/679) & \lawart{5} & Principper for dataminimering og lovlig behandling af persondata i ESG-data. \\
    \bottomrule
  \end{tabularx}
  \TableSource{\parencite{EU2022CSRD,EU2020Taxonomy,EU2016GDPR}}
  \TableNote{Paragraffer i dansk lovgivning angives typisk med \lawpar{...}.}
\end{table}

Samlet peger rammerne på et flerniveau-regime med høje krav til data, sporbarhed og fortolkning. Det gør SMV'ers kapacitet og behov for forenkling afgørende.
\subsection{SMV'er og behovet for forenkling}
\label{subsec:smver-og-behovet-for-forenkling}

SMV'er defineres efter EU's størrelsesgrænser og vil ofte ligge uden for den direkte CSRD-pligt, men påvirkes indirekte gennem kundekrav, banker og leverandørrelationer.\parencite{Virksomhedsguiden2025SMVDefinition,PwC2025GuideESGSMV,GrantThorntonDK2025Omnibus} Det betyder, at frivillighed i praksis ofte bliver en markedsforventning snarere end et reelt valg.

EU har derfor udviklet en frivillig VSME-standard (Voluntary Sustainability Reporting Standard for SMEs), der skal sikre ensartet dataudveksling og forhindre uforholdsmæssige datakrav fra større virksomheder.\parencite{Virksomhedsguiden2025VSMEIntro} VSME er opbygget af et basismodul og et udvidet modul, hvor basismodullets 11 datapunkter kan anvendes som minimumsniveau.\parencite{Virksomhedsguiden2025VSMEModules} Standarden kræver ikke en dobbelt væsentlighedsanalyse, hvilket reducerer kompleksitet og ressourceforbrug.\parencite{Virksomhedsguiden2025NoDualMateriality} Erhvervsstyrelsen har samtidig udviklet en skabelon, der samler datapunkterne og understøtter ensartet rapportering.\parencite{Erhvervsstyrelsen2025ESGTemplate}

Indfasningen af CSRD er trinvist implementeret, og den seneste stop-the-clock-aftale udskyder rapporteringskrav for flere virksomhedstyper, hvilket giver SMV'er mere tid, men også skaber usikkerhed om krav og timing.\parencite{Erhvervsstyrelsen2025StopClock} Erfaringer fra de første VSME-rapporter viser, at mange virksomheder rapporterer ud over basismodullets krav, hvilket indikerer både ambition og behov for klar prioritering.\parencite{EY2025VSME} Det peger på, at VSME ofte bruges som minimumsramme snarere end som endemål for rapporteringen.

Samlet peger udviklingen på et behov for forenklede processer, klare minimumskrav og teknisk støtte, så SMV'er kan levere sporbar ESG-dokumentation uden at belaste kerneopgaven.

Behovet for teknisk og organisatorisk støtte gør ESG-as-a-Service relevant som servicekoncept i en reguleret sundhedssektor.
\subsection{ESG-as-a-Service som servicekoncept}
\label{subsec:esg-as-a-service-som-servicekoncept}

ESG-as-a-Service betegner en kombineret software- og serviceleverance, der omsætter regulatoriske krav til operationelle datapunkter, kontroller og rapportoutput. Konceptet adskiller sig fra klassisk SaaS (Software as a Service) ved at inkludere faglig sparring, konfigurerede standarder og løbende datakvalitetssikring, men adskiller sig også fra traditionel konsulentbistand ved at bygge på en fast digital infrastruktur. Verdantix dokumenterer et voksende marked for ESG-software, hvilket understreger, at rapportering i stigende grad institutionaliseres som en digital proces.\parencite{Verdantix2025Software}

Værdiforslaget kan forankres i stakeholder- og shared value-perspektiver, hvor dokumenteret ESG-indsats er en forudsætning for legitimitet og langsigtet værdiskabelse.\parencite{Freeman1984Stakeholder,PorterKramer2011CSV,Elkington1998TripleBottomLine} Samtidig eksisterer en modposition, hvor virksomhedens primære ansvar er over for aktionærerne, hvilket understreger behovet for at gøre compliance og økonomisk relevans eksplicit i servicekonceptet.\parencite{Friedman1970Profits}

I en reguleret sundhedssektor fungerer ESG-as-a-Service som et organisatorisk mellemled, der kan standardisere dataindsamling, reducere transaktionsomkostninger og skabe et auditspor, som gør rapportering beslutningsrelevant for ledelse, revisor og myndigheder. IWA 48 fremhæver, at ESG-implementering kræver standardiserede KPI'er, datakvalitet og klare rapporteringsprincipper, hvilket styrker behovet for en struktureret infrastruktur.\parencite{ISO2024IWA48} Sektoren er et relevant startmarked, fordi compliance-krav, datakompleksitet og forsyningskædepres gør behovet for struktureret ESG-rapportering særligt tydeligt.

Konceptet fungerer som et praktisk svar på kløften mellem krav og datapraksis og motiverer fokus på standardisering, governance og værdiskabelse som analytiske perspektiver.
