\section{Kontekst og rammer}
\label{sec:kontekst-og-rammer}

Den sektorielle og regulatoriske ramme former ESG-rapportering i sundhedssektoren og gør det tydeligt, hvor krav, data og organisatoriske kapaciteter ikke helt passer sammen. Afsnittet skitserer sektorens ESG-profil, de regulatoriske krav, SMV'ers ressourcevilkår og ESG-as-a-Service som svar på konkrete barrierer. Det giver et empirisk og regulatorisk fundament for at forstå kløften mellem krav og praksis.
\subsection{ESG-rapportering i sundhedssektoren}
\label{subsec:esg-rapportering-i-sundhedssektoren}

ESG-rapportering i sundhedssektoren er en styringsopgave med høj kompleksitet, fordi data og ansvar er fordelt på drift, forsyningskæder og arbejdsmiljø. \textcite{Sepetis2024Healthcare} viser, at sektoren må koble bæredygtighedsarbejde med digital transformation og klare processer for at kunne dokumentere resultater og ansvar, mens \textcite{Bosco2024ESGHealth} underbygger behovet gennem sektorspecifikke proceskort. \textcite{Vegro2025OneHealth} peger på, at ESG-rammer overlapper med bredere sundheds- og miljøhensyn, hvilket gør governance til en tværgående data- og procesdisciplin snarere end en ren compliance-opgave.

\textcite{HCWH2019ClimateFootprint}, \textcite{WHO2024HealthCareWaste} og \textcite{WHO2021HealthWorkerDeaths} dokumenterer, at sektorens klimaaftryk er domineret af scope~3, at affaldsprofilen kræver fraktionsspecifik dokumentation, og at arbejdsmiljørisici er empirisk veldokumenterede. Det betyder, at rapporteringen skal kunne samle leverandør-, affalds- og arbejdsmiljødata i konsistente formater, også for SMV'er med begrænset datakapacitet.

De centrale nøgletal opsummeres i tabel \ref{tab:esg-sundhedssektor-noegletal} og viser, hvorfor ESG-rapportering i sektoren kræver data på tværs af energi, forsyningskæde, affald og arbejdsmiljø.
\begin{table}[htbp]
  \caption{Empiriske nøgletal, der illustrerer hvorfor ESG-rapportering kræver data på tværs af drift og forsyningskæde.}
  \label{tab:esg-sundhedssektor-noegletal}
  \begingroup
  \renewcommand{\arraystretch}{1.2}
  \begin{tabularx}{\textwidth}{>{\raggedright\arraybackslash}p{0.20\textwidth} >{\raggedright\arraybackslash}p{0.30\textwidth} >{\raggedright\arraybackslash}X}
    \toprule
    Tema & Nøgletal & Betydning for rapportering \\
    \midrule
    Klimaaftryk & Ca. 4,4~\% af globale netto-udledninger (omtrent 2~Gt CO$_2$e). & Peger på behov for opgørelse af emissioner og reduktionsplaner på tværs af drift og forsyningskæde. \\
    Energi og scope & Over halvdelen af aftrykket kommer fra energi; ca. 17~\% scope~1, 12~\% scope~2 og 71~\% scope~3. & Understreger behovet for energidata og systematisk indsamling af leverandør- og indkøbsdata. \\
    Affald & Ca. 85~\% ikke-farligt affald og 15~\% farligt affald. & Kræver sporbarhed for affaldsstrømme og dokumentation for behandling og compliance. \\
    Arbejdsvilkår & Mindst 115.500 dødsfald blandt sundheds- og omsorgspersonale under COVID-19 (18 måneder). & Viser behovet for robuste sociale indikatorer om arbejdsmiljø og sikkerhed. \\
    \bottomrule
  \end{tabularx}
  \endgroup
  \TableSource{\parencite{HCWH2019ClimateFootprint,WHO2024HealthCareWaste,WHO2021HealthWorkerDeaths}}
\end{table}

Samlet peger empiri og litteratur på, at ESG-rapportering i sundhedssektoren må dække både direkte driftstal og indirekte forhold i forsyningskæden samt sociale og organisatoriske risici. Det skaber et behov for standardiserede processer, der kan omsætte kvalitative krav til sammenlignelige data og sporbar dokumentation, og gør de regulatoriske rammer til en central drivkraft.
\subsection{Regulatoriske rammer: CSRD, EU-taksonomien og GRI}
\label{subsec:regulatoriske-rammer-csrd-eu-taksonomien-og-gri}

De regulatoriske rammer for ESG-rapportering i EU består af flere sammenhængende elementer. \textcite{EU2022CSRD} udvider rapporteringspligten og gør bæredygtighedsrapportering til en integreret del af virksomhedernes officielle rapportering. Direktivet fastsætter, at rapporteringen skal ske efter ESRS, som er vedtaget som delegerede standarder, og at oplysningerne leveres i et standardiseret, digitalt format med XBRL-tagging, jf. \textcite{EU2023ESRS}. Implementeringen er trinvist indfaset og politisk justeret gennem den såkaldte stop-the-clock-aftale, jf. \textcite{Erhvervsstyrelsen2025StopClock}.

Indfasningen og de politiske justeringer betyder, at virksomheder skal planlægge rapportering under bevægelige rammer, hvor tidsplaner og præciseringer ændres over tid. \textcite{Erhvervsstyrelsen2025StopClock} peger på, at det gør fortolkning og prioritering til en reel del af compliance-arbejdet, særligt for SMV'er med begrænset kapacitet.

\textcite{EU2023ESRS} konkretiserer, hvilke oplysninger virksomhederne skal levere, og giver struktur til sammenhængen mellem strategi, risici, mål og resultater. Standarderne operationaliserer kravet om dobbelt væsentlighed og etablerer en fælles logik for datagrundlag og rapportering. \textcite{IFRS2023S1} fastlægger samtidig en investorrettet global baseline uden dobbelt væsentlighed, hvilket øger spændet mellem globale standarder og EU's krav og skærper kompleksiteten for SMV'er.

Dobbelt væsentlighed betyder, at virksomheder skal rapportere både deres påvirkning af mennesker og miljø og hvordan bæredygtighedsforhold påvirker virksomheden finansielt.\parencite{EU2023ESRS} Det udvider datakravet til værdikæde, governance og sociale forhold og gør materialitetsprocessen til en central del af rapporteringen.

EU-taksonomien supplerer rapporteringen ved at definere, hvilke økonomiske aktiviteter der kan betragtes som miljømæssigt bæredygtige.\parencite{EU2020Taxonomy} Rammeværket kræver, at virksomheder kan dokumentere overensstemmelse med taksonomien og derved knytte finansielle aktiviteter til konkrete miljømæssige mål.

GRI er et globalt, frivilligt rammeværk, som anvendes bredt til sammenlignelig ESG-rapportering, og standarden beskriver en struktureret tilgang til indikatorer og narrativer, der ofte bruges som supplement til regulatoriske krav.\parencite{GRI2021} \textcite{OECD2020ESG} peger på, at ESG-data stadig er fragmenterede og vanskelige at sammenligne, hvilket gør frivillige rammer utilstrækkelige uden en stærk datainfrastruktur.

ISO's IWA 48 samler principper for ESG-implementering og lægger vægt på standardiserede KPI'er, datakvalitet og rapporteringsprincipper.\parencite{ISO2024IWA48} Aftalen er ikke bindende i EU-retlig forstand, men fungerer som et globalt referencepunkt for, hvordan organisationer kan operationalisere ESG i praksis. Implikationen er, at der opstår en forventning om konsistente målinger, sporbarhed og systematisk dokumentation på tværs af sektorer, hvilket styrker behovet for robuste dataflows og governance. For SMV'er betyder det, at selv frivillige rammer kan drive professionalisering af ESG-data og gøre implementering til en løbende kapacitetsopgave.

\subsubsection{EU-retligt overblik og retskildernes funktion}
\label{subsec:eu-retligt-overblik}

ESG-regimet fungerer som et EU-retligt flerniveau-system, hvor retskilderne har forskellige funktioner og bindende karakter. Det gør det muligt at forstå, hvorfor rapporteringen både kræver juridisk pligt, tekniske standarder og klassifikationslogik:
\begin{enumerate}
  \item \textbf{Direktiv (CSRD).} Etablerer rapporteringspligt og overordnede krav, som skal implementeres i national ret.\parencite{EU2022CSRD}
  \item \textbf{Delegerede standarder (ESRS).} Konkretiserer indhold, datapunkter og struktur, så rapportering bliver sammenlignelig og auditerbar.\parencite{EU2023ESRS}
  \item \textbf{Forordning (EU-taksonomien).} Fastlægger en klassifikation af bæredygtige aktiviteter og kobler rapportering til miljømål og finansielle nøgletal.\parencite{EU2020Taxonomy}
  \item \textbf{Frivillige standarder (GRI).} Udgør et udbredt supplement, især hvor globale interessenter efterspørger sammenlignelige ESG-oplysninger.\parencite{GRI2021}
\end{enumerate}

Det EU-retlige hierarki fremgår af tabel \ref{tab:eu-retligt-hierarki}.
\begin{table}[htbp]
  \caption{EU-retligt hierarki for ESG-rapportering og dets funktionelle rolle i styringskæden.}
  \label{tab:eu-retligt-hierarki}
  \begingroup
  \renewcommand{\arraystretch}{1.2}
  \begin{tabularx}{\textwidth}{>{\raggedright\arraybackslash}p{0.22\textwidth} >{\raggedright\arraybackslash}p{0.32\textwidth} >{\raggedright\arraybackslash}X}
    \toprule
    Niveau & Retskilde og bindende karakter & Funktion i ESG-regimet \\
    \midrule
    Direktiv & CSRD; bindende mål, implementeres i national ret. & Etablerer rapporteringspligt, indfasning og krav til digital rapportering. \\
    Delegerede standarder & ESRS; bindende teknisk konkretisering. & Definerer datapunkter, struktur og auditbarhed. \\
    Forordning & EU-taksonomien; direkte gældende. & Klassificerer aktiviteter og forbinder rapportering med miljømål og nøgletal. \\
    Frivillige standarder & GRI; ikke-bindende, globalt udbredt. & Supplerer sammenlignelighed, især i globale værdikæder. \\
    \bottomrule
  \end{tabularx}
  \endgroup
\TableSource{\parencite{EU2022CSRD,EU2023ESRS,EU2020Taxonomy,GRI2021}}
\end{table}

\begin{figure}[htbp]
  \centering
  \begin{minipage}{0.9\textwidth}
    \begin{mdframed}[linecolor=KUrod, linewidth=1pt, backgroundcolor=black!3]
    \textbf{EU's styringslogik for ESG - fra direktiv til datapunkt}\\
    \small
    Direktivniveau (CSRD): pligt og ansvar for rapportering.\\
    Standardniveau (ESRS): datapunkter, auditspor og digital tagging.\\
    Forordningsniveau (taksonomi): klassificerer aktiviteter og økonomisk relevans.
    \end{mdframed}
  \end{minipage}
  \caption{EU's styringslogik som arkitektur for ESG-rapporteringen.}
  \label{fig:eu-styringslogik}
\end{figure}

\begin{displayquote}
ESG does not currently benefit from a universally accepted common set of standards.
\quoteattrib{\textcite{ESGBook}}
\end{displayquote}

Reguleringen er analytisk interessant: på den ene side øger standarderne sammenligneligheden, på den anden side efterlader de et fortolkningsrum, hvor rapportering kan glide over i minimum compliance frem for beslutningsrelevant styring.

Hierarkiet bruges som analytisk ramme i resten af rapporten. Direktivniveauet forklarer, hvorfor rapportering er uomgængelig og hvilke pligter der udløses; standardniveauet (ESRS) forklarer, hvordan data skal struktureres, valideres og gøres auditerbare; og forordningsniveauet (taksonomien) forklarer, hvilke aktiviteter der bliver økonomisk relevante. Dermed etableres en eksplicit kontrakt med læseren: senere analyser af datamodel, governance og værdiforslag må kunne spores tilbage til disse niveauer.

\subsubsection{Traceability-matrice som analytisk greb}
\label{subsec:traceability-matrice}

For at gøre kontrakten operationel anvendes en fast traceability-matrice, som binder retskilde til datapunkt, input, validering, beregning, output og audit-evidens. Matricen fungerer som et gennemgående analysegreb og genbruges i software- og evalueringsafsnittene, så hver central påstand kan spores tilbage til et regulatorisk krav og et konkret dataflow.

\begin{table}[htbp]
  \caption{Traceability-matrice for rapportens evidensscope (B2, C5, S1), der binder retskilde, datapunkt, datafelter og auditspor.}
  \label{tab:traceability-matrix}
  \begingroup
  \renewcommand{\arraystretch}{1.1}
  \small
  \begin{tabularx}{\textwidth}{>{\raggedright\arraybackslash}p{0.17\textwidth} >{\raggedright\arraybackslash}p{0.12\textwidth} >{\raggedright\arraybackslash}p{0.13\textwidth} >{\raggedright\arraybackslash}p{0.11\textwidth} >{\raggedright\arraybackslash}p{0.12\textwidth} >{\raggedright\arraybackslash}p{0.12\textwidth} >{\raggedright\arraybackslash}X}
    \toprule
    Retskilde + reference & ESRS-datapunkt & Inputfelter & Valideringsregel & Beregningsregel & Outputfelt & Audit-evidens \\
    \midrule
    CSRD \lawart{19a}, \lawart{29a} (rapportering og digital tagging) & ESRS E1-5/E1-6 (energi og GHG) & El/varme (kWh), brændsel (L/kg), emissionsfaktorer & Enheder, periodelås, kilde angivet & CO$_2$e = aktivitet $\times$ EF; energisum pr. periode & Scope 1/2 CO$_2$e; energiforbrug pr. periode & Kilde-ID, EF-version, ændringslog, godkendelse af dataejer \\
    CSRD \lawart{19a} (bæredygtighedsoplysninger) & ESRS E5 (affald og ressourcebrug) & Affald pr. fraktion (kg/ton), behandlingsform & Ikke-negative mængder, fraktionsmatch & Summering pr. fraktion; evt. CO$_2$e = m $\times$ EF & Affald pr. fraktion; andel farligt affald & Vejesedler, leverandørdokumentation, godkendt af drift \\
    CSRD \lawart{19a} (sociale forhold) & ESRS S1 (arbejdsmiljø) & Antal hændelser, arbejdstimer, medarbejderantal & Konsistens med HR-data, periodelås & Frekvens = hændelser/arbejdstimer & Arbejdsmiljø-KPI'er & HR-log, hændelsesjournal, sign-off af HR \\
    \bottomrule
  \end{tabularx}
  \endgroup
  \TableSource{\parencite{EU2022CSRD,EU2023ESRS,GHGProtocol2004}}
  \TableNote{Matricen er fuldt udfyldt for de moduler, der evalueres i rapporten; øvrige moduler er afgrænset i tabel \ref{tab:traceability-coverage}.}
\end{table}

\subsubsection{Dækningsafgrænsning for MVP-scope}
\label{subsec:daekning-traceability}

MVP'ens modulstruktur er bred, men rapportens evidensscope er afgrænset til de moduler, der faktisk demonstreres og evalueres (B2, C5 og S1). Det betyder, at traceability-matricen er fuldt udfyldt for disse moduler, mens øvrige moduler alene er specificeret på struktur- og designniveau (bilag \ref{app:teknisk-dokumentation} og \ref{tab:app-moduloversigt}). For kravene i tabel \ref{tab:krav-prioritering} er 6 ud af 8 krav operationaliseret i MVP'en (alle markeret som MVP-krav).

\begin{table}[htbp]
  \caption{Dækningsafgrænsning, der viser hvilke moduler der er demonstreret og hvilke der kun er specificeret.}
  \label{tab:traceability-coverage}
  \begingroup
  \renewcommand{\arraystretch}{1.2}
  \begin{tabularx}{\textwidth}{l X l}
    \toprule
    Modul/område & Datatype og output & Status i rapporten \\
    \midrule
    B2 (energi) & Energiforbrug og Scope 2 CO$_2$e. & Demonstreret og evalueret. \\
    C5 (affald) & Affald pr. fraktion og evt. CO$_2$e. & Demonstreret og evalueret. \\
    S1 (arbejdsmiljø) & Arbejdsmiljø-KPI'er. & Demonstreret og evalueret. \\
    Øvrige moduler i moduloversigten & Udledninger, aktiviteter og sociale/governance-data uden evaluering. & Specificeret, men ikke demonstreret. \\
    \bottomrule
  \end{tabularx}
  \endgroup
  \TableSource{Egen fremstilling baseret på bilag \ref{app:teknisk-dokumentation} og tabel \ref{tab:app-moduloversigt}.}
\end{table}

\subsubsection{Gennemgående regulatoriske principper}
\label{subsec:gennemgaaende-regulatoriske-principper}

På tværs af retskilderne går en række gentagende principper, der forklarer, hvorfor ESG-rapportering kræver systematik. Proportionalitet og faseindføring skalerer og udskyder kravene, hvilket skaber et gradvist men vedvarende implementeringspres.\parencite{EU2022CSRD,Erhvervsstyrelsen2025StopClock} Dobbelt væsentlighed gør materialitetsprocessen til et kernekrav og udvider datagrundlaget til både påvirkning og finansiel risiko.\parencite{EU2023ESRS} Dokumentations- og sporbarhedskrav følger af standardernes struktur og taksonomiens klassifikationslogik og nødvendiggør auditspor og konsistente datakilder.\parencite{EU2023ESRS,EU2020Taxonomy} \textcite{Finanstilsynet2025ESGRisk} viser, at finansielt tilsyn skaber en risikobaseret forventning om ESG-data, der forstærker kravene uden for den direkte rapporteringspligt.

\subsubsection{Implementering i dansk ret og praksis}
\label{subsec:implementering-i-dansk-ret}

Som direktiv forudsætter CSRD national implementering. I Danmark sker implementeringen via lov nr 480 af 22/05/2024, som ændrer årsregnskabsloven og relaterede love, og som operationaliseres gennem Erhvervsstyrelsens vejledninger og værktøjer til SMV'er.\parencite{EU2022CSRD,Lov2024CSRD,Aarsregnskabsloven2022,Erhvervsstyrelsen2025ESGTemplate,Virksomhedsguiden2025VSMEIntro} I praksis mødes virksomhederne ofte gennem nationale kanaler som myndigheder, revisorer og finansielle institutioner; \textcite{Finanstilsynet2025ESGRisk} og \textcite{GrantThorntonDK2025BankESG} viser, at bankernes ESG-risikostyring gør data til en forudsætning for kredit, også når rapportering formelt er frivillig.\parencite{DanskErhvervFSR2025CSRDTimeline}

Den juridiske struktur forstærker fokus på at reducere kløften mellem krav og datapraksis: ESG-as-a-Service er ikke blot et teknisk valg, men et regulatorisk mellemled i et flerniveau-regime. Det perspektiv danner overgang til SMV'ernes behov for forenkling.

Dansk implementering bliver mest håndgribelig, når den omsættes til konkrete kontrolpunkter, som SMV'er faktisk møder. Tabel \ref{tab:danske-kontrolpunkter} viser, hvordan lovgivning, vejledninger og finansielle krav bliver til operative grænseflader, der skal dokumenteres i data og auditspor.

\begin{table}[htbp]
  \caption{Danske kontrolpunkter, der gør transpositionen operationel for SMV'er.}
  \label{tab:danske-kontrolpunkter}
  \begingroup
  \renewcommand{\arraystretch}{1.2}
  \begin{tabularx}{\textwidth}{>{\raggedright\arraybackslash}p{0.28\textwidth} >{\raggedright\arraybackslash}p{0.32\textwidth} >{\raggedright\arraybackslash}X}
    \toprule
    Styringsartefakt & Udløst dokumentationskrav & Kontrolgrænseflade i praksis \\
    \midrule
    Lov nr 480/2024 og årsregnskabsloven & Bæredygtighedsoplysninger i ledelsesberetning samt dokumenteret datagrundlag for revision. & SMV'er møder kravet i årsrapportprocessen via revisor; datakilder og beregninger skal kunne genskabes. \\
    Erhvervsstyrelsens VSME-skabelon & Standardiserede datapunkter (basis/udvidet modul) til ensartet rapportering. & SMV'er bruger skabelonen i dialog med kunder, banker og myndigheder som minimums-datasæt. \\
    Finanstilsynets ESG-risikostyring og bankkrav & ESG-data som input til kredit- og risikovurdering (udledninger, risici, politikker). & SMV'er møder kravet i kreditprocesser; data skal være sporbare og validerede. \\
    \bottomrule
  \end{tabularx}
  \endgroup
  \TableSource{\parencite{Lov2024CSRD,Aarsregnskabsloven2022,Erhvervsstyrelsen2025ESGTemplate,Finanstilsynet2025ESGRisk,GrantThorntonDK2025BankESG}}
\end{table}

De centrale forskelle mellem rammerne og deres implikationer for datakrav og rapporteringslogik fremgår af tabel \ref{tab:rammer-sammenligning}.
\begin{table}[htbp]
  \caption{Sammenligning, der viser hvordan retskilderne udfylder forskellige styringsfunktioner i ESG-rapportering.}
  \label{tab:rammer-sammenligning}
  \begingroup
  \renewcommand{\arraystretch}{1.2}
  \begin{tabularx}{\textwidth}{>{\raggedright\arraybackslash}p{0.24\textwidth} >{\raggedright\arraybackslash}p{0.30\textwidth} >{\raggedright\arraybackslash}X}
    \toprule
    Rammeværk & Status og formål & Implikation for rapportering \\
    \midrule
    CSRD (direktiv) & Obligatorisk ramme for rapporteringspligt og indfasning. & Forankrer rapportering i års- og koncernrapport med krav om digital tagging. \\
    ESRS (delegerede standarder) & Tekniske standarder for datapunkter og struktur. & Operationaliserer dobbelt væsentlighed og sikrer sammenlignelighed. \\
    EU-taksonomien & Obligatorisk supplement; klassificerer miljømæssigt bæredygtige aktiviteter. & Kræver dokumentation for aktiviteters bidrag og overensstemmelse med minimumsgarantier. \\
    GRI & Frivilligt rammeværk, anvendes globalt. & Indikatorbaseret rapportering med fokus på væsentlige forhold og sammenlignelighed. \\
    \bottomrule
  \end{tabularx}
  \endgroup
  \TableSource{\parencite{EU2022CSRD,EU2023ESRS,EU2020Taxonomy,GRI2021}}
\end{table}

\subsubsection{Lovhenvisninger og paragrafformater (eksempler)}
\label{subsec:lovhenvisninger-og-paragrafformater}

For at gøre reguleringen operationel i rapporteringen kan centrale retskilder bindes til konkrete artikelhenvisninger. EU-lovgivning angives typisk med artikelnumre, mens danske love bruger paragraftegn. Udvalgte eksempler på formater og relevans fremgår af tabel \ref{tab:lovhenvisninger-eksempler}.
\begin{table}[htbp]
  \caption{Eksempler på lovhenvisninger, der gør juridiske krav operationelle i rapporteringen.}
  \label{tab:lovhenvisninger-eksempler}
  \begingroup
  \renewcommand{\arraystretch}{1.2}
  \begin{tabularx}{\textwidth}{>{\raggedright\arraybackslash}p{0.30\textwidth} >{\raggedright\arraybackslash}p{0.22\textwidth} >{\raggedright\arraybackslash}X}
    \toprule
    Retskilde & Henvisning & Relevans for ESG-rapportering \\
    \midrule
    CSRD (Dir. EU 2022/2464) & \lawart{19a}, \lawart{29a} & Krav til bæredygtighedsrapportering i års- og koncernrapport. \\
    EU-taksonomien (Reg. EU 2020/852) & \lawart{8} & Oplysning om taksonomiforenelige aktiviteter og nøgletal. \\
    GDPR (Reg. EU 2016/679) & \lawart{5} & Principper for dataminimering og lovlig behandling af persondata i ESG-data. \\
    \bottomrule
  \end{tabularx}
  \endgroup
  \TableSource{\parencite{EU2022CSRD,EU2020Taxonomy,EU2016GDPR}}
  \TableNote{Paragraffer i dansk lovgivning angives typisk med \lawpar{...}.}
\end{table}

For at gøre transpositionen juridisk operationel i systemdesignet kan kravene bindes til konkrete datapunkter, validering og auditfelter. Tabel \ref{tab:juridisk-til-data} viser et mini-katalog for rapportens MVP-scope, hvor den transitive kæde fra retskilde til output gøres eksplicit.

\begin{table}[htbp]
  \caption{Mini-katalog, der binder retskilde til datapunkt, validering og auditspor i MVP-scope.}
  \label{tab:juridisk-til-data}
  \begingroup
  \renewcommand{\arraystretch}{1.2}
  \begin{tabularx}{\textwidth}{>{\raggedright\arraybackslash}p{0.26\textwidth} >{\raggedright\arraybackslash}p{0.18\textwidth} >{\raggedright\arraybackslash}p{0.16\textwidth} >{\raggedright\arraybackslash}p{0.16\textwidth} >{\raggedright\arraybackslash}X}
    \toprule
    Retskilde (EU + DK) & Datapunkt & Valideringsregel & Auditfelt & Outputfelt \\
    \midrule
    CSRD \lawart{19a}/\lawart{29a} transponeret via lov nr 480/2024 & ESRS E1 (energi/Scope 2) & Enheder og periodelås & Kilde-ID, EF-version, sign-off & Scope 2 CO$_2$e pr. periode \\
    CSRD \lawart{19a} transponeret via lov nr 480/2024 & ESRS E5 (affald) & Fraktionsmatch og ikke-negative mængder & Vejeseddel, ændringslog & Affald pr. fraktion \\
    CSRD \lawart{19a} transponeret via lov nr 480/2024 & ESRS S1 (arbejdsmiljø) & Konsistens med HR-data & HR-log, ansvarlig godkendelse & Arbejdsmiljø-KPI'er \\
    \bottomrule
  \end{tabularx}
  \endgroup
  \TableSource{\parencite{EU2022CSRD,Lov2024CSRD,Aarsregnskabsloven2022,EU2023ESRS}}
\end{table}

Samlet peger rammerne på et flerniveau-regime med høje krav til data, sporbarhed og fortolkning. Det gør SMV'ers kapacitet og behov for forenkling afgørende.
\subsection{SMV'er og behovet for forenkling}
\label{subsec:smver-og-behovet-for-forenkling}

\textcite{Virksomhedsguiden2025SMVDefinition} fastlægger, at SMV'er defineres efter EU's størrelsesgrænser og ofte ligger uden for den direkte CSRD-pligt, men påvirkes indirekte gennem kundekrav, banker og leverandørrelationer. \textcite{PwC2025GuideESGSMV} og \textcite{GrantThorntonDK2025Omnibus} peger på, at dette gør frivillighed til en markedsforventning snarere end et reelt valg.

\textcite{Virksomhedsguiden2025VSMEIntro} beskriver, at EU har udviklet en frivillig VSME-standard (Voluntary Sustainability Reporting Standard for SMEs) for at sikre ensartet dataudveksling og forhindre uforholdsmæssige datakrav fra større virksomheder. \textcite{Virksomhedsguiden2025VSMEModules} præciserer, at VSME er opbygget af et basismodul og et udvidet modul, hvor basismodullets 11 datapunkter kan anvendes som minimumsniveau. \textcite{Virksomhedsguiden2025NoDualMateriality} viser, at standarden ikke kræver en dobbelt væsentlighedsanalyse, hvilket reducerer kompleksitet og ressourceforbrug. \textcite{Erhvervsstyrelsen2025ESGTemplate} dokumenterer, at Erhvervsstyrelsen samtidig har udviklet en skabelon, der samler datapunkterne og understøtter ensartet rapportering.

Indfasningen af CSRD er trinvist implementeret, og den seneste stop-the-clock-aftale udskyder rapporteringskrav for flere virksomhedstyper, hvilket giver SMV'er mere tid, men også skaber usikkerhed om krav og timing, jf. \textcite{Erhvervsstyrelsen2025StopClock}. \textcite{EY2025VSME} viser, at mange virksomheder rapporterer ud over basismodullets krav, hvilket indikerer både ambition og behov for klar prioritering. Det peger på, at VSME ofte bruges som minimumsramme snarere end som endemål for rapporteringen.

Samlet peger udviklingen på et behov for forenklede processer, klare minimumskrav og teknisk støtte, så SMV'er kan levere sporbar ESG-dokumentation uden at belaste kerneopgaven.

Behovet for teknisk og organisatorisk støtte gør ESG-as-a-Service relevant som servicekoncept i en reguleret sundhedssektor.
\subsection{ESG-as-a-Service som servicekoncept}
\label{subsec:esg-as-a-service-som-servicekoncept}

ESG-as-a-Service betegner en kombineret software- og serviceleverance, der omsætter regulatoriske krav til operationelle datapunkter, kontroller og rapportoutput. Konceptet adskiller sig fra klassisk SaaS (Software as a Service) ved at inkludere faglig sparring, konfigurerede standarder og løbende datakvalitetssikring, men adskiller sig også fra traditionel konsulentbistand ved at bygge på en fast digital infrastruktur. \textcite{Verdantix2025Software} dokumenterer, at markedet for ESG-software er voksende, hvilket understreger, at rapportering i stigende grad institutionaliseres som en digital proces.

Værdiforslaget kan forankres i stakeholder- og shared value-perspektiver, hvor dokumenteret ESG-indsats er en forudsætning for legitimitet og langsigtet værdiskabelse. \textcite{Freeman1984Stakeholder}, \textcite{PorterKramer2011CSV} og \textcite{Elkington1998TripleBottomLine} viser, at værdiskabelse forbindes til flere interessenter og bredere performance. Samtidig eksisterer en modposition, hvor virksomhedens primære ansvar er over for aktionærerne, hvilket \textcite{Friedman1970Profits} understreger og gør compliance og økonomisk relevans eksplicit i servicekonceptet.

I en reguleret sundhedssektor fungerer ESG-as-a-Service som et organisatorisk mellemled, der kan standardisere dataindsamling, reducere transaktionsomkostninger og skabe et auditspor, som gør rapportering beslutningsrelevant for ledelse, revisor og myndigheder. \textcite{ISO2024IWA48} fremhæver, at ESG-implementering kræver standardiserede KPI'er, datakvalitet og klare rapporteringsprincipper, hvilket styrker behovet for en struktureret infrastruktur. Sektoren er et relevant startmarked, fordi compliance-krav, datakompleksitet og forsyningskædepres gør behovet for struktureret ESG-rapportering særligt tydeligt.

Konceptet fungerer som et praktisk svar på kløften mellem krav og datapraksis og motiverer fokus på standardisering, governance og værdiskabelse som analytiske perspektiver.
