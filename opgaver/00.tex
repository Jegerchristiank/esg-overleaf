\section*{Resumé}
\addcontentsline{toc}{section}{Resumé}

Undersøgelsen analyserer ESG-rapportering i sundhedssektoren med fokus på regulatoriske rammer, organisatorisk implementering og økonomisk relevans for små og mellemstore virksomheder (SMV'er). CSRD og ESRS flytter rapporteringen fra narrativ til data gennem krav om sporbarhed og digital tagging; rapporten vurderer derfor, hvordan ESG-as-a-Service og et Minimum Viable Product (MVP) kan omsætte kravene i en SMV-kontekst.\parencite{EU2022CSRD,EU2023ESRS}

Metoden kombinerer dokumentanalyse af CSRD, ESRS, EU-taksonomien og GRI med empiriske sektorkilder om klimaaftryk, affald og arbejdsvilkår. Sektorkilderne dokumenterer datatunge og fragmenterede påvirkninger; derfor indgår et casebaseret pilotstudie og en artefaktanalyse af MVP'ens dataflow for at koble krav, praksis og software.\parencite{HCWH2019ClimateFootprint,WHO2024HealthCareWaste,WHO2021HealthWorkerDeaths}

Analysen viser, at ESG fungerer som organisationsstandard, der skaber legitimitet og sammenlignelighed, men også en risiko for dekobling mellem rapportering og praksis.\parencite{BrunssonWorldOfStandards,Berg2022AggregateConfusion} ESG-as-a-Service kan reducere denne risiko ved at omsætte krav til standardiserede arbejdsgange, validering og auditspor, men værdien afhænger af fokus på materielle temaer og organisatorisk forankring.

Konklusionen peger på, at MVP'en fremstår praktisk gennemførbar for SMV'er, men at udbredelse kræver gradvis implementering, bedre integrationer og fortsat fokus på datakvalitet. Det peger på en trinvist opbygget ESG-rapportering, hvor minimumsmoduler etableres først og udvides i takt med kapacitet og modenhed.
