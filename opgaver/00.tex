\section*{Resumé}
\addcontentsline{toc}{section}{Resumé}

Undersøgelsen analyserer rapportering af miljø-, sociale og governanceforhold (ESG) i sundhedssektoren med fokus på regulatoriske rammer, organisatorisk implementering og økonomisk relevans for små og mellemstore virksomheder (SMV'er). I tråd med \textcite{EU2022CSRD} og \textcite{EU2023ESRS} stiller rapporteringen krav til dataindsamling, sporbarhed og rapportoutput; formålet er at vurdere, hvordan ESG-as-a-Service og et Minimum Viable Product (MVP) kan omsætte disse krav i en SMV-kontekst.

Metoden kombinerer dokumentanalyse af CSRD, ESRS, EU-taksonomien og GRI med empiriske sektorkilder om klimaaftryk, affald og arbejdsvilkår. Ifølge \textcite{HCWH2019ClimateFootprint} og \textcite{WHO2024HealthCareWaste} er sektorens påvirkninger datatunge og fragmenterede; derfor indgår et casebaseret pilotstudie af egen virksomhed og en artefaktanalyse af MVP'ens dataflow og funktionalitet. Det giver et samlet grundlag for at koble krav, praksis og software.

Analysen peger på, at ESG fungerer som en organisationsstandard, der skaber legitimitet og sammenlignelighed, men samtidig risiko for dekobling mellem rapportering og praksis. Som det formuleres hos \textcite{BrunssonWorldOfStandards}, kræver standarder operationalisering gennem arbejdsgange og kontrol; ESG-as-a-Service kan reducere denne risiko ved at tilbyde standardiserede arbejdsgange, validering og auditspor. Værdiforslaget afhænger dog af, at indsatsen rettes mod materielle temaer og at data forankres organisatorisk.

Konklusionen peger på, at MVP'en fremstår praktisk gennemførbar for SMV'er, men at udbredelse kræver gradvis implementering, bedre integrationer og fortsat fokus på datakvalitet. Dermed peges der på en trinvist opbygget ESG-rapportering, hvor minimumsmoduler etableres først og udvides i takt med kapacitet og modenhed.
