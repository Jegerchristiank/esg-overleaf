\section{Software og MVP}
\label{sec:software-og-mvp}

MVP'en omsætter ESG-krav til konkrete data- og rapporteringsprocesser i en SMV-kontekst. Fokus er på den organisatoriske \textit{oversættelse} fra krav til praksis: hvilke data der efterspørges, hvordan de valideres, og hvordan rapportoutput gøres \textit{sporbart}. Case- og teknisk dokumentation understøtter sporbarhed og efterprøvning (bilag \ref{app:teknisk-dokumentation}).

MVP'en operationaliserer forskningsspørgsmål~3 ved at konkretisere, hvordan krav bliver til data, kontroller og rapportoutput. Det giver en direkte forbindelse mellem regulative krav og softwarelogik.
Afsnittet er organiseret i krav, arkitektur, datamodel, dataindsamling og rapportoutput, så koblingen mellem jura og datapunkter kan følges trin for trin.
\textit{Traceability-matricen} fra afsnit \ref{subsec:traceability-matrice} fungerer som reference for, hvilke felter der skal indsamles, valideres og logges i \textit{auditspor}; fuld matrix fremgår af bilagstabel \ref{tab:traceability-matrix-full}.
\subsection{Formål og krav}
\label{subsec:formaal-og-krav}

Formålet med MVP'en er at reducere rapporteringsbyrden for SMV'er ved at samle ESG-data i en ensartet proces og skabe et \textit{auditabelt output}. Løsningen skal gøre det muligt at indsamle kernedata, validere dem og generere dokumentation, som kan anvendes over for ledelse, revisorer og myndigheder. Kravene operationaliseres i case- og softwarematerialet i bilag \ref{app:teknisk-dokumentation}.

Kravene dækker både funktionelle og ikke-funktionelle behov og prioriteres efter, hvad der er nødvendigt i MVP'en frem for en fuld løsning. Prioriteringen, der afgrænser MVP'ens scope, fremgår af tabel \ref{tab:krav-prioritering}.
\begin{table}[h]
  \caption{Krav og prioritering, der afgrænser hvad MVP'en skal kunne dokumentere.}
  \label{tab:krav-prioritering}
  \rowcolors{2}{black!5}{white}
  \begin{tabularx}{\textwidth}{X l X}
    \toprule
    Krav & Prioritet & Begrundelse \\
    \midrule
    Modulbaseret dataindsamling & MVP & Nødvendig for at strukturere ESG-data efter scope og tema. \\
    Validering af input & MVP & Sikrer datakvalitet og konsistens på tværs af moduler. \\
    Beregning af indikatorer & MVP & Omsætter input til målbare resultater og beslutningsgrundlag. \\
    Sporbarhed og ændringshistorik & MVP & Understøtter efterprøvbarhed og audit. \\
    Rapporteksport (PDF og standardiserede formater) & MVP & Gør output anvendeligt for ledelse og compliance. \\
    Integration til eksisterende datakilder & Fuld & Reducerer manuelt arbejde og øger skalerbarhed. \\
    Benchmark og proaktive alarmer & Fuld & Tilføjer strategisk indsigt ud over compliance. \\
    Rollebaseret adgang og sikkerhed & MVP & Krævet for GDPR-kompatibel databehandling. \\
    \bottomrule
  \end{tabularx}
  \TableSource{Egen fremstilling baseret på case- og softwaremateriale (bilag \ref{app:teknisk-dokumentation}).}
\end{table}

\subsubsection{Konsekvens af fravalg}
\label{subsec:konsekvens-fravalg}

Fravalget af iXBRL/XBRL i MVP'en er et bevidst designvalg for at holde scope på dataindsamling, validering og beregningsspor. Konsekvensen er, at output ikke kan indsendes som digitalt tagged compliance-format og derfor kræver manuel konvertering ved ekstern indberetning. For at sikre fremtidig kompatibilitet fastholdes en \textit{minimumskontrakt} i datamodellen: entydigt datapunkt-ID, ESRS-reference, periode, enhed, kilde-ID og beregningsregel-ID. En minimal teknisk pathway til senere implementering omfatter (1) mapping af datapunkter til XBRL-taxonomi, (2) namespace- og tag-regler for hver datapunktklasse og (3) validering mod taxonomiens konsistensregler før eksport. Dermed er fravalget en prioritering af kernefunktionalitet over indberetningsformat, men uden at ændre datamodellens krav til sporbarhed.

\subsection{Systemoversigt og arkitektur}
\label{subsec:systemoversigt-og-arkitektur}

MVP'en er en webbaseret løsning med en klar opdeling mellem brugergrænseflade, applikationslogik og datalagring. Løsningen håndterer brugerflow, dataindtastning, validering og lagring i et sammenhængende forløb. Teknisk dokumentation og software-specifikation fremgår af bilag \ref{app:teknisk-dokumentation} og bilag \ref{app:software-spec}. Den detaljerede software-specifikation vedlægges som separat fil.

Teknisk er MVP'en implementeret som en Next.js-baseret webapp med en Node.js-backend, der persisterer wizard-data i PostgreSQL. Beregninger og schema-validering ligger i en fælles komponent, mens rapportoutput genereres via en PDF-renderer. Backend udstiller et snapshot-endpoint, som understøtter hentning og lagring af state og gør auditspor og versionshistorik operationelle. Det understreger, at artefakten er et fungerende softwareprodukt og ikke kun en konceptuel skitse (bilag \ref{app:teknisk-dokumentation} og \ref{app:software-spec}).

\textbf{Dataflowet er kontrolleret: input indsamles, valideres og gemmes med historik, så ændringer kan spores og genskabes.} Arkitekturen understøtter sporbarhed og gør det muligt at udvide komponenter uden at bryde den samlede proceslogik. Det skaber en stabil ramme for audit og videreudvikling.

Arkitekturens hovedkomponenter og relationer fremgår af figur \ref{fig:system-arkitektur}.
\begin{figure}[h]
  \centering
  \fbox{\begin{tabular}{c}
    Brugergrænseflade \\
    $\downarrow$ \\
    Applikationslogik \\
    $\downarrow$ \\
    Datalagring \\
  \end{tabular}}
  \caption{Systemoversigt, der viser hvor data valideres og gøres sporbare.}
  \label{fig:system-arkitektur}
  \FigureSource{Egen fremstilling.}
\end{figure}

Sporbarhed operationaliseres i en datamodel og i valget af centrale ESG-indikatorer.
\subsection{Datamodel og ESG-indikatorer}
\label{subsec:datamodel-og-esg-indikatorer}

Datamodelafsnittet forklarer \textit{sporbarhedsprincippet} og angiver, hvilke indikatorer der operationaliseres i MVP'en.
Datamodellen er designet til at understøtte sporbarhed gennem \textit{versionshistorik} og \textit{auditspor}. Den organiserer data i tre logiske lag: profil og afgrænsning, indikatorregistreringer og ændringshistorik.

Datamodellens hovedkomponenter og deres funktion i MVP'en fremgår af tabel \ref{tab:datamodel-kernetable}.
\begin{table}[h]
  \caption{Datamodelkomponenter, der binder input til rapporterbare felter og auditspor.}
  \label{tab:datamodel-kernetable}
  \rowcolors{2}{black!5}{white}
  \begin{tabularx}{\textwidth}{l X X}
    \toprule
    Dataobjekt & Formål & Eksempler på indhold \\
    \midrule
    Profil og afgrænsning & Fastlægger organisationens scope og relevante moduler. & Grundoplysninger, perioder, afgrænsninger. \\
    Indikatorregistrering & Samler data pr. modul til beregning og rapportering. & Forbrugstal, mængder, enheder, datakilder. \\
    Ændrings- og beregningsspor & Dokumenterer ændringer og udledte resultater. & Tidsstempel, ændringstype, beregningsgrundlag. \\
    \bottomrule
  \end{tabularx}
  \TableSource{Egen fremstilling; datadefinitioner fremgår af bilag \ref{app:teknisk-dokumentation}.}
\end{table}

MVP'en fokuserer på indikatorer inden for energi, affald og sociale forhold, i tråd med casebeskrivelsen og sektorens ESG-profil. Eksempler på indikatorer og datakilder er flyttet til bilagstabel \ref{tab:esg-indikatorer}.

\subsubsection{Operationalisering af ESRS E1 til rapporteringsvariable}
\label{subsec:operationalisering-esrs-e1}

For at gøre oversættelsen fra standard til software eksplicit er udvalgte datapunkter fra ESRS E1 operationaliseret til konkrete input, beregningsregler og outputfelter i MVP'en. Den forenklede mapping af ESRS E1 til input, beregning og output fremgår af tabel \ref{tab:esrs-e1-operationalisering}.
\begin{table}[h]
  \caption{Operationalisering af ESRS E1, der viser sporbarhed fra standard til beregning.}
  \label{tab:esrs-e1-operationalisering}
  \rowcolors{2}{black!5}{white}
  \begin{tabularx}{\textwidth}{l X X X}
    \toprule
    ESRS E1-datapunkt & Input & Beregningsregel & Output i MVP \\
    \midrule
    Energiforbrug & El- og varmeforbrug (kWh) pr. periode. & $E = \sum kWh$ & Energiforbrug pr. modul/periode. \\
    Scope 2-udledninger & Energiforbrug og emissionsfaktor. & $\mathrm{CO_2e}_{\text{el}} = kWh \cdot EF_{\text{el}}$ & Scope 2 \ensuremath{\mathrm{CO_2e}} pr. periode. \\
    Scope 1-udledninger & Brændselsforbrug og emissionsfaktor. & $\mathrm{CO_2e}_{\text{br}} = m \cdot EF_{\text{br}}$ & Scope 1 \ensuremath{\mathrm{CO_2e}} pr. periode. \\
    Scope 3 (udvalgt kategori) & Affaldsmængder og faktor pr. fraktion. & $\mathrm{CO_2e}_{\text{aff}} = m \cdot EF_{\text{aff}}$ & Scope 3 \ensuremath{\mathrm{CO_2e}} pr. fraktion. \\
    \bottomrule
  \end{tabularx}
  \TableSource{\parenciteshort{EU2023ESRS,GHGProtocol2004}}
  \TableNote{Mappingen er illustrativ og viser principper for oversættelse fra standard til datapunkter.}
\end{table}

De centrale beregninger kan udtrykkes med følgende forenklede relationer, der gør auditlogikken synlig:
\begin{align*}
  \mathrm{CO_2e}_{\text{total}} &= \sum_i \left( A_i \cdot EF_i \right) \\
  \mathrm{CO_2e}_{\text{el}} &= E_{\text{el}} \cdot EF_{\text{el}} \\
  \mathrm{CO_2e}_{\text{varme}} &= E_{\text{varme}} \cdot EF_{\text{varme}}
\end{align*}
hvor $A_i$ er aktivitetsdata (fx \si{\kilo\watt\hour} eller \si{\kilogram}) og $EF_i$ er emissionsfaktorer (kg \ensuremath{\mathrm{CO_2e}} pr. enhed). \textcite{GHGProtocol2004} angiver, at emissionsfaktorer antages at være scope- og periodespecifikke og hentes fra standardtabeller.

\paragraph{Antagelser og usikkerhed i beregninger}
For at gøre auditbarhed substantiel er beregningerne bundet til et sæt governance-antagelser, som logges i auditsporet:
\begin{itemize}
  \item \textbf{Kildehierarki for emissionsfaktorer:} Nationale eller EU-officielle tabeller prioriteres, derefter leverandør- eller aktivitetsdata og til sidst generiske standardtabeller.\parencite{GHGProtocol2004}
  \item \textbf{Periodespecificitet:} Aktivitetsdata og emissionsfaktorer knyttes til et bestemt rapporteringsår og kan ikke blandes på tværs af perioder.
  \item \textbf{Afrunding og konvertering:} Enheder normaliseres (fx kWh til MWh), og afrundingsregler dokumenteres i beregningssporet.
  \item \textbf{Manglende data:} Manglende input markeres som estimat med begrundelse og synliggøres i output.
  \item \textbf{Sensitivitet:} Valg af emissionsfaktorer kan give variation i output; ændringer i faktorvalg logges, så effekten kan vurderes.
\end{itemize}

\paragraph{Illustrative beregningseksempler}
For at gøre input-output konkret viser tabel \ref{tab:beregningseksempler} to forenklede eksempler, der følger principperne i tabel \ref{tab:esrs-e1-operationalisering}. Tallene er illustrative og anvendes til at synliggøre auditspor og sporbarhed.
\begin{table}[h]
  \caption{Illustrative eksempler på input, beregning og output i MVP'en.}
  \label{tab:beregningseksempler}
  \rowcolors{2}{black!5}{white}
  \begin{tabularx}{\textwidth}{l X X X}
    \toprule
    Eksempel & Input & Beregning & Output \\
    \midrule
    B2 varme (Scope 2) & Varmeforbrug 120.000 kWh; genindvundet varme 20.000 kWh; EF 0,20 kg CO$_2$e/kWh; vedvarende andel 10\%. & Nettoforbrug = 100.000 kWh; brutto = 20.000 kg CO$_2$e; reduktion = 2.000 kg; netto = 18.000 kg. & 18,0 t CO$_2$e registreres i beregningsspor. \\
    C5 affald (Scope 3) & Affaldsmængde 500 kg; EF 1,2 kg CO$_2$e/kg. & 500 $\times$ 1,2 = 600 kg CO$_2$e. & 0,6 t CO$_2$e pr. fraktion. \\
    \bottomrule
  \end{tabularx}
  \TableSource{Egen fremstilling baseret på beregningsprincipperne i MVP'en.}
\end{table}

Indikatorerne er organiseret efter ESG-domæner, hvilket gør det muligt at mappe data til de overordnede strukturer i ESRS og GRI uden at påstå fuld dækning af alle datapunkter. I tråd med \textcitefull{EU2023ESRS} og \textcitefull{GRI2021} afspejler afgrænsningen standardernes modulære logik og peger på, at fuld dækning kræver yderligere moduler og datakilder.
En samlet moduloversigt med scopes og moduler fremgår af bilagene (tabel \ref{tab:app-moduloversigt}).

\subsection{Dataindsamling og automatisering}
\label{subsec:dataindsamling-og-automatisering}

Dataindsamlingen er organiseret som et guidet modulforløb, hvor brugeren indtaster eller uploader data pr. modul. Input valideres gennem faste regler, og data gemmes med \textit{versionshistorik}, så ændringer kan spores over tid. Detaljer om datadefinitioner og kontrolpunkter fremgår af bilag \ref{app:teknisk-dokumentation}.

Dataflowet fra datakilder til rapportoutput samt placeringen af validering og beregning fremgår af figur \ref{fig:dataflow-pipeline}.
\begin{figure}[h]
  \centering
  \fbox{\begin{tabular}{c}
    Datakilder (drift, HR, økonomi) \\
    $\downarrow$ Strukturering og manuel input \\
    Guidet indsamling og validering \\
    $\downarrow$ Beregningsmotor \\
    $\downarrow$ Auditspor og versionshistorik \\
    $\downarrow$ Rapportoutput (PDF og standardiserede formater) \\
  \end{tabular}}
  \caption{Forenklet dataflow, der viser hvor validering, beregning og auditspor skaber sporbarhed.}
  \label{fig:dataflow-pipeline}
  \FigureSource{Egen fremstilling.}
\end{figure}

Automatisering sker gennem løbende gemmefunktion og beregning af indikatorer. Det reducerer risikoen for datatab og skaber et konsistent grundlag for rapportoutput. \textit{Auditsporet} registrerer ændringer og versioner og gør det muligt at efterprøve resultater og antagelser.

Dataindsamling kan ske via API-integrationer eller CSV-udtræk fra drifts-, økonomi- og HR-data, men løsningen understøtter også manuel indtastning for at sikre, at rapporteringen kan gennemføres uden fulde integrationer. Kvalitetskontrol sker gennem valideringsregler, obligatoriske felter og konsistente enheder. \textcitefull{ISO2024IWA48} understreger, at ESG-implementering kræver standardiserede KPI'er og rapporteringsprincipper med validering og dokumentation, hvilket understøtter krav om sporbarhed og datakvalitet i MVP'en. Det betyder, at valideringsreglerne ikke blot er tekniske, men en direkte operationalisering af standardernes krav til konsistens.

For at synliggøre governance i datakvalitet er det nyttigt at knytte typiske fejltyper til konkrete kontrolpunkter. Sammenhængen mellem fejltyper og MVP-kontrolpunkter er flyttet til bilagstabel \ref{tab:datakvalitet-kontrol}.

\subsubsection{Rolle- og kontrolfordeling}
\label{subsec:rolle-kontrolfordeling}

ESG-as-a-Service forudsætter en tydelig rolle- og kontrolfordeling mellem software, serviceleverandør og virksomheden selv. Kontrollerne er derfor opdelt, så tekniske valideringer håndteres automatisk, mens dataejerskab og godkendelser er menneskeligt forankret og dokumenteret i auditsporet.

\begin{table}[h]
  \caption{Kontrolfordeling, der viser hvad software gør, hvad mennesker godkender, og hvordan det dokumenteres.}
  \label{tab:kontrolfordeling}
  \rowcolors{2}{black!5}{white}
  \begin{tabularx}{\textwidth}{l X X}
    \toprule
    Kontrolniveau & Ansvar & Dokumentation i MVP \\
    \midrule
    Softwarekontrol & Validering af enheder, periode, obligatoriske felter og beregninger. & Automatisk valideringslog, fejlmeddelelser og beregningsspor. \\
    Dataejer (SMV) & Godkendelse af indtastninger, kildeangivelse og endelig rapport. & Sign-off, kilde-ID og versionshistorik pr. datapunkt. \\
    Serviceleverandør & Fortolkning af standardkrav, opsætning af regelsæt og support ved afvigelser. & Konfigurationslog, ændringshistorik og dokumenterede beslutningsregler. \\
    \bottomrule
  \end{tabularx}
  \TableSource{Egen fremstilling.}
\end{table}

\subsection{Brugerflow og rapportoutput}
\label{subsec:brugerflow-og-rapportoutput}

Brugerflowet er designet til at guide en SMV gennem en struktureret indsamling af ESG-data. Profiloverblik, moduloverblik og profil-flowets progression er dokumenteret i bilag \ref{app:supplerende-figurer-og-tabeller} (figurer \ref{fig:app-landing-profil-overblik} til \ref{fig:app-profil-flow-progression}), så hovedteksten kan fokusere på proceslogik frem for dokumentation.

Brugerrejsen fra onboarding til eksport fremgår af figur \ref{fig:brugerflow-diagram} og illustrerer den overordnede proceslogik.
\begin{figure}[h]
  \centering
  \fbox{\begin{tabular}{c}
    Onboarding og profil \\
    $\downarrow$ \\
    Modulvalg \\
    $\downarrow$ \\
    Dataindsamling og validering \\
    $\downarrow$ \\
    Review og kvalitetstjek \\
    $\downarrow$ \\
    Eksport (PDF og standardiserede formater) \\
  \end{tabular}}
  \caption{Brugerflow, der synliggør overgangen fra dataindsamling til rapportoutput.}
  \label{fig:brugerflow-diagram}
  \FigureSource{Egen fremstilling.}
\end{figure}

Dataindtastningen foregår modul for modul. Eksempler på udfyldt input, beregnet CO$_2$-estimat, beregningstrace og review-side er samlet i bilag \ref{app:supplerende-figurer-og-tabeller} (figurer \ref{fig:app-b2-tomt} til \ref{fig:app-review-topoverblik}), mens hovedteksten fokuserer på krav, validering og auditspor.

I \textcitefull{EU2022CSRD} kræves, at rapportoutput genereres som PDF til ledelse og interessenter samt som strukturerede data til compliance i standardiserede formater (fx XBRL), hvilket gør output anvendeligt på tværs af beslutnings- og dokumentationssammenhænge. \textcite{Faccia2021XBRL} og \textcite{XBRLUS2022} fremhæver, at XBRL-tagging styrker maskinlæsbarhed og sammenlignelighed og er en forudsætning for automatiseret kontrol og audit. MVP'en genererer i denne version PDF-preview og et struktureret datasæt til intern brug, men implementerer ikke iXBRL/XBRL-tagging; det kræver taksonomi-mapping, namespace-håndtering og valideringsregler mod XBRL-taxonomien. Et eksempel på PDF-preview er gengivet i figur \ref{fig:app-reportoutput-preview} (bilag \ref{app:supplerende-figurer-og-tabeller}).

\subsection{Demonstration og evaluering}
\label{subsec:demonstration-og-evaluering}

Evalueringen tager udgangspunkt i kravene i tabel \ref{tab:krav-prioritering} og afprøver, om MVP'en leverer de forventede funktioner. Demonstrationen er kvalitativ og bygger på tre scenarier, der dækker energi, affald og sociale indikatorer (bilagstabel \ref{tab:testscenarier}). De fulde traceability-resultater rapporteres for to komplette testruns (B2 og C5), mens det sociale scenarie primært bruges som output-check. Evalueringen er en self-audit udført af forfatteren; for at reducere cirkularitet suppleres den med en negativ test og en reproducerbarhedstest baseret på auditspor.

\subsubsection{Uafhængighed og supplerende testtyper}
\label{subsec:uafhaengighed-testtyper}

Da evalueringen ikke er udført af en uafhængig part, dokumenteres testtyper, der kan falsificere output og efterprøve reproducerbarhed uden at kende de forventede resultater. Den negative test kontrollerer, at validering blokerer datapunkter med mangelfulde eller inkonsistente input, mens reproducerbarhedstesten tester, om output kan rekonstrueres ud fra auditspor alene.

\begin{table}[h]
  \caption{Supplerende testtyper, der dokumenterer fejldetektion og reproducerbarhed.}
  \label{tab:supplerende-tests}
  \rowcolors{2}{black!5}{white}
  \begin{tabularx}{\textwidth}{l X X}
    \toprule
    Testtype & Testdesign & Resultat i MVP \\
    \midrule
    Negativ test (fejldetektion) & Manglende kilde-ID og inkonsistent enhed i affaldsmodul. & Validering blokerer datapunkt; C$_{\text{trace}}$ falder til 60\% (3/5 felter) og eksport stoppes. \\
    Reproducerbarhedstest (self-audit) & Output rekonstrueres udelukkende fra auditspor, uden forventet output. & Output genskabes inden for afrundingsregel; auditspor gør beregningen verificerbar. \\
    \bottomrule
  \end{tabularx}
  \TableSource{Egen fremstilling baseret på MVP-logik.}
  \TableNote{Testene er udført som self-audit og bør replikkeres af en uafhængig part for fuld styrke.}
\end{table}

\subsubsection{Acceptkriterier for sporbarhed}
\label{subsec:acceptkriterier-sporbarhed}

Sporbarhed evalueres med udgangspunkt i traceability-matricen (eksempel i tabel \ref{tab:traceability-matrix}, fuld matrix i bilagstabel \ref{tab:traceability-matrix-full}) og et minimumssæt af auditfelter. Disse felter definerer, hvad der mindst skal logges for, at et datapunkt kan anses som auditabelt.

\begin{table}[h]
  \caption{Minimum auditfelter, der bruges som acceptkriterier i evalueringen.}
  \label{tab:auditfelter-minimum}
  \rowcolors{2}{black!5}{white}
  \begin{tabularx}{\textwidth}{l X}
    \toprule
    Auditfelt & Formål \\
    \midrule
    Tidsstempel og ændringstype & Dokumenterer hvornår data ændres, og hvad der ændres. \\
    Datakilde og kilde-ID & Gør det muligt at genskabe input og kildegrundlag. \\
    Valideringsresultat & Viser om input overholder enheder, periode og regler. \\
    Beregningsgrundlag & Angiver faktorer, version og formel for output. \\
    Ansvarlig godkendelse & Dokumenterer hvem der attesterer datapunktet. \\
    \bottomrule
  \end{tabularx}
  \TableSource{Egen fremstilling.}
\end{table}

Trace completeness beregnes som andelen af udfyldte obligatoriske auditfelter pr. datapunkt:
\begin{equation*}
  C_{\text{trace}} = \frac{F_{\text{udfyldt}}}{F_{\text{krav}}} \cdot 100\%
\end{equation*}
I evalueringen kræves som minimum, at $C_{\text{trace}} \geq 90\%$ for hvert scenarie.
Som supplement til udfyldningsgrad anvendes negativ test og reproducerbarhedstest (tabel \ref{tab:supplerende-tests}), så sporbarhed ikke kun måler komplethed, men også fejldetektion og efterprøvelighed.

Beslutningsreglen er eksplicit: hvis $C_{\text{trace}} < 90\%$ markeres datapunktet som \emph{fail}, og rapportoutput blokeres, indtil mangler er rettet; hvis $C_{\text{trace}} \geq 90\%$ markeres datapunktet som \emph{pass}, men med auditflag, hvis der er anvendt estimater eller manuelle korrektioner.

\begin{table}[h]
  \caption{Traceability-resultater for to fulde testruns baseret på minimums-auditfelter.}
  \label{tab:traceability-evaluering}
  \rowcolors{2}{black!5}{white}
  \begin{tabularx}{\textwidth}{l l r r l X}
    \toprule
    Scenarie & Matrix-ID & $n_{\text{total}}$ & $n_{\text{fuldfort}}$ & $C_{\text{trace}}$ & Udfyldte auditfelter \\
    \midrule
    Energi (B2) & M1 & 5 & 5 & 100\% & Tidsstempel, kilde-ID, valideringslog, EF-version, sign-off. \\
    Affald (C5) & M2 & 5 & 5 & 100\% & Tidsstempel, fraktionskilde, valideringslog, beregningsspor, sign-off. \\
    \bottomrule
  \end{tabularx}
  \TableSource{Egen fremstilling.}
  \TableNote{Vurderingen er baseret på de dokumenterede scenarier i bilag \ref{app:teknisk-dokumentation}.}
\end{table}

Resultaterne indikerer, at MVP'en opfylder de centrale MVP-krav om dataindsamling, validering, beregning og sporbarhed, og at fulde testruns kan dokumenteres for mindst to modultyper. Den negative test viser samtidig, at valideringslogikken kan blokere mangelfulde datapunkter, hvilket gør sporbarhed efterprøvelig frem for kun komplet. Dokumentation i bilag \ref{app:supplerende-figurer-og-tabeller} og \ref{app:teknisk-dokumentation} understøtter vurderingen.

\subsubsection{Indikativt tidsestimat}
\label{subsec:indikativt-tidsestimat}

For at gøre værdiforslaget mere håndgribeligt kan tidsforbrug estimeres pr. rapporteringscyklus for en mindre klinik. Et groft overslag over tidsforbrug fremgår af tabel \ref{tab:tidsforbrug-mvp}.
\begin{table}[h]
  \caption{Illustrativt tidsforbrug før og efter MVP, der synliggør forventet effektivisering.}
  \label{tab:tidsforbrug-mvp}
  \rowcolors{2}{black!5}{white}
  \begin{tabularx}{\textwidth}{X r r}
    \toprule
    Aktivitet & Manuel (timer) & MVP (timer) \\
    \midrule
    Dataindsamling (energi, affald, sociale KPI'er) & 24 & 12 \\
    Validering og fejlkontrol & 8 & 4 \\
    Konsolidering og rapporteksport & 10 & 4 \\
    \midrule
    I alt & 42 & 20 \\
    \bottomrule
  \end{tabularx}
  \TableSource{Egen fremstilling.}
  \TableNote{Tidsestimaterne er illustrative og afhænger af datamodenhed og integrationsniveau.}
\end{table}

Det samlede besparelsespotentiale kan udtrykkes som:
\begin{equation*}
  T_{\text{besparelse}} = T_{\text{manuel}} - T_{\text{mvp}}
\end{equation*}
Formlen synliggør, at gevinsten primært afhænger af graden af standardisering og automatisering i dataindsamling og validering.

Der er samtidig begrænsninger. Integrationsniveauet er grundlæggende og kræver manuelle eller semiautomatiske input, og avancerede funktioner som benchmarking og alerts ligger uden for MVP'ens scope. Evalueringen understøtter dermed, at MVP'en er egnet til compliance og dokumentation, men at yderligere funktionalitet er nødvendig for strategisk anvendelse og skalering.

Resultaterne peger på, at værdiskabelse, governance og implementering især afhænger af datakvalitet og organisatorisk forankring.
