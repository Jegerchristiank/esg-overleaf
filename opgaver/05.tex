\section{Metode}
\label{sec:metode}

Metodeafsnittet beskriver, hvordan dokumentanalyse af regulatoriske krav, empirisk sektorviden, en casebaseret pilot og en MVP som artefakt kobles, så der skabes \textit{sporbarhed} mellem policy, organisatorisk praksis og teknisk operationalisering af ESG-krav. Det giver et konsistent grundlag for at vurdere implementering og beslutningsrelevans.
\subsection{Design og tilgang}
\label{subsec:design-og-tilgang}

Underafsnittet præciserer designvalg og den abduktive logik, der forbinder teori, empiri og artefakt.
Undersøgelsen anvender et kvalitativt, eksplorativt design. Formålet er at forklare, hvordan ESG-krav omsættes til organisatoriske processer og tekniske datastrukturer i sundhedssektoren, snarere end at teste kausale effekter. Designet kombinerer dokumentanalyse af regulering med empirisk sektorkontekst og en casebaseret pilot.

Tilgangen er \textit{abduktiv}: den teoretiske ramme bruges til at strukturere fortolkningen af case og MVP, mens empiriske fund justerer og nuancerer analysen. Der arbejdes i tre analytiske spor: (1) regulatorisk kortlægning og kravfortolkning, (2) empirisk kontekstualisering af sektorens ESG-udfordringer, og (3) \textit{artefaktanalyse} af MVP'en som operationalisering af krav til data, sporbarhed og rapportoutput. Det giver en tydelig kobling mellem teori, empiri og artefakt.

Som analytisk greb anvendes en \textit{traceability-matrice}, der binder retskilde til datapunkt, input, validering, beregning, output og audit-evidens (eksempel i tabel \ref{tab:traceability-matrix}, fuld matrix i bilagstabel \ref{tab:traceability-matrix-full}). \textbf{Matricen fungerer som et fast mapping-skema, der genbruges i analysen og i evalueringen af MVP'en, så hvert delresultat kan spores tilbage til et konkret krav og et dokumenteret dataflow.}

\textit{Triangulering} opnås ved at sammenholde regulatoriske dokumenter, empiriske sektorkilder og interne case- og softwarekilder. Det giver et samlet billede af, hvordan standarder, \textit{governance} og praktiske constraints former ESG-rapportering for SMV'er i sundhedssektoren. \textcite{OECD2020ESG} peger på fragmenterede ESG-data og begrænset sammenlignelighed, hvilket gør krydstjek mellem krav, sektorviden og praksisbaserede artefakter nødvendigt, og \textcite{Berg2022AggregateConfusion} underbygger, at datadivergens kan opstå, selv når standarder er fælles.

Sammenhængen mellem forskningsspørgsmål og evidensgrundlag fremgår af tabel \ref{tab:rq-evidens-overblik}.
\begin{table}[h]
  \caption{Forskningsspørgsmål og evidensgrundlag, der synliggør den røde tråd i analysen.}
  \label{tab:rq-evidens-overblik}
  \begin{tabularx}{\textwidth}{X X l}
    \toprule
    Forskningsspørgsmål & Primært evidensgrundlag & Afsnit \\
    \midrule
    RQ1: Regulative rammer og konsekvenser for SMV'er. & CSRD/ESRS/EU-taksonomien/GRI samt sektorstatistik og nationale vejledninger. & \ref{sec:kontekst-og-rammer} \\
    RQ2: Værdiforslag og governance i ESG-as-a-Service. & Teori om standardisering, governance og materialitet samt markeds- og praksiskilder. & \ref{sec:teoretisk-ramme}, \ref{sec:analyse} \\
    RQ3: MVP som operationalisering af krav. & Case- og softwaremateriale, artefaktanalyse og bilag. & \ref{sec:software-og-mvp} \\
    \bottomrule
  \end{tabularx}
  \TableSource{Egen fremstilling.}
\end{table}

\subsection{Datagrundlag og kilder}
\label{subsec:datagrundlag-og-kilder}

Underafsnittet afgrænser datatyper, kildehierarki og evidensstyrke, så læseren kan se, hvad der bærer hvilke dele af analysen.
Datagrundlaget består af fire hovedkategorier: (1) regulatoriske kilder og standarder, (2) empiriske sektorkilder, (3) teoretisk litteratur og (4) case- og softwaremateriale. Kombinationen giver både normative rammer og praktiske indsigter, som er nødvendige for at analysere ESG-rapporteringens implementering.

Kilderne organiseres i et tydeligt \textit{kildehierarki}. Regulatoriske dokumenter og peer-reviewet litteratur udgør det normative og analytiske fundament, mens branche- og konsulentkilder primært bruges til at beskrive markedstendenser, implementeringspraksis og kontekstuelle forhold. Case- og softwaremateriale anvendes til at dokumentere operationalisering og sporbarhed. Opdelingen gør det klart, hvad der er krav og teori, og hvad der er kontekst, og reducerer risikoen for at konklusioner hviler på markedsnarrativer alene.

Regulatoriske kilder vurderes som høje på autenticitet og stabilitet, men kræver fortolkning i anvendelsen. Empiriske sektorkilder er robuste, men aggregerede og derfor mindre egnede til at forklare lokale variationer. Case- og softwaremateriale giver høj detaljegrad og sporbarhed, men har begrænset ekstern validitet og behandles derfor som casebaseret evidens. CSRD og ESRS fastlægger en bindende baseline for datapunkter og struktur, mens Global Reporting Initiative (GRI) og International Organization for Standardization (ISO) bidrager med frivillige principper for indikatorvalg og datakvalitet.\parenciteshort{EU2022CSRD,EU2023ESRS,GRI2021,ISO2024IWA48}

Datatyperne med periode, rolle og kvalitetsvurdering fremgår af tabel \ref{tab:datakilder-overblik}.
\begin{table}[h]
  \caption{Datakilder og kvalitetsvurdering, der afgrænser evidensstyrken.}
  \label{tab:datakilder-overblik}
  \begin{tabularx}{\textwidth}{X X X}
    \toprule
    Datakategori & Rolle i analysen & Kvalitetsvurdering \\
    \midrule
    Regulering og standarder (2020--2025) & Fastlægger krav, scope og væsentlighed. & Høj autenticitet; fortolkningsrum i praksis. \\
    Empiriske sektorkilder (2019--2024) & Understøtter sektorens ESG-profil og behov. & Robuste, men aggregerede data. \\
    Teoretisk litteratur (1970--2022) & Analytisk ramme for fortolkning. & Relevant, men kontekstafhængig. \\
    Case- og softwaremateriale (2025--2026) & Operationalisering af krav og processer (bilag \ref{app:teknisk-dokumentation}). & Høj detaljegrad; begrænset ekstern validitet. \\
    \bottomrule
  \end{tabularx}
  \TableSource{\parenciteshort{EU2022CSRD,EU2023ESRS,EU2020Taxonomy,GRI2021,HCWH2019ClimateFootprint,WHO2024HealthCareWaste,WHO2021HealthWorkerDeaths}}
  \TableNote{Oversigten er ikke udtømmende.}
\end{table}

Sporbarhed sikres gennem dokumenteret case- og softwaremateriale (bilag \ref{app:teknisk-dokumentation}). Det skaber grundlag for den efterfølgende caseanalyse.

For at løfte triangulering fra kildehierarki til \textit{inferenslogik} tydeliggøres, hvordan observationer kan forklares af konkurrerende mekanismer, og hvilke data der i princippet kan skelne mellem dem. Tabel \ref{tab:inferenslogik} gør alternative forklaringer testbare ved at koble dem til konkrete, observerbare implikationer.

\begin{table}[h]
  \caption{Inferenslogik med konkurrerende forklaringer og testbare implikationer.}
  \label{tab:inferenslogik}
  \begin{tabularx}{\textwidth}{>{\raggedright\arraybackslash}p{0.26\textwidth} >{\raggedright\arraybackslash}p{0.32\textwidth} >{\raggedright\arraybackslash}X}
    \toprule
    Observation & Alternative forklaringer & Testbar implikation og datakrav \\
    \midrule
    Frivillig ESG-rapportering i SMV'er & A) Regulatorisk forventning; B) Bank- og kundekrav. & A: adoption følger rapporteringsdeadlines; B: adoption følger kreditprocesser og kundekrav. Data: tidslinjer, kreditkrav, kundekontrakter. \\
    Forbedret sporbarhed i MVP & A) Teknisk auditspor driver kvalitet; B) Governance/rollefordeling driver kvalitet. & A: kvalitet forbedres uden ændret rollefordeling; B: kvalitet forbedres kun efter klar sign-off. Data: auditlogs, governance-roller, ændringshistorik. \\
    Værdiforslag for ESG-as-a-Service & A) Tidsbesparelse/omkostning; B) Risikoreduktion og adgang til kapital. & A: besparelser uafhængigt af kravintensitet; B: gevinster opstår primært ved ekstern efterspørgsel. Data: tidsmålinger, kreditvilkår, kundekrav. \\
    \bottomrule
  \end{tabularx}
  \TableSource{Egen fremstilling.}
\end{table}

Casen anvendes som pilot for operationalisering og konkretiserer datagrundlaget i praktiske processer.
\subsection{Case: Egen virksomhed som pilot}
\label{subsec:case-egen-virksomhed-som-pilot}

Casen tager udgangspunkt i en egen virksomhed, der udvikler en ESG-as-a-Service-løsning til SMV'er i sundhedssektoren. Casen anvendes som pilot for at konkretisere, hvordan regulatoriske krav omsættes til datafelter, processer og rapportoutput i en praktisk kontekst. \textcite{InternalSoftwareSpec2026} og \textcite{InternalSoftwareDetails2026} dokumenterer, at system og dataflows er beskrevet i interne specifikationer, der samles i bilag \ref{app:teknisk-dokumentation}.

Datamaterialet er struktureret i en caseskabelon, der dækker produktbeskrivelse, målgruppe, leveranceprocesser og økonomiske antagelser. Casen beskriver modulbaseret dataindsamling inden for energi, affald og sociale KPI'er og viser, hvordan input valideres og omsættes til rapportoutput. Den giver dermed et konkret grundlag for at vurdere gennemførbarhed, ressourcebehov og organisatoriske implikationer.

Casen er valgt, fordi den har karakteristika, der ligner en typisk SMV-kontekst med begrænsede ressourcer og samtidig markeds- og compliancepres. Samtidig indebærer valget en risiko for bias, da materialet er egenproduceret. Det håndteres ved at afgrænse generaliseringer og anvende casen som illustrativ evidens frem for statistisk repræsentativ dokumentation.

Egenproduceret case kræver særlig opmærksomhed på etik, compliance og kvalitet.
\subsection{Etik, compliance og kvalitet}
\label{subsec:etik-compliance-og-kvalitet}

Etik og compliance vurderes med udgangspunkt i EU's databeskyttelsesforordning (GDPR) og generelle krav til behandling af personoplysninger i en ESG-kontekst. \textcitefull{EU2016GDPR} fastlægger, at principper om dataminimering, formålsbegrænsning og adgangskontrol anvendes, når persondata indgår (fx medarbejderdata). For case og MVP betyder det, at datatyper holdes på et nødvendigt niveau, og at adgang til systemet er rollebaseret.

Kvalitet og sporbarhed understøttes af versionshistorik og auditspor, der dokumenterer ændringer i data og beregninger (bilag \ref{app:teknisk-dokumentation}). Det gør det muligt at efterprøve resultater og identificere antagelser og datakilder bag rapporterede tal. Transparens om antagelser indgår som en eksplicit del af den faglige redegørelse.

Der er en potentiel interessekonflikt, idet casen bygger på egen virksomhed. Dette håndteres ved at dokumentere metodevalg og antagelser eksplicit og ved at afgrænse analysens generaliserbarhed.

\subsection{Validitet og afbødning}
\label{subsec:begraensninger}

Begrænsninger håndteres gennem konkrete afbødninger, så evidensens svagheder reduceres aktivt frem for kun at blive noteret. Tabel \ref{tab:validitet-afboedning} samler de centrale trusler, deres konsekvens og den praktiske afbødning.

\begin{table}[h]
  \caption{Validitet: trussel, konsekvens og afbødning.}
  \label{tab:validitet-afboedning}
  \begin{tabularx}{\textwidth}{l X X}
    \toprule
    Trussel & Konsekvens & Afbødning \\
    \midrule
    Single-case og egenproduceret materiale & Begrænset generaliserbarhed og risiko for bias. & Self-audit suppleres med negativ test og reproducerbarhedstest; anbefaling om uafhængig replikation. \\
    Begrænset empirisk bredde & Få observationer om lokale variationer. & Tydelig afgrænsning af evidensscope og eksplicit kildehierarki med inklusionskriterier. \\
    Regulering i bevægelse & Usikkerhed om timing og krav. & Stop-the-clock behandles som metodisk usikkerhed; scenarier formuleres som betingede. \\
    Artefaktanalyse uden multi-site test & Risiko for overstating af effekt. & Foruddefineret evalueringsprotokol og dokumenterede mapping-regler i bilag. \\
    \bottomrule
  \end{tabularx}
  \TableSource{Egen fremstilling.}
\end{table}

Stop-the-clock og andre tidsplansskift behandles derfor som metodisk usikkerhed: ændrede tidslinjer påvirker incitamenter og betalingsvillighed, hvilket betyder, at den kommercielle analyse er betinget af, at rapporteringskrav enten fastholdes eller udskydes.

Begrænsningerne indebærer, at konklusionerne primært er analytiske og konceptuelle og bør valideres i fremtidige empiriske studier.
