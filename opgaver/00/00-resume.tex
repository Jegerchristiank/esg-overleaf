\section*{Resumé}
\addcontentsline{toc}{section}{Resumé}

Undersøgelsen analyserer rapportering af miljø-, sociale og governanceforhold (ESG) i sundhedssektoren med fokus på regulatoriske rammer, organisatorisk implementering og økonomisk relevans for små og mellemstore virksomheder (SMV'er). Formålet er at vurdere, hvordan ESG-as-a-Service og et Minimum Viable Product (MVP) kan omsætte krav til dataindsamling, sporbarhed og rapportoutput.

Metoden kombinerer dokumentanalyse af CSRD, ESRS, EU-taksonomien og GRI med empiriske sektorkilder om klimaaftryk, affald og arbejdsvilkår. Dertil kommer et casebaseret pilotstudie af egen virksomhed og en artefaktanalyse af MVP'ens dataflow og funktionalitet.

Analysen viser, at ESG fungerer som en organisationsstandard, der skaber legitimitet og sammenlignelighed, men samtidig risiko for dekobling mellem rapportering og praksis. ESG-as-a-Service kan reducere denne risiko ved at tilbyde standardiserede arbejdsgange, validering og auditspor. Værdiforslaget afhænger dog af, at indsatsen rettes mod materielle temaer og at data forankres organisatorisk.

Konklusionen er, at MVP'en demonstrerer praktisk gennemførbarhed for SMV'er, men at udbredelse kræver gradvis implementering, bedre integrationer og fortsat fokus på datakvalitet. Der peges på en trinvist opbygget ESG-rapportering, hvor minimumsmoduler etableres først og udvides i takt med kapacitet og modenhed.
