% IMPORTANT! In order for the document to compile, one needs to use XeLaTeX or LuaLaTeX as compiler.
% In Overleaf this can be done via Menu -> Settings -> Compiler -> Choose XeLaTeX/LuaLaTeX.
\documentclass[12pt]{article}

% Common packages and configuration for the KU thesis template
\usepackage{KUstyle}
\usepackage[margin=0.8in]{geometry}
\usepackage{iftex}
\usepackage{fontspec}
\usepackage[english,danish]{babel}
\usepackage{csquotes}
\usepackage{setspace}
\usepackage{tabularx}
\usepackage{amsmath}
\usepackage{siunitx}
\usepackage{mdframed}
\usepackage{booktabs} % Consistent table rules for evidence tables.

% Package decisions:
% - siunitx: used for numeric examples/units.
% - longtable: add only if a table must span multiple pages.

% Label conventions:
% sec:, subsec:, fig:, tab:, eq:, app:

% Main font configuration
\setmainfont{Arial}

% Numeric and unit formatting (Danish decimal comma).
\sisetup{
  output-decimal-marker = {,},
  group-separator = {.},
  group-minimum-digits = 4
}

% Table/figure source and note helpers (use below floats).
\newcommand{\TableSource}[1]{\par{\small\textit{Kilde: #1}}}
\newcommand{\TableNote}[1]{\par{\small\textit{Note: #1}}}
\newcommand{\FigureSource}[1]{\par{\small\textit{Kilde: #1}}}

% Table standard: caption above (label after caption), booktabs rules, source/note below tabular.
% Figure standard: PDF/PNG preferred, 300 dpi for raster, caption+label, source below.

% Bibliography setup
\usepackage[backend=biber,style=authoryear,maxcitenames=2,giveninits=true]{biblatex}
\addbibresource{references.bib}

% Customisation of default names
\renewcommand{\contentsname}{Indholdsfortegnelse}

% Legal reference helpers.
\newcommand{\lawpar}[1]{\S~#1}
\newcommand{\lawart}[1]{art.~#1}

% Quote styling: make display quotes visually distinct and attributable.
\newmdenv[
  leftline=true,
  rightline=false,
  topline=false,
  bottomline=false,
  linecolor=KUrod,
  linewidth=2pt,
  backgroundcolor=black!5,
  skipabove=\baselineskip,
  skipbelow=\baselineskip,
  innerleftmargin=0.8em,
  innerrightmargin=0.8em,
  innertopmargin=0.6em,
  innerbottommargin=0.6em
]{quoteblock}

\newcommand{\quoteattrib}[1]{\par\hfill--- \normalfont #1}

\renewenvironment{displayquote}{\begin{quoteblock}\small\itshape}{\end{quoteblock}}

% Normal page count (2400 characters incl. spaces) for the cover.
% LuaLaTeX writes normalpages-count.tex at end; recompile to refresh the cover.
\newcommand{\NormalPageCount}{?}
\newcommand{\NormalCharCount}{0}
\InputIfFileExists{normalpages-count.tex}{}{}

\ifluatex


  \newcommand{\NormalTextOff}{%
    \global\advance\KU@countlevel by 1\relax
    \KU@updatecountstate
  }

  \newcommand{\NormalTextOn}{%
    \global\advance\KU@countlevel by -1\relax
    \ifnum\KU@countlevel<0\relax
      \global\KU@countlevel=0\relax
    \fi
    \KU@updatecountstate
  }

  \AtBeginDocument{%
    \KU@updatecountstate
    \directlua{
      ku_charcount = 0
      ku_counting = false

      local node_id = node.id
      local glyph_id = node_id("glyph")
      local glue_id = node_id("glue")
      local hlist_id = node_id("hlist")
      local vlist_id = node_id("vlist")
      local ins_id = node_id("ins")
      local math_id = node_id("math")

      local glue_subtypes = node.subtypes("glue")
      local space_subtypes = {}
      if glue_subtypes then
        local spaceskip = glue_subtypes.spaceskip or glue_subtypes.space
        local xspaceskip = glue_subtypes.xspaceskip or glue_subtypes.xspace
        if spaceskip then
          space_subtypes[spaceskip] = true
        end
        if xspaceskip then
          space_subtypes[xspaceskip] = true
        end
      end

      local function count_list(head)
        for n in node.traverse(head) do
          local id = n.id
          if id == glyph_id then
            ku_charcount = ku_charcount + 1
          elseif id == glue_id then
            if space_subtypes[n.subtype] then
              ku_charcount = ku_charcount + 1
            end
          elseif id == hlist_id or id == vlist_id or id == ins_id or id == math_id then
            local list = n.list
            if list then
              count_list(list)
            end
          end
        end
      end

      local function count_paragraph(head)
        if ku_counting then
          count_list(head)
        end
        return head
      end

      callback.register("pre_linebreak_filter", count_paragraph)
    }
  }

  \AtEndDocument{%
    \directlua{
      local pages = ku_charcount / 2400
      local pages_str = string.format("%.1f", pages)
      pages_str = pages_str:gsub("%.", ",")
      local f = io.open("normalpages-count.tex", "w")
      if f then
        f:write("%% Auto-generated.\n")
        f:write("\\renewcommand{\\NormalPageCount}{", pages_str, "}\n")
        f:write("\\renewcommand{\\NormalCharCount}{", tostring(ku_charcount), "}\n")
        f:close()
      end
    }
  }
\else
  \newcommand{\NormalTextOff}{}
  \newcommand{\NormalTextOn}{}
\fi

\AddToHook{env/table/begin}{\NormalTextOff}
\AddToHook{env/table/end}{\NormalTextOn}
\AddToHook{env/table*/begin}{\NormalTextOff}
\AddToHook{env/table*/end}{\NormalTextOn}
\AddToHook{env/figure/begin}{\NormalTextOff}
\AddToHook{env/figure/end}{\NormalTextOn}
\AddToHook{env/figure*/begin}{\NormalTextOff}
\AddToHook{env/figure*/end}{\NormalTextOn}
\AddToHook{env/tabular/begin}{\NormalTextOff}
\AddToHook{env/tabular/end}{\NormalTextOn}
\AddToHook{env/tabular*/begin}{\NormalTextOff}
\AddToHook{env/tabular*/end}{\NormalTextOn}
\AddToHook{env/tabularx/begin}{\NormalTextOff}
\AddToHook{env/tabularx/end}{\NormalTextOn}
\AddToHook{env/longtable/begin}{\NormalTextOff}
\AddToHook{env/longtable/end}{\NormalTextOn}
\AddToHook{env/thebibliography/begin}{\NormalTextOff}
\AddToHook{cmd/printbibliography/before}{\NormalTextOff}
\AddToHook{cmd/appendix/after}{\NormalTextOff}
\AddToHook{cmd/caption/before}{\NormalTextOff}
\AddToHook{cmd/caption/after}{\NormalTextOn}
\AddToHook{cmd/captionof/before}{\NormalTextOff}
\AddToHook{cmd/captionof/after}{\NormalTextOn}


% This change the content of the frontpage
\ptype{Selvtilrettelagt valgfag (15 ECTS)}
\author{Christian Kristensen}
\title{ESG-rapportering i sundhedssektoren}
\subtitle{Policy, organisering, økonomisk relevans og kommercialisering}
\date{Dato: 15. januar 2026}
\advisor{Vejleder: Karsten Vrangbæk}
\normalsider{Normalsider (2400 tegn inkl. mellemrum): \NormalPageCount}
% \fpimage{picture.png} % Optional cover image; leave commented for KU SUND front page.

\begin{document}

\maketitle

\onehalfspacing

\begin{table}[h]
\def\arraystretch{1.5}
\begin{tabularx}{\textwidth}{l X}
Institution: & Københavns Universitet, Fakultet for Sundheds- og Medicinvidenskab \\
Institut: & Institut for Folkesundhedsvidenskab \\
Uddannelse: & Bachelor i Sundhed og Informatik (5. semester) \\
Kursus: & SITA17001U Individuelt tilrettelagt studieenhed (15 ECTS) \\
Forfatter: & Christian Kristensen \\
KUID: & tdh522 \\
Titel og undertitel: & ESG-rapportering i sundhedssektoren: Policy, organisering, økonomisk relevans og kommercialisering \\
Vejleder: & Karsten Vrangbæk \\
Prøveform: & Skriftlig rapport \\
Bedømmelse: & Bestået / Ikke bestået \\
Maks. omfang: & 40 sider \\
Dato: & 15. januar 2026
\end{tabularx}
\end{table}
\newpage

\tableofcontents
\clearpage
\NormalTextOn

\section*{Resumé}
\addcontentsline{toc}{section}{Resumé}

Undersøgelsen analyserer rapportering af miljø-, sociale og governanceforhold (ESG) i sundhedssektoren med fokus på regulatoriske rammer, organisatorisk implementering og økonomisk relevans for små og mellemstore virksomheder (SMV'er). Formålet er at vurdere, hvordan ESG-as-a-Service og et Minimum Viable Product (MVP) kan omsætte krav til dataindsamling, sporbarhed og rapportoutput.

Metoden kombinerer dokumentanalyse af CSRD, ESRS, EU-taksonomien og GRI med empiriske sektorkilder om klimaaftryk, affald og arbejdsvilkår. Dertil kommer et casebaseret pilotstudie af egen virksomhed og en artefaktanalyse af MVP'ens dataflow og funktionalitet.

Analysen viser, at ESG fungerer som en organisationsstandard, der skaber legitimitet og sammenlignelighed, men samtidig risiko for dekobling mellem rapportering og praksis. ESG-as-a-Service kan reducere denne risiko ved at tilbyde standardiserede arbejdsgange, validering og auditspor. Værdiforslaget afhænger dog af, at indsatsen rettes mod materielle temaer og at data forankres organisatorisk.

Konklusionen er, at MVP'en demonstrerer praktisk gennemførbarhed for SMV'er, men at udbredelse kræver gradvis implementering, bedre integrationer og fortsat fokus på datakvalitet. Der peges på en trinvist opbygget ESG-rapportering, hvor minimumsmoduler etableres først og udvides i takt med kapacitet og modenhed.

\clearpage
\section*{Abstract}
\addcontentsline{toc}{section}{Abstract}

\begin{otherlanguage}{english}
This report examines environmental, social, and governance (ESG) reporting in the healthcare sector with a focus on regulatory frameworks, organizational implementation, and economic relevance for small and medium-sized enterprises (SMEs). The goal is to assess how an ESG-as-a-Service concept and an MVP can translate requirements into data collection, traceability, and report outputs.

The study combines document analysis of CSRD, ESRS, the EU Taxonomy, and GRI with empirical sector sources on climate footprint, waste, and working conditions. It also includes a case-based pilot of the author's company and an artefact analysis of the MVP's data flow and functionality.

The analysis shows that ESG operates as an organizational standard that enhances legitimacy and comparability but also introduces a risk of decoupling between reporting and practice. ESG-as-a-Service can mitigate this risk by providing standardized workflows, validation, and audit trails. The value proposition depends on prioritizing material topics and embedding data practices in the organization.

The conclusion is that the MVP demonstrates practical feasibility for SMEs, but scaling requires gradual implementation, stronger integrations, and sustained focus on data quality. The report therefore recommends a phased reporting approach, starting with minimum modules and expanding as capacity and maturity grow.
\end{otherlanguage}

\clearpage
\input{opgaver/01/01-01-begrebsliste}
\clearpage

\section{Indledning}
\label{sec:indledning}

\ESG-rapportering i sundhedssektoren er i stigende grad et styrings- og legitimitetskrav snarere end en frivillig kommunikationsopgave. Sektoren kombinerer kritiske ydelser, komplekse forsyningsk\ae der og h\o je krav til dokumentation, hvilket g\o r sporbar og konsistent rapportering s\ae rligt udfordrende for mange akt\o rer.

Rapporten analyserer, hvordan \CSRD, ESRS, EU-taksonomien og GRI oms\ae tter politiske m\aa ls\ae tninger til konkrete rapporteringskrav, og hvordan disse krav kan operationaliseres i praksis for \SMV'er i sundhedssektoren. Analysen kombinerer policy- og organisationsperspektiver med et \ESG-as-a-Service-koncept og en \MVP-baseret case, der viser dataindsamling, sporbarhed og rapportoutput.

Hovedproblemet er pr\ae ciseret i \ref{subsec:problemformulering} og afgr\ae nset i \ref{subsec:afgraensning}. Baggrund og motivation uddybes i \ref{subsec:baggrund-og-motivation}, mens kontekst, teori, metode og software behandles i \ref{sec:kontekst-og-rammer}--\ref{sec:software-og-mvp}. Analysen og diskussionen samler resultaterne i \ref{sec:analyse} og \ref{sec:diskussion}, og de endelige svar opsummeres i \ref{sec:konklusion}.

\subsection{Baggrund og motivation}
\label{subsec:baggrund-og-motivation}


\subsection{Form\aa l og forskningssp\o rgsm\aa l}
\label{subsec:formaal-og-forskningssporgsmaal}

Form\aa let med rapporten er at analysere \ESG-rapportering i sundhedssektoren med s\ae rligt fokus p\aa regulatoriske krav, organisatorisk implementering og \o konomisk relevans for \SMV'er. Rapporten unders\o ger samtidig, hvordan et \ESG-as-a-Service-koncept og en \MVP kan oms\ae tte krav til konkret dataindsamling, sporbarhed og rapportoutput.

Rapporten besvarer f\o lgende forskningssp\o rgsm\aa l:
\begin{enumerate}
  \item Hvordan p\aa virker \CSRD, ESRS, EU-taksonomien og GRI kravene til \ESG-rapportering i sundhedssektoren, is\ae r for \SMV'er?
  \item Hvilket v\ae rdiforslag skaber \ESG-as-a-Service i forhold til standardisering, governance og compliance?
  \item Hvordan kan en \MVP oms\ae tte regulatoriske krav til dataindsamling, sporbarhed og rapportoutput?
\end{enumerate}

Sp\o rgsm\aa lene besvares gennem analysen i \ref{sec:analyse}, diskuteres i \ref{sec:diskussion} og sammenfattes i \ref{sec:konklusion}.

\subsection{Problemformulering}
\label{subsec:problemformulering}

Analysen tager udgangspunkt i ESG som samlebetegnelse for miljø-, sociale og governance-forhold, der forventes dokumenteret i virksomheders rapportering. SMV'er forstås som ikke-børsnoterede virksomheder, der falder under EU's SMV-definition, og som typisk har begrænsede ressourcer til compliance og datastyring.\parencite{Virksomhedsguiden2025SMVDefinition} ESG-as-a-Service anvendes her som betegnelse for en kombineret software- og serviceleverance, der oversætter standardkrav til operationelle datapunkter, kontroller og rapportoutput.

Reguleringen skaber et pres for ensartet og sporbar rapportering, men i sundhedssektoren er data ofte fragmenterede på tværs af systemer, leverandører og organisatoriske enheder. Dette gør der en kløft mellem normative standarder og den praktiske mulighed for at levere valide og sammenlignelige ESG-data.

\textbf{Problemformulering:} Hvordan kan ESG-as-a-Service, understøttet af en MVP, reducere kløften mellem regulatoriske krav og operationel datapraksis for SMV'er i sundhedssektoren, så rapportering bliver både sporbar og beslutningsrelevant?

Problemformuleringen fungerer som gennemgående test i rapporten: Hvert kapitel bidrager med en del af forklaringen på, hvordan krav oversættes til praksis og hvor rapporteringens beslutningsrelevans opstår eller brydes.

\subsection{Afgr\ae nsning}
\label{subsec:afgraensning}

Rapporten afgr\ae nser sig p\aa f\o lgende punkter:
\begin{itemize}
  \item Geografisk og regulatorisk fokus er EU/Danmark med \CSRD, ESRS, EU-taksonomien og GRI som centrale rammer.
  \item Sektorfokus er sundhedssektoren, prim\ae rt private klinikker, mindre hospitaler og medtech-leverand\o rer.
  \item Virksomhedsfokus er \SMV'er; store b\o rsnoterede virksomheder behandles kun som kontekst.
  \item Emnefokus er \ESG-rapportering og compliance; fulde LCA-modeller og dyb klima-science er ikke omfattet.
  \item Empirien bygger prim\ae rt p\aa sekundaerdata samt egen case og interne softwarebeskrivelser.
  \item Ekstern assurance, fuld finansiel v\ae rdians\ae ttelse og brancheregnskaber er uden for scope.
\end{itemize}

\subsection{Rapportens opbygning}
\label{subsec:rapportens-opbygning}



\section{Kontekst og rammer}
\label{sec:kontekst-og-rammer}

Form\aa let med dette kapitel er at etablere den sektorielle og regulatoriske ramme, som resten af rapporten bygger p\aa. Kapitlet samler de centrale forhold, der former \ESG-rapportering i sundhedssektoren, og skaber et f\ae lles udgangspunkt for analyse og metode.

F\o rst beskrives sektorens \ESG-profil og centrale n\o gletal i \ref{subsec:esg-rapportering-i-sundhedssektoren}. Dern\ae st gennemg\aa s de regulatoriske rammer i \ref{subsec:regulatoriske-rammer-csrd-eu-taksonomien-og-gri}. \ref{subsec:smver-og-behovet-for-forenkling} behandler \SMV'ers ressourcevilk\aa r og behovet for forenkling, og \ref{subsec:esg-as-a-service-som-servicekoncept} introducerer \ESG-as-a-Service som servicekoncept og respons p\aa de identificerede barrierer.

\subsection{ESG-rapportering i sundhedssektoren}
\label{subsec:esg-rapportering-i-sundhedssektoren}


\subsection{Regulatoriske rammer: CSRD, EU-taksonomien og GRI}
\label{subsec:regulatoriske-rammer-csrd-eu-taksonomien-og-gri}

De regulatoriske rammer for ESG-rapportering i EU består af flere sammenhængende elementer. CSRD udvider rapporteringspligten og gør bæredygtighedsrapportering til en integreret del af virksomhedernes officielle rapportering. Direktivet fastsætter, at rapporteringen skal ske efter ESRS og i et standardiseret, digitalt format, og implementeringen er trinvist indfaset og politisk justeret gennem den såkaldte stop-the-clock-aftale.\parencite{EU2022CSRD,EU2023ESRS,Erhvervsstyrelsen2025StopClock}

ESRS konkretiserer de oplysninger, virksomhederne skal levere, og giver struktur til sammenhængen mellem strategi, risici, mål og resultater. Standarderne operationaliserer kravet om dobbelt væsentlighed og etablerer en fælles logik for datagrundlag og rapportering.\parencite{EU2023ESRS}

Dobbelt væsentlighed betyder, at virksomheder skal rapportere både deres påvirkning af mennesker og miljø og hvordan bæredygtighedsforhold påvirker virksomheden finansielt. Det udvider datakravet til værdikæde, governance og sociale forhold og gør materialitetsprocessen til en central del af rapporteringen.\parencite{EU2023ESRS}

EU-taksonomien supplerer rapporteringen ved at definere, hvilke økonomiske aktiviteter der kan betragtes som miljømæssigt bæredygtige. Rammeværket kræver, at virksomheder kan dokumentere overensstemmelse med taksonomien og derved knytte finansielle aktiviteter til konkrete miljømæssige mål.\parencite{EU2020Taxonomy}

GRI er et globalt, frivilligt rammeværk, som anvendes bredt til sammenlignelig ESG-rapportering. Det tilbyder en struktureret tilgang til indikatorer og narrativer og bruges ofte som supplement til regulatoriske krav.\parencite{GRI2021}
\begin{displayquote}
ESG does not currently benefit from a universally accepted common set of standards.
\end{displayquote}
\parencite{ESGBook}

Tabel \ref{tab:rammer-sammenligning} opsummerer de centrale forskelle mellem rammerne og deres implikationer for datakrav og rapporteringslogik.
\begin{table}[t]
  \caption{Sammenligning af centrale rammer for ESG-rapportering.}
  \label{tab:rammer-sammenligning}
  \begin{tabularx}{\textwidth}{l X X}
    \toprule
    Rammeværk & Status og formål & Implikation for rapportering \\
    \midrule
    CSRD/ESRS & Obligatorisk og indfaset i bølger; etablerer standardiserede ESG-krav. & Dobbelt væsentlighed og sammenhæng mellem strategi, risici og mål med krav om digital rapportering. \\
    EU-taksonomien & Obligatorisk supplement; klassificerer miljømæssigt bæredygtige aktiviteter. & Kræver dokumentation for aktiviteters bidrag og overensstemmelse med minimumsgarantier. \\
    GRI & Frivilligt rammeværk, anvendes globalt. & Indikatorbaseret rapportering med fokus på væsentlige forhold og sammenlignelighed. \\
    \bottomrule
  \end{tabularx}
  \TableSource{\parencite{EU2022CSRD,EU2023ESRS,EU2020Taxonomy,GRI2021}}
\end{table}

\subsection{SMV'er og behovet for forenkling}
\label{subsec:smver-og-behovet-for-forenkling}

SMV'er defineres efter EU's størrelsesgrænser og vil ofte ligge uden for den direkte CSRD-pligt, men påvirkes indirekte gennem kundekrav, banker og leverandørrelationer.\parencite{Virksomhedsguiden2025SMVDefinition,PwC2025GuideESGSMV,GrantThorntonDK2025Omnibus} Dette skaber et rapporteringspres, selv hvor rapporteringen formelt er frivillig.

EU har derfor udviklet en frivillig VSME-standard (Voluntary Sustainability Reporting Standard for SMEs), der skal sikre ensartet dataudveksling og forhindre uforholdsmæssige datakrav fra større virksomheder.\parencite{Virksomhedsguiden2025VSMEIntro} VSME er opbygget af et basismodul og et udvidet modul, hvor basismodullets 11 datapunkter kan anvendes som minimumsniveau.\parencite{Virksomhedsguiden2025VSMEModules} Standarden kræver ikke en dobbelt væsentlighedsanalyse, hvilket reducerer kompleksitet og ressourceforbrug.\parencite{Virksomhedsguiden2025NoDualMateriality} Erhvervsstyrelsen har samtidig udviklet en skabelon, der samler datapunkterne og understøtter ensartet rapportering.\parencite{Erhvervsstyrelsen2025ESGTemplate}

Indfasningen af CSRD er trinvist implementeret, og den seneste stop-the-clock-aftale udskyder rapporteringskrav for flere virksomhedstyper, hvilket giver SMV'er mere tid, men også skaber usikkerhed om krav og timing.\parencite{Erhvervsstyrelsen2025StopClock} Erfaringer fra de første VSME-rapporter viser, at mange virksomheder rapporterer ud over basismodullets krav, hvilket indikerer både ambition og behov for klar prioritering.\parencite{EY2025VSME}

Samlet peger udviklingen på et behov for forenklede processer, klare minimumskrav og teknisk støtte, så SMV'er kan levere sporbar ESG-dokumentation uden at belaste kerneopgaven.

\subsection{ESG-as-a-Service som servicekoncept}
\label{subsec:esg-as-a-service-som-servicekoncept}

ESG-as-a-Service betegner en kombineret software- og serviceleverance, der omsætter regulatoriske krav til operationelle datapunkter, kontroller og rapportoutput. Konceptet adskiller sig fra klassisk SaaS (Software as a Service) ved at inkludere faglig sparring, konfigurerede standarder og løbende datakvalitetssikring, men adskiller sig også fra traditionel konsulentbistand ved at bygge på en fast digital infrastruktur.

Værdiforslaget kan forankres i stakeholder- og shared value-perspektiver, hvor dokumenteret ESG-indsats er en forudsætning for legitimitet og langsigtet værdiskabelse.\parencite{Freeman1984Stakeholder,PorterKramer2011CSV,Elkington1998TripleBottomLine} Samtidig eksisterer en modposition, hvor virksomhedens primære ansvar er over for aktionærerne, hvilket understreger behovet for at gøre compliance og økonomisk relevans eksplicit i servicekonceptet.\parencite{Friedman1970Profits}

I en reguleret sundhedssektor fungerer ESG-as-a-Service som et organisatorisk mellemled, der kan standardisere dataindsamling, reducere transaktionsomkostninger og skabe en audit-log, som gør rapportering beslutningsrelevant for ledelse, revisor og myndigheder. Sektoren er et relevant startmarked, fordi compliance-krav, datakompleksitet og forsyningskædepres gør behovet for struktureret ESG-rapportering særligt tydeligt.


\section{Teoretisk ramme}
\label{sec:teoretisk-ramme}

Dette kapitel etablerer den teoretiske ramme, der anvendes til at analysere \ESG-rapportering i sundhedssektoren og \ESG-as-a-Service som svar p\aa regulatoriske krav. Valget af teori er styret af tre behov: at forst\aa \ESG som standardiseringspraksis, at forklare hvordan policy oms\ae ttes til organisatorisk handling, og at vurdere \o konomisk relevans og kommercialiseringsmuligheder.

Rammen best\aa r derfor af (1) standardisering og organisationer med afs\ae t i Brunsson, (2) policy- og governanceperspektiver med fokus p\aa translasjon og institutionel styring, og (3) \o konomiske og kommercielle perspektiver p\aa v\ae rdiskabelse og performance. Tilsammen udg\o r de et analysekatalog, der bruges til at fortolke empirien, vurdere \MVP'ens rolle og strukturere analysen i kapitel \ref{sec:analyse}.

\subsection{Standardisering og organisationer (Brunsson)}
\label{subsec:standardisering-og-organisationer-brunsson}


\subsection{Policy- og governanceperspektiver}
\label{subsec:policy-og-governanceperspektiver}

Policy- og governanceperspektiver fokuserer p\aa, hvordan regulering oms\ae ttes til praksis gennem institutionelle mekanismer, standarder og mellemled. \ESG-rapportering i EU er et eksempel p\aa flerniveau-styring, hvor politiske m\aa ls\ae tninger realiseres gennem direktiver, standarder og vejledninger, som organisationer skal fortolke og implementere lokalt.

\textcite{RoevikTrenderTranslasjoner} beskriver, hvordan ideer og styringskoncepter vandrer mellem organisationer og bliver oversat til lokale praksisser. I hans translationsteori betones, at implementering ikke er en ren kopiering, men en proces, hvor krav redigeres, omformuleres og tilpasses til organisatoriske betingelser.\parencite{TranslationTheoryKnowledgeTransfer} Det betyder, at ensartede standarder kan skabe variation i praksis, afh\ae ngigt af akt\o rers kapacitet, fortolkning og incitamenter.

I \ESG-rapportering fungerer revisorer, konsulenter og softwareleverand\o rer som intermediaere akt\o rer, der overs\ae tter policy til datafelter, processer og beslutningsregler. Disse akt\o rer bidrager til governance ved at definere, hvad der opfattes som tilstr\ae kkelig dokumentation, og ved at skabe tekniske rammer for sporbarhed.

Analytisk betyder perspektivet, at rapporten unders\o ger, hvordan regulatoriske krav transformeres til operationelle krav i sundhedssektoren, og hvordan \ESG-as-a-Service fungerer som en oversaettelsesmekanisme mellem policy og praksis. Det giver et grundlag for at vurdere, hvorvidt governance styrker faktisk implementering eller prim\ae rt producerer formel overholdelse.

\subsection{\O konomiske og kommercielle perspektiver}
\label{subsec:oekonomiske-og-kommercielle-perspektiver}


\subsection{Opsamling af den teoretiske ramme}
\label{subsec:opsamling-af-den-teoretiske-ramme}

Den teoretiske ramme samler tre komplementære perspektiver, der tilsammen forklarer, hvorfor ESG-rapportering opfattes som nødvendig, hvordan den implementeres, og hvilke økonomiske konsekvenser den kan have. Standardisering bidrager med begreber om legitimitet og sammenlignelighed, governance forklarer oversættelse af policy til praksis, og økonomiske perspektiver adresserer værdiskabelse og betalingsvillighed.

Tabel \ref{tab:teoretisk-ramme-overblik} opsummerer de centrale begreber og viser, hvordan de anvendes i analysen.
\begin{table}[t]
  \caption{Teoretiske linser og analytiske anvendelser i rapporten.}
  \label{tab:teoretisk-ramme-overblik}
  \begin{tabularx}{\textwidth}{l X X}
    \toprule
    Perspektiv & Kernebegreber & Analytisk anvendelse \\
    \midrule
    Standardisering & Legitimitet, sammenlignelighed, dekobling. & Vurderer hvordan ESG-krav omsættes til standardiserede data og om rapportering bliver substans eller symbol. \\
    Policy og governance & Translasjon, intermediære aktører, flerniveau-styring. & Forklarer hvordan regulering og standarder oversættes til lokale processer og tekniske arbejdsgange. \\
    Økonomi og kommercialisering & Stakeholder/shareholder, shared value, finansiel materialitet. & Vurderer økonomisk relevans, forretningsmodel og betalingsvillighed for ESG-as-a-Service. \\
    \bottomrule
  \end{tabularx}
  \TableSource{Egen fremstilling.}
\end{table}

Opsamlingen fungerer som analytisk ramme for kapitel \ref{sec:analyse}, hvor teorierne anvendes til at vurdere ESG-rapporteringens praksis og MVP'ens kommercielle og organisatoriske implikationer.


\section{Metode}
\label{sec:metode}

Metodeafsnittet redegør for tilrettelæggelsen af undersøgelsen med fokus på sporbarhed, konsistens og analytisk holdbarhed. Tilgangen kombinerer dokumentanalyse af regulatoriske krav med empirisk sektorviden, en casebaseret pilot og en analyse af en MVP som artefakt. Det muliggør en kobling mellem policy, organisatorisk praksis og teknisk operationalisering af ESG-krav.

I \ref{subsec:design-og-tilgang} beskrives forskningsdesign og analysestrategi. Datagrundlag og kilder uddybes i \ref{subsec:datagrundlag-og-kilder}, mens caseafgrænsning og relevans behandles i \ref{subsec:case-egen-virksomhed-som-pilot}. Etik, compliance og kvalitetskontrol er samlet i \ref{subsec:etik-compliance-og-kvalitet}, og de metodiske begrænsninger opsummeres i \ref{subsec:begraensninger}.

Metoden etablerer dermed grundlaget for, at softwarekapitlet kan fungere som operationalisering af kravene og som empirisk artefakt i analysen.

\subsection{Design og tilgang}
\label{subsec:design-og-tilgang}

Undersøgelsen anvender et kvalitativt, eksplorativt design. Formålet er at forklare, hvordan ESG-krav omsættes til organisatoriske processer og tekniske datastrukturer i sundhedssektoren, snarere end at teste kausale effekter. Designet kombinerer dokumentanalyse af regulering med empirisk sektorkontekst og en casebaseret pilot.

Tilgangen er abduktiv: den teoretiske ramme bruges til at strukturere fortolkningen af case og MVP, mens empiriske fund justerer og nuancerer analysen. Der arbejdes i tre analytiske spor: (1) regulatorisk kortlægning og kravfortolkning, (2) empirisk kontekstualisering af sektorens ESG-udfordringer, og (3) artefaktanalyse af MVP'en som operationalisering af krav til data, sporbarhed og rapportoutput.

Triangulering opnås ved at sammenholde regulatoriske dokumenter, empiriske sektorkilder og interne case- og softwarekilder. Det giver et samlet billede af, hvordan standarder, governance og praktiske constraints former ESG-rapportering for SMV'er i sundhedssektoren.

\subsection{Datagrundlag og kilder}
\label{subsec:datagrundlag-og-kilder}

Datagrundlaget består af fire hovedkategorier: (1) regulatoriske kilder og standarder, (2) empiriske sektorkilder, (3) teoretisk litteratur og (4) case- og softwaremateriale. Kombinationen giver både normative rammer og praktiske indsigter, som er nødvendige for at analysere ESG-rapporteringens implementering.

Regulatoriske kilder vurderes som høj på autenticitet og stabilitet, men kræver fortolkning i anvendelsen. Empiriske sektorkilder er robuste, men aggregerede og derfor mindre egnede til at forklare lokale variationer. Case- og softwaremateriale giver detaljegrad og sporbarhed, men har begrænset ekstern validitet og behandles derfor som casebaseret evidens.

Tabel \ref{tab:datakilder-overblik} opsummerer datatyperne med periode, rolle og kvalitetsvurdering.
\begin{table}[t]
  \caption{Datakilder og kvalitetsvurdering, der afgrænser evidensstyrken.}
  \label{tab:datakilder-overblik}
  \begin{tabularx}{\textwidth}{X X X}
    \toprule
    Datakategori & Rolle i analysen & Kvalitetsvurdering \\
    \midrule
    Regulering og standarder (2020--2025) & Fastlægger krav, scope og væsentlighed. & Høj autenticitet; fortolkningsrum i praksis. \\
    Empiriske sektorkilder (2019--2024) & Understøtter sektorens ESG-profil og behov. & Robuste, men aggregerede data. \\
    Teoretisk litteratur (1970--2022) & Analytisk ramme for fortolkning. & Relevant, men kontekstafhængig. \\
    Case- og softwaremateriale (2025--2026) & Operationalisering af krav og processer (bilag \ref{app:teknisk-dokumentation}). & Høj detaljegrad; begrænset ekstern validitet. \\
    \bottomrule
  \end{tabularx}
  \TableSource{\parencite{EU2022CSRD,EU2023ESRS,EU2020Taxonomy,GRI2021,HCWH2019ClimateFootprint,WHO2024HealthCareWaste,WHO2021HealthWorkerDeaths}}
  \TableNote{Oversigten er ikke udtømmende.}
\end{table}

Alle eksterne kilder håndteres konsistent via \texttt{references.bib}. Case- og softwaremateriale er dokumenteret i bilag \ref{app:teknisk-dokumentation} for at sikre sporbarhed og efterprøvning.

\subsection{Case: Egen virksomhed som pilot}
\label{subsec:case-egen-virksomhed-som-pilot}

Casen tager udgangspunkt i en egen virksomhed, der udvikler en ESG-as-a-Service-løsning til SMV'er i sundhedssektoren. Casen anvendes som pilot for at konkretisere, hvordan regulatoriske krav omsættes til datafelter, processer og rapportoutput i en praktisk kontekst. Dokumentationen er samlet i bilag \ref{app:teknisk-dokumentation}.

Datamaterialet er struktureret i en caseskabelon, der dækker produktbeskrivelse, målgruppe, leveranceprocesser og økonomiske antagelser. Casen beskriver modulbaseret dataindsamling inden for energi, affald og sociale KPI'er og viser, hvordan input valideres og omsættes til rapportoutput. Den giver dermed et konkret grundlag for at vurdere gennemførbarhed, ressourcebehov og organisatoriske implikationer.

Casen er valgt, fordi den repræsenterer en typisk SMV-kontekst med begrænsede ressourcer og samtidig markeds- og compliancepres. Samtidig indebærer valget en risiko for bias, da materialet er egenproduceret. Det håndteres ved at afgrænse generaliseringer og anvende casen som illustrativ evidens frem for statistisk repræsentativ dokumentation.

Da casen er egenproduceret, adresseres etik, compliance og kvalitet eksplicit i det næste afsnit (\ref{subsec:etik-compliance-og-kvalitet}).

\subsection{Etik, compliance og kvalitet}
\label{subsec:etik-compliance-og-kvalitet}


\subsection{Begr\ae nsninger}
\label{subsec:begraensninger}



\section{Software og MVP}
\label{sec:software-og-mvp}


\subsection{Form\aa l og krav}
\label{subsec:formaal-og-krav}


\subsection{Systemoversigt og arkitektur}
\label{subsec:systemoversigt-og-arkitektur}

MVP'en er bygget som en webbaseret løsning med en klar opdeling mellem frontend, backend og database. Frontend er en Next.js-applikation, der håndterer brugerflow og dataindtastning, mens backend stiller et API til rådighed for lagring og hentning af data. Persistens sker i PostgreSQL, og en fælles pakke indeholder typer, validering og beregninger.\parencite{InternalSoftwareSpec2026,InternalSoftwareDetails2026}

Dataflowet er simpelt og kontrolleret: Frontend henter og persisterer snapshots via \texttt{/wizard/snapshot}, backend validerer og logger ændringer, og data lagres i kerne-tabellerne. Denne arkitektur understøtter sporbarhed, fordi alle ændringer registreres og kan genskabes, og skalerbarhed, fordi komponenterne kan udvides uafhængigt.

Figur \ref{fig:system-arkitektur} viser arkitekturens hovedkomponenter og deres relationer.
\begin{figure}[t]
  \centering
  \fbox{\begin{tabular}{c}
    Frontend (Next.js) \\
    $\downarrow$ API \\
    Backend (Node.js) \\
    $\downarrow$ \\
    PostgreSQL \\
  \end{tabular}}
  \caption{Systemoversigt for MVP med centrale komponenter.}
  \label{fig:system-arkitektur}
  \FigureSource{Egen fremstilling baseret på \parencite{InternalSoftwareSpec2026,InternalSoftwareDetails2026}}
\end{figure}

\subsection{Datamodel og ESG-indikatorer}
\label{subsec:datamodel-og-esg-indikatorer}

Datamodellen er designet til at understøtte sporbarhed gennem versionshistorik og auditspor. Den organiserer data i tre logiske lag: profil og afgrænsning, indikatorregistreringer og ændringshistorik.

Tabel \ref{tab:datamodel-kernetable} viser datamodellens hovedkomponenter og deres funktion i MVP'en.
\begin{table}[t]
  \caption{Hovedkomponenter i datamodellen.}
  \label{tab:datamodel-kernetable}
  \begin{tabularx}{\textwidth}{l X X}
    \toprule
    Dataobjekt & Formål & Eksempler på indhold \\
    \midrule
    Profil og afgrænsning & Fastlægger organisationens scope og relevante moduler. & Grundoplysninger, perioder, afgrænsninger. \\
    Indikatorregistrering & Samler data pr. modul til beregning og rapportering. & Forbrugstal, mængder, enheder, datakilder. \\
    Ændrings- og beregningsspor & Dokumenterer ændringer og udledte resultater. & Tidsstempel, ændringstype, beregningsgrundlag. \\
    \bottomrule
  \end{tabularx}
  \TableSource{Egen fremstilling; datadefinitioner fremgår af bilag \ref{app:teknisk-dokumentation}.}
\end{table}

MVP'en fokuserer på indikatorer inden for energi, affald og sociale forhold, i tråd med casebeskrivelsen og sektorens ESG-profil. Tabel \ref{tab:esg-indikatorer} viser eksempler på indikatorer og deres datakilder.
\begin{table}[t]
  \caption{Eksempler på ESG-indikatorer i MVP'en.}
  \label{tab:esg-indikatorer}
  \begin{tabularx}{\textwidth}{l X l X}
    \toprule
    Område & Eksempel på datafelt & Enhed & Typisk datakilde \\
    \midrule
    Energi & Varme- og elforbrug & kWh & Drifts- og energidata. \\
    Affald & Mængde affald pr. fraktion & kg/ton & Affaldsopgørelser og leverandørdata. \\
    Sociale forhold & Arbejdsmiljøindikatorer & antal/\% & HR-data og interne registreringer. \\
    \bottomrule
  \end{tabularx}
  \TableSource{Egen fremstilling baseret på case-materiale (bilag \ref{app:teknisk-dokumentation}).}
  \TableNote{Indikatorerne er illustrative og afspejler MVP'ens fokus.}
\end{table}

Indikatorerne er organiseret efter ESG-domæner, hvilket gør det muligt at mappe data til de overordnede strukturer i ESRS og GRI uden at påstå fuld dækning af alle datapunkter.\parencite{EU2023ESRS,GRI2021}
En samlet moduloversigt for MVP'en er placeret i bilagene (tabel \ref{tab:app-moduloversigt}) for at give et supplerende overblik over scopes og moduler.

\subsection{Dataindsamling og automatisering}
\label{subsec:dataindsamling-og-automatisering}


\subsection{Brugerflow og rapportoutput}
\label{subsec:brugerflow-og-rapportoutput}


\subsection{Demonstration og evaluering}
\label{subsec:demonstration-og-evaluering}

Evalueringen tager udgangspunkt i kravene i tabel \ref{tab:krav-prioritering} og demonstrerer, om MVP'en leverer de forventede funktioner. Demonstrationen er kvalitativ og bygger på tre scenarier, der dækker energi, affald og sociale indikatorer (tabel \ref{tab:testscenarier}).

\begin{table}[h]
  \caption{Testscenarier, der viser hvordan kravene afprøves mod konkrete outputs.}
  \label{tab:testscenarier}
  \begin{tabularx}{\textwidth}{l X X}
    \toprule
    Scenarie & Forventet output & Relaterede krav \\
    \midrule
    Energi (B2) & Valideret input, beregnet resultat og auditspor. & Dataindsamling, validering, beregning, auditspor. \\
    Affald & Registrering af mængder og konsistent rapportering. & Dataindsamling, validering, rapporteksport. \\
    Sociale indikatorer & Samlet KPI-oversigt til rapport. & Dataindsamling, rapporteksport. \\
    \bottomrule
  \end{tabularx}
  \TableSource{Egen fremstilling.}
  \TableNote{Scenarierne er illustrative og afspejler MVP'ens scope.}
\end{table}

Resultaterne viser, at MVP'en opfylder de centrale MVP-krav om dataindsamling, validering, beregning og sporbarhed. Brugerflowet giver en klar progression fra profil til modul og review, og output kan eksporteres i de forventede formater. Dokumentationen fremgår af bilag \ref{app:supplerende-figurer-og-tabeller} og \ref{app:teknisk-dokumentation}.

\subsubsection{Indikativt tidsestimat}
\label{subsec:indikativt-tidsestimat}

For at gøre værdiforslaget mere håndgribeligt kan tidsforbrug estimeres pr. rapporteringscyklus for en mindre klinik. Tabel \ref{tab:tidsforbrug-mvp} viser et groft overslag, hvor MVP'en reducerer manuel indsats på centrale aktiviteter.
\begin{table}[h]
  \caption{Illustrativt tidsforbrug før og efter MVP, der synliggør forventet effektivisering.}
  \label{tab:tidsforbrug-mvp}
  \begin{tabularx}{\textwidth}{X r r}
    \toprule
    Aktivitet & Manuel (timer) & MVP (timer) \\
    \midrule
    Dataindsamling (energi, affald, sociale KPI'er) & 24 & 12 \\
    Validering og fejlkontrol & 8 & 4 \\
    Konsolidering og rapporteksport & 10 & 4 \\
    \midrule
    I alt & 42 & 20 \\
    \bottomrule
  \end{tabularx}
  \TableSource{Egen fremstilling.}
  \TableNote{Tidsestimaterne er illustrative og afhænger af datamodenhed og integrationsniveau.}
\end{table}

Det samlede besparelsespotentiale kan udtrykkes som:
\begin{equation*}
  T_{\text{besparelse}} = T_{\text{manuel}} - T_{\text{mvp}}
\end{equation*}
Formlen synliggør, at gevinsten primært afhænger af graden af standardisering og automatisering i dataindsamling og validering.

Der er samtidig begrænsninger. Integrationsniveauet er grundlæggende og kræver manuelle eller semiautomatiske input, og avancerede funktioner som benchmarking og alerts ligger uden for MVP'ens scope. Evalueringen understøtter dermed, at MVP'en er egnet til compliance og dokumentation, men at yderligere funktionalitet er nødvendig for strategisk anvendelse og skalering.

Evalueringen danner det empiriske udgangspunkt for analysen af værdiskabelse, governance og implementering i \ref{sec:analyse}.


\section{Analyse}
\label{sec:analyse}

Analysen anvender den teoretiske ramme fra \ref{sec:teoretisk-ramme} og metodevalget i \ref{sec:metode} til at fortolke empirien og MVP'ens rolle i ESG-rapportering. Fokus er på, hvordan standarder og governance omsættes til praksis, hvordan værdiforslaget kan begrundes økonomisk, og hvilke organisatoriske konsekvenser en ESG-as-a-Service-løsning skaber for SMV'er i sundhedssektoren.

Afsnit \ref{subsec:esg-som-organisationsstandard} analyserer ESG som organisationsstandard med udgangspunkt i pensum om standardisering og translasjon. Afsnit \ref{subsec:vaerdiforslag-og-forretningsmodel} vurderer værdiforslag og forretningsmodel i relation til stakeholder-/shareholder-logikker og performance-evidens. Prisfastsættelse og segmentering behandles i \ref{subsec:prisfastsaettelse-og-kundesegmentering}, mens \ref{subsec:organisatoriske-og-oekonomiske-implikationer} diskuterer governance og ressourcekrav. \ref{subsec:implementering-i-sundhedssektoren} samler barrierer og incitamenter for implementering i sektoren.

\subsection{ESG som organisationsstandard}
\label{subsec:esg-som-organisationsstandard}


\subsection{Værdiforslag og forretningsmodel}
\label{subsec:vaerdiforslag-og-forretningsmodel}

Værdiforslaget i ESG-as-a-Service kan forstås gennem en kombination af stakeholder- og shared value-perspektiver. Stakeholder-tilgangen betoner, at legitimitet og ansvarlighed skaber værdi for flere interessenter end aktionærer, mens shared value argumenterer for, at samfundsmæssige udfordringer kan omsættes til forretningsmuligheder.\parencite{Freeman1984Stakeholder,PorterKramer2011CSV} Dette støttes af pensum om ESG, som understreger behovet for at integrere ESG i kerneforretningen og bruge data som beslutningsgrundlag, ikke kun som compliance.\parencite{ESGBook}

MVP'ens værdiforslag består i at reducere transaktionsomkostninger ved dataindsamling, standardisere input og skabe sporbar dokumentation, der kan anvendes i dialog med banker, kunder og myndigheder (bilag \ref{app:teknisk-dokumentation}). For SMV'er er dette centralt, fordi de ofte mangler interne ESG-specialister og har begrænsede ressourcer til komplekse rapporteringsprocesser.\parencite{Virksomhedsguiden2025SMVDefinition}

Forretningsmodellen hviler på tre elementer: et digitalt produkt (dataindsamling og beregning), en compliance-service (validering og dokumentation) og et rapportoutput (PDF og standardiserede formater). Den reducerer måle- og rapporteringsfriktion, hvilket gør ESG til en operationaliserbar praksis snarere end et abstrakt krav. Det understøtter samtidig et strategisk narrativ om, at ESG-data kan bruges til risikostyring og værdiskabelse, i tråd med evidensen for positiv eller neutral sammenhæng mellem ESG og finansiel performance.\parencite{Eccles2014Impact,Friede2015Meta}

Alternativer til ESG-as-a-Service er typisk konsulentprojekter, interne ESG-teams eller generiske rapporteringsværktøjer. Konsulenter kan levere specialviden, men skaber ofte højere transaktionsomkostninger og mindre løbende standardisering. Interne løsninger giver kontrol, men kræver kapacitet og datafaglighed, som mange SMV'er ikke har. Generiske værktøjer kan være billige, men matcher sjældent sektorspecifikke databehov.

Samtidig peger \textcite{Khan2016Materiality} på, at ESG kun skaber finansiel værdi, når indsatsen er rettet mod materielle temaer. Det betyder, at ESG-as-a-Service skal prioritere data, der er relevante for sundhedssektorens kernerisici (energi, affald, arbejdsmiljø) og ikke blot dække alle standarder mekanisk. Værdiforslaget afhænger derfor af evnen til at koble standardkrav til sektorrelevante indikatorer og til at omsætte dem til beslutningsrelevante outputs.

En bæredygtig forretningsmodel forudsætter, at kunderne oplever et klart compliance- og risikobenefit, at data kan indsamles med begrænset friktion, og at onboarding og support kan skaleres uden at øge omkostningsbasen proportionalt.

\subsubsection{Fra governance til forretning: hvor grænsen går}
\label{subsec:fra-governance-til-forretning}

ESG-as-a-Service bevæger sig fra governance til forretning, når data ikke længere primært fungerer som dokumentation, men som aktivt beslutningsgrundlag for drift, investeringer og markedspositionering. I governance-laget er værdien knyttet til compliance, sporbarhed og risikoreduktion; i forretningslaget opstår værdi, når ESG-data bruges til at optimere omkostninger (fx energi og affald), differentiere produkter eller understøtte pris- og segmentstrategier. Det skift kan operationaliseres ved, at output går fra minimumsrapporter til KPI'er, der indgår i ledelsesrapportering og kundedialog. Grænsen mellem governance og forretning ligger derfor ikke i selve rapportformatet, men i om data anvendes aktivt til strategiske beslutninger og kommercielle prioriteringer.\parencite{PorterKramer2011CSV,ESGBook}

\subsection{Prisfastsættelse og kundesegmentering}
\label{subsec:prisfastsaettelse-og-kundesegmentering}

Segmenteringen bør styres af datakompleksitet og compliancepres snarere end alene virksomhedstype. I sundhedssektoren kan der skelnes mellem (1) private klinikker med begrænset datagrundlag, (2) mindre hospitaler med mere kompleks drift, og (3) medtech-leverandører med compliancepres fra forsyningskæden. Segmenterne adskiller sig på rapporteringsbehov, integrationskrav og betalingsvillighed.

Prisfastsættelsen er derfor differentieret efter kompleksitet og behov. Casen peger på en abonnementsmodel for standardmoduler, et projektgebyr for integration og opsætning, samt usage-gebyrer for eksport af specifikke rapportpakker (bilag \ref{app:teknisk-dokumentation}). Denne struktur afspejler, at en stor del af omkostningen ligger i initial opsætning og datakortlægning, mens løbende drift kan standardiseres.

Et simpelt prisudtryk kan beskrives som:
\begin{equation*}
  P = \frac{F}{N} + V + m
\end{equation*}
hvor $F$ er faste omkostninger, $N$ er antal kunder, $V$ er variable omkostninger pr. kunde, og $m$ er en risikomargin. Modellen synliggør, at skalerbarhed i onboarding og support er afgørende for at holde $P$ konkurrencedygtig.

Betalingsvilligheden er tæt knyttet til compliancepres og interessentkrav. Selvom mange SMV'er ikke er direkte CSRD-pligtige, møder de dataanmodninger fra kunder og finansielle aktører og vælger derfor frivillig rapportering som risikoreduktion.\parencite{PwC2025GuideESGSMV,Erhvervsstyrelsen2025StopClock} EY's analyse af VSME-rapporter indikerer, at mange virksomheder rapporterer ud over basismodullets krav, hvilket peger på en villighed til at investere i bedre datakvalitet.\parencite{EY2025VSME}

Analytisk betyder det, at prisstrategien bør balancere lav adgangsbarriere for mindre aktører med mulighed for opgradering for kunder med højere compliancekrav. En modulopbygget prisstruktur understøtter denne balance, fordi den knytter omkostninger direkte til datakompleksitet og rapporteringsomfang.

\begin{table}[t]
  \caption{Illustrativ prisdifferentiering, der afspejler, at compliancepres og datakompleksitet driver betalingsvillighed.}
  \label{tab:pris-segmenter}
  \begin{tabularx}{\textwidth}{l X r}
    \toprule
    Segment & Karakteristika & Indikativ pris pr. måned (DKK) \\
    \midrule
    Private klinikker & Basismodul, få datakilder, lav integrationsgrad. & \numrange{1500}{3000} \\
    Mindre hospitaler & Flere moduler, flere datakilder, højere kompleksitet. & \numrange{4000}{7000} \\
    Medtech-leverandører & Høj compliance, eksportkrav, dokumentationskrav fra kunder. & \numrange{7000}{12000} \\
    \bottomrule
  \end{tabularx}
  \TableSource{Egen fremstilling.}
  \TableNote{Prisintervallerne er illustrative og afhænger af datakompleksitet og serviceomfang.}
\end{table}

\subsection{Organisatoriske og økonomiske implikationer}
\label{subsec:organisatoriske-og-oekonomiske-implikationer}

Implementering er i praksis en governance-opgave: hvem ejer data, hvem godkender dem, og hvordan sikres sporbarhed. Standarder og regulering etablerer forventninger til dokumentation, hvilket betyder, at SMV'er må definere ansvar for dataindsamling, validering og rapportering.\parencite{EU2022CSRD,EU2023ESRS} Det skaber behov for en intern funktion eller et eksternt serviceled, der kan oversætte krav til operative beslutninger.\parencite{RoevikTrenderTranslasjoner,TranslationTheoryKnowledgeTransfer}

Ressourcebehovet optræder på flere niveauer: tid til dataindsamling, kompetencer til fortolkning af standarder og tekniske ressourcer til at integrere data. Løsningen kan reducere noget af dette ved at tilbyde standardiserede arbejdsgange og auditspor, men kræver stadig organisatorisk forankring for at undgå dekobling mellem rapportering og praksis.\parencite{BrunssonWorldOfStandards}

Omkostningsstrukturen er typisk fronttung med udgifter til datakortlægning, integration og oplæring, mens den løbende drift domineres af datavedligehold, rapportering og kontrol. Return on investment afhænger af, om virksomheden kan reducere manuelle processer, mindske compliance-risici og anvende ESG-data i beslutninger. Manglende datakvalitet øger revisionsrisikoen og kan gøre rapporteringen mindre brugbar, hvilket svækker både governance og den forventede værdi.

Økonomisk set handler implikationerne om forholdet mellem omkostninger og potentiel værdi. Metastudier peger på, at ESG-arbejde ofte er forbundet med neutral eller positiv finansiel performance, men effekten er betinget af fokus på materielle temaer.\parencite{Friede2015Meta,Khan2016Materiality} For SMV'er betyder det, at ressourcer bør prioriteres til data og indikatorer, der er direkte relevante for sundhedssektorens risici og interessentkrav.

I analysen betyder dette, at servicekonceptet skal vurderes på sin evne til at reducere faste compliance-omkostninger og skabe beslutningsrelevante data, snarere end blot at levere rapporter. Governance-implikationen er, at organisationen må etablere en stabil rapporteringsrutine, hvor dataindsamling bliver en integreret del af driften.

\subsection{Implementering i sundhedssektoren}
\label{subsec:implementering-i-sundhedssektoren}



\section{Diskussion}
\label{sec:diskussion}

\begin{quote}
Diskussionen afvejer evidens, begrænsninger og implikationer for, hvad der faktisk kan konkluderes.
\end{quote}

Diskussionen tager afsæt i tre hovedfund: (1) reguleringen skaber et vedvarende behov for standardiserede data og sporbarhed, (2) servicekonceptet og MVP'en kan operationalisere centrale krav, og (3) organisatorisk forankring er afgørende for at undgå symbolsk rapportering. På den baggrund vurderes, hvorvidt løsningen faktisk understøtter de teoretiske antagelser om standardisering, governance og værdiskabelse.

Evidensgrundlaget er stærkt for de regulatoriske krav og de teoretiske perspektiver, men mere begrænset empirisk, fordi sektordata er aggregerede og casen er single-case. Det betyder, at alternative forklaringer og implikationer for SMV'er må vurderes kritisk, og at konklusionerne bør forstås som analytiske snarere end generaliserbare.

I \ref{subsec:sammenhaeng-mellem-teori-empiri-og-software} diskuteres sammenhængen mellem teori, empiri og software. \ref{subsec:implikationer-for-smver} behandler konsekvenser og handlemuligheder for SMV'er, mens \ref{subsec:begraensninger-og-alternative-forklaringer} redegør for metodiske begrænsninger og mulige alternative forklaringer. Diskussionen nuancerer dermed svarene på forskningsspørgsmålene ved at afveje evidensstyrke, praktiske implikationer og usikkerheder.

Diskussionens afvejninger leder frem mod konklusionens endelige svar og prioritering af bidrag.

\subsection{Sammenhæng mellem teori, empiri og software}
\label{subsec:sammenhaeng-mellem-teori-empiri-og-software}

Analysen peger på en grundlæggende konsistens mellem teori, empiri og MVP. Brunssons standardiseringsperspektiv beskriver, hvordan standarder skaber legitimitet og sammenlignelighed, men også risiko for dekobling mellem rapportering og praksis.\parencite{BrunssonWorldOfStandards} MVP'en adresserer denne risiko ved at indbygge auditspor, validering og beregningsspor, som gør data efterprøvbare og reducerer symbolsk rapportering (bilag \ref{app:teknisk-dokumentation}). Det understøtter den teoretiske antagelse om, at standardisering kræver konkrete infrastrukturer for at få organisatorisk effekt.

Pensum om translasjon understreger, at ideer altid oversættes til lokale praksisser.\parencite{RoevikTrenderTranslasjoner,TranslationTheoryKnowledgeTransfer} Det genfindes i casen, hvor MVP'en prioriterer et begrænset sæt af indikatorer (energi, affald og sociale KPI'er) for at gøre rapporteringen gennemførbar for SMV'er. Dette bekræfter, at implementering ikke er en direkte kopi af ESRS/GRI, men en kontekstualiseret oversættelse.

Samtidig peger empirien på afvigelser, hvor implementering i praksis ofte drives af kundekrav og ressourcebegrænsninger snarere end af standardlogik alene. Det kan betyde, at rapporteringen bliver mere compliance-orienteret end den teoretiske ambition om strategisk værdiskabelse.

Empirien om sundhedssektorens klimaaftryk og arbejdsvilkår viser, at centrale ESG-dimensioner ligger i scope~3, affaldsstrømme og sociale risici.\parencite{HCWH2019ClimateFootprint,WHO2024HealthCareWaste,WHO2021HealthWorkerDeaths} MVP'ens fokus på sporbar dataindsamling og strukturerede outputs matcher disse behov og fungerer som et praktisk svar på sektorens dokumentationskrav.\parencite{Sepetis2024Healthcare,Bosco2024ESGHealth}

Samtidig viser pensum om ESG, at mangel på standardisering og datakvalitet er en tilbagevendende barriere for værdiskabelse.\parencite{ESGBook} ESG-ratinger divergerer og understreger, at standarder endnu ikke skaber fuld sammenlignelighed på tværs af aktører.\parencite{Berg2022AggregateConfusion} Det betyder, at MVP'en bør ses som et skridt mod standardisering, men ikke som en garanti for fuld konsensus eller endelig validitet. Diskussionen peger derfor på, at software kan reducere friktion og skabe bedre datakvalitet, men at organisatorisk forankring og governance stadig er afgørende for at undgå dekobling.

\subsubsection{Minimum compliance versus beslutningsrelevant ESG}
\label{subsec:minimum-compliance-vs-beslutningsrelevant-esg}

Den analytiske spændvidde i ESG-as-a-Service kan tydeliggøres ved at skelne mellem minimum compliance og beslutningsrelevant ESG. Distinktionen er central, fordi den markerer, hvornår rapportering bliver styringsinformation frem for dokumentation. Tabel \ref{tab:compliance-vs-beslutningsrelevant} opsummerer forskelle i formål, datakrav og organisatorisk anvendelse.
\begin{table}[t]
  \caption{Kontrast, der viser hvorfor beslutningsrelevant ESG kræver mere end minimum compliance.}
  \label{tab:compliance-vs-beslutningsrelevant}
  \begin{tabularx}{\textwidth}{l X X}
    \toprule
    Dimension & Minimum compliance & Beslutningsrelevant ESG \\
    \midrule
    Formål & Opfylde eksterne krav og dokumentation. & Understøtte strategiske beslutninger og værdiskabelse. \\
    Datakrav & Minimumsdatapunkter og lav granuleringsgrad. & Højere granularitet, sporbarhed og sammenhæng til KPI'er. \\
    Proces & Periodisk rapportering med fokus på kontrol. & Løbende styring og integration i driftsprocesser. \\
    Output & Rapport og dokumentation til myndigheder/kunder. & Ledelsesinformation, risikostyring og forbedringstiltag. \\
    \bottomrule
  \end{tabularx}
  \TableSource{Egen fremstilling baseret på teori og case.}
\end{table}

Normativt positioneres rapporten således: Minimum compliance er utilstrækkelig governance i sundhedssektoren, når ESG-data bruges til risikostyring, prioritering og ansvarlighed. ``God nok'' ESG er derfor dårlig governance, når den reducerer ESG til efterlevelse uden læring, prioritering eller beslutningsrelevans.

Troværdig ESG-rapportering kræver derfor både tekniske kontroller og organisatoriske rutiner, der sikrer, at data faktisk afspejler praksis og ikke kun formel efterlevelse.

\subsection{Implikationer for SMV'er}
\label{subsec:implikationer-for-smver}

Ifølge \textcite{PwC2025GuideESGSMV} bliver ESG-rapportering et vedvarende krav for SMV'er, selv når virksomhederne ikke er direkte omfattet af CSRD. Kravene kommer i stigende grad via kunder, banker og forsyningskæder, hvilket gør frivillig rapportering til en praktisk nødvendighed.\parencite{Virksomhedsguiden2025SMVDefinition}

VSME-standarden tilbyder en lavere adgangsbarriere og reducerer kompleksiteten ved at opstille et basismodul med begrænset datakrav og uden krav om dobbelt væsentlighed.\parencite{Virksomhedsguiden2025VSMEIntro,Virksomhedsguiden2025VSMEModules,Virksomhedsguiden2025NoDualMateriality} EY's evaluering viser samtidig, at mange virksomheder vælger at rapportere ud over minimumsniveauet, hvilket indikerer en strategisk anvendelse af ESG-data snarere end ren compliance.\parencite{EY2025VSME}

VSME kan være tilstrækkelig som minimumsniveau, men den kan også vise sig utilstrækkelig, når kunder, banker eller offentlige aktører kræver mere detaljeret dokumentation. Det kan skabe dobbeltarbejde, hvis SMV'er både skal opfylde VSME og mere avancerede krav, og dermed øge byrden frem for at reducere den.

Implikationen for SMV'er er derfor todelt. På kort sigt er der behov for forenklede arbejdsgange og teknisk støtte, som ESG-as-a-Service kan levere. På lang sigt er der behov for organisatorisk læring og datakapacitet, så ESG-arbejdet kan integreres i driften og skabe beslutningsrelevante indsigter. Stop-the-clock-aftalen giver ekstra tid, men nedsætter ikke markedets efterspørgsel efter dokumentation, hvilket betyder, at udsættelse ikke bør tolkes som en pause i implementeringen.\parencite{Erhvervsstyrelsen2025StopClock}

Samlet set understøtter diskussionen, at SMV'er bør anvende standarder og digitale løsninger som en gradvis overgangsstrategi: start med minimumsmoduler, dokumenter sporbarhed, og udvid derefter til mere komplekse krav, når datakvalitet og ressourcer tillader det.

\subsection{Begrænsninger og alternative forklaringer}
\label{subsec:begraensninger-og-alternative-forklaringer}

Diskussionens begrænsninger afspejler metodens karakter som casebaseret og konceptuel analyse. Casen er egenproduceret, og MVP'en er ikke afprøvet i flere organisationer. Det betyder, at konklusionerne primært er analytiske og bør valideres i fremtidige empiriske studier.

Der findes flere alternative forklaringer på de observerede sammenhænge. For det første kan motivationen for ESG-rapportering være drevet af generel digitaliseringsmodenhed snarere end af standardiseringslogikken i sig selv. For det andet kan implementering handle mere om efterlevelse af kundekrav i forsyningskæden end om strategisk værdiskabelse, hvilket reducerer muligheden for at udlede økonomiske effekter af ESG-arbejdet.\parencite{Berg2022AggregateConfusion}

En metodisk begrænsning er antagelserne i MVP'en om datatilgængelighed og standardiserede inputformater. Hvis data er mere fragmenterede end antaget, kan beregninger og rapportoutput blive mindre valide, hvilket påvirker konklusionernes styrke.

Regulatorisk usikkerhed er en tredje forklaring. CSRD, ESRS og EU-taksonomien er under løbende justering, og stop-the-clock-aftalen kan skabe midlertidige tilpasninger i virksomhedernes prioriteringer.\parencite{EU2022CSRD,EU2023ESRS,EU2020Taxonomy,Erhvervsstyrelsen2025StopClock} Effekter, der i analysen tolkes som implementeringsbarrierer, kan derfor delvist skyldes timing og regulatorisk bevægelse.

Endelig kan dataforskelle og rating-divergens betyde, at selv standardiseret rapportering ikke skaber entydige evalueringer på tværs af interessenter.\parencite{Berg2022AggregateConfusion} Dette understreger, at diskussionens konklusioner skal tolkes med forbehold for datakvalitet, fortolkningsrum og aktørers forskellige anvendelse af ESG-information. Yderligere empirisk test bør derfor omfatte flere cases, kvantitative data og brugeradoption over tid.


\section{Konklusion}
\label{sec:konklusion}

Analysen har belyst ESG-rapportering i sundhedssektoren og vurderet, hvordan et ESG-as-a-Service-koncept og en MVP kan omsætte regulatoriske krav til praksis. Arbejdet er forankret i standardiserings- og governanceperspektiver og suppleret af empiriske sektordata og casebaseret evidens.

Forskningsspørgsmålene kan besvares således:
\begin{enumerate}
  \item CSRD, ESRS, EU-taksonomien og GRI etablerer standardiserede krav til data, væsentlighed og dokumentation. For SMV'er sker påvirkningen primært indirekte via forsyningskæder og finansielle aktører, mens VSME giver et forenklet minimumsniveau.\parencite{EU2022CSRD,EU2023ESRS,EU2020Taxonomy,GRI2021,Virksomhedsguiden2025VSMEIntro,Virksomhedsguiden2025VSMEModules,Virksomhedsguiden2025NoDualMateriality}
  \item ESG-as-a-Service skaber værdiforslag ved at oversætte standarder til konkrete arbejdsgange, reducere transaktionsomkostninger og styrke sporbarhed og governance. Værdien afhænger af fokus på materielle temaer og af organisatorisk forankring, så rapportering ikke reduceres til symbolsk compliance.\parencite{BrunssonWorldOfStandards,RoevikTrenderTranslasjoner,TranslationTheoryKnowledgeTransfer,Khan2016Materiality}
  \item MVP'en kan omsætte krav til praksis gennem modulbaseret dataindsamling, validering, beregning og rapportoutput kombineret med auditspor. Den demonstrerer gennemførbarhed for SMV'er, men også behov for videre integration og udvidet dækning (bilag \ref{app:teknisk-dokumentation}).
\end{enumerate}

Samlet set viser analysen, at ESG-rapportering i sundhedssektoren kræver standardiseret datahåndtering og governance, og at ESG-as-a-Service kan fungere som en praktisk bro mellem regulering og drift. Konklusionerne er dog begrænset af case- og datagrundlag og bør valideres gennem flere empiriske studier.


\section{Perspektivering og anbefalinger}
\label{sec:perspektivering-og-anbefalinger}



\printbibliography[heading=bibintoc,title={Referencer}]

\appendix
\section{Bilag}
\label{app:bilag}

Bilagene understøtter sporbarhed og dokumentation for rapportens metode, software og projektkontekst. Bilag \ref{app:projektbeskrivelse} indeholder den formelle projektbeskrivelse. Bilag \ref{app:supplerende-figurer-og-tabeller} samler supplerende figurer og tabeller, som refereres fra hovedteksten, og bilag \ref{app:teknisk-dokumentation} uddyber den tekniske dokumentation.

\subsection{Projektbeskrivelse}
\label{app:projektbeskrivelse}

Bilaget relaterer til problemformuleringen i \ref{subsec:problemformulering} ved at præcisere kursens fokus på ESG-rapportering som serviceydelse og de organisatoriske og økonomiske implikationer, som rapporten analyserer.

\textbf{Kursets titel:} ESG-rapportering i sundhedssektoren: Policy, organisering, økonomisk relevans og kommercialisering.

\textbf{English title:} ESG reporting in the healthcare sector: Policy, organization, economic relevance, and commercialization.

\textbf{Kursets faglige indhold:} Kurset undersøger ESG-rapporteringens rolle og betydning i sundhedssektoren, og hvordan rapporteringen kan udvikles til en selvstændig serviceydelse (ESG-as-a-Service) i en startup-kontekst. Der lægges vægt på:
\begin{itemize}
  \item Relevans, lovkrav og strategiske gevinster ved ESG-rapportering for sundhedsorganisationer og deres leverandørkæder.
  \item Nationale og EU-rammer (bl.a. CSRD, EU-taksonomien, GRI-standarderne).
  \item Udvikling af forretningsmodeller for ESG-rapportering som service, herunder prisfastsættelse, værdiforslag og kundesegmentering.
  \item Digitalisering og automatisering af ESG-dataindsamling og -rapportering (softwareplatforme og data-API'er).
  \item Casestudie: Den studerendes egen virksomhed som pilot for udvikling af en ESG-rapporteringstjeneste til eksterne kunder.
\end{itemize}

Kurset kulminerer i en skriftlig rapport, der kombinerer teori, praksis, analyse og konkrete anbefalinger til både sundhedssektoren og den studerendes virksomhed.

\textbf{Kursets formål:} At give den studerende en sundhedsvidenskabeligt forankret forståelse af ESG-rapportering og at kunne analysere, vurdere og kommercialisere dens organisatoriske, politiske og økonomiske implikationer. Dette gælder både internt i sundhedsorganisationer og som ekstern serviceydelse leveret af den studerendes egen startup.

\textbf{Kursets omfang i ECTS:} 15 ECTS. Cirka 375 timer total (25 timer pr. ECTS), cirka 275 timers forberedelse/læsning, cirka 100 timer på skriftlig aflevering.

\textbf{Supplerende målbeskrivelse:}

\textbf{Viden:}
\begin{itemize}
  \item Redegøre for ESG-rapporteringens betydning i en sundhedssektor-kontekst.
  \item Forklare centrale lovrammer (CSRD, GRI, EU-taksonomien) og markedsforventninger.
  \item Beskrive forretningsmæssige aspekter ved at tilbyde ESG-rapportering som service.
  \item Anvende Brunsson og andre teoretikere til at forstå ESG som eksempel på en organisationsstandard.
\end{itemize}

\textbf{Færdigheder:}
\begin{itemize}
  \item Analysere sundhedspolitiske, organisatoriske og økonomiske implikationer af ESG-rapportering.
  \item Udarbejde et koncept og en MVP for en ESG-rapporteringstjeneste målrettet sundhedssektorens aktører.
  \item Redegøre for digitale værktøjer til dataindsamling, analyse og visualisering af ESG-nøgletal.
\end{itemize}

\textbf{Kompetencer:}
\begin{itemize}
  \item Gennemføre en tværfaglig analyse i spændingsfeltet mellem sundhedsvidenskab, bæredygtighed og forretningsudvikling.
  \item Skabe en selvstændig, markedsorienteret service baseret på ESG-rapportering og implementere denne i egen virksomhed.
  \item Reflektere kritisk over etiske, compliance-mæssige og økonomiske aspekter ved ESG-data og -rapportering.
  \item Reflektere kritisk over ESG som organisationsstandard, herunder styringsmæssige og institutionelle implikationer.
\end{itemize}

\textbf{Form for produkt:} Skriftlig rapport (maks. 40 sider).

\textbf{Supplerende karakterbeskrivelse:} Et bestået projekt demonstrerer:
\begin{itemize}
  \item Klar struktur, metodisk stringens og faglig dybde.
  \item Velunderbygget forretningsmodel for ESG-rapportering som service.
  \item Selvstændig vurdering af ESG's betydning for sundhedsorganisationer samt startup-markedsmuligheder.
  \item Refleksion over økonomisk, politisk og etisk kontekst.
\end{itemize}

\textbf{Sprog:} Dansk.

\textbf{Prøveform:} Skriftlig rapport.

\textbf{Beståelsesform:} Bestået / Ikke bestået.

\par{\small\textit{Kilde: biblen-projektbeskrivelse.txt}}

\subsection{Supplerende figurer og tabeller}
\label{app:supplerende-figurer-og-tabeller}

Dette bilag samler supplerende materiale, der refereres fra hovedteksten, men som ikke er nødvendigt at placere i de centrale kapitler. Moduloversigten i tabel \ref{tab:app-moduloversigt} og skærmbillederne dokumenterer den fulde brugerrejse fra profilafgrænsning til rapportoutput.

\begin{table}[t]
  \caption{Moduloversigt pr. scope i MVP'en.}
  \label{tab:app-moduloversigt}
  \begin{tabularx}{\textwidth}{l X}
    \toprule
    Område & Moduler (oversigt) \\
    \midrule
    Scope 1 & A1--A4 (direkte emissioner og processer). \\
    Scope 2 & B1--B11 (energi og varme). \\
    Scope 3 & C1--C15 (værdikæde og affald). \\
    Socialt & S1--S4 (arbejdsmiljø og sociale forhold). \\
    Governance og strategisk & G1, D1--D2, SBM, IRO, MR. \\
    \bottomrule
  \end{tabularx}
  \TableSource{Egen fremstilling baseret på case-materiale.}
\end{table}

\paragraph{Profil og modulvalg} Profiloverblik og profil-flow vises i figurer \ref{fig:app-landing-profil-overblik}, \ref{fig:app-profil-flow-start}, \ref{fig:app-profil-flow-ja}, \ref{fig:app-profil-flow-nej} og \ref{fig:app-profil-flow-progression}. Moduloverblik og navigation er dokumenteret i figurer \ref{fig:app-modul-overblik} og \ref{fig:app-modul-navigation}.

\begin{figure}[t]
  \centering
  \includegraphics[width=\textwidth]{\detokenize{billeder af software/Landing – profil‑overblik.png}}
  \caption{Profiloverblik som startpunkt for brugerrejsen.}
  \label{fig:app-landing-profil-overblik}
  \FigureSource{Egen fremstilling (skærmbillede fra MVP).}
\end{figure}

\begin{figure}[t]
  \centering
  \includegraphics[width=\textwidth]{\detokenize{billeder af software/Profil‑flow – startskærm.png}}
  \caption{Profil-flowets startskærm med formål og progression.}
  \label{fig:app-profil-flow-start}
  \FigureSource{Egen fremstilling (skærmbillede fra MVP).}
\end{figure}

\begin{figure}[t]
  \centering
  \includegraphics[width=\textwidth]{\detokenize{billeder af software/Profil‑flow – svar “Ja” .png}}
  \caption{Profil-flow med svar ``Ja'' og opdateret progression.}
  \label{fig:app-profil-flow-ja}
  \FigureSource{Egen fremstilling (skærmbillede fra MVP).}
\end{figure}

\begin{figure}[t]
  \centering
  \includegraphics[width=\textwidth]{\detokenize{billeder af software/Profil‑flow – svar “Nej” .png}}
  \caption{Profil-flow med svar ``Nej'' og opdateret progression.}
  \label{fig:app-profil-flow-nej}
  \FigureSource{Egen fremstilling (skærmbillede fra MVP).}
\end{figure}

\begin{figure}[t]
  \centering
  \includegraphics[width=\textwidth]{\detokenize{billeder af software/Profil‑flow – progression.png}}
  \caption{Profil-flow efter flere svar med progression.}
  \label{fig:app-profil-flow-progression}
  \FigureSource{Egen fremstilling (skærmbillede fra MVP).}
\end{figure}

\begin{figure}[t]
  \centering
  \includegraphics[width=\textwidth]{\detokenize{billeder af software/Wizard – overblik før moduler.png}}
  \caption{Moduloverblik med adgang til moduler.}
  \label{fig:app-modul-overblik}
  \FigureSource{Egen fremstilling (skærmbillede fra MVP).}
\end{figure}

\begin{figure}[t]
  \centering
  \includegraphics[width=\textwidth]{\detokenize{billeder af software/Modul‑navigation – overblik .png}}
  \caption{Modulnavigation opdelt efter scope.}
  \label{fig:app-modul-navigation}
  \FigureSource{Egen fremstilling (skærmbillede fra MVP).}
\end{figure}

\paragraph{Dataindtastning og beregning} B2-modulet illustrerer dataindsamling, beregning og sporbarhed i figurer \ref{fig:app-b2-tomt}, \ref{fig:app-b2-udfyldt}, \ref{fig:app-b2-resultat} og \ref{fig:app-b2-trace}.

\begin{figure}[t]
  \centering
  \includegraphics[width=\textwidth]{\detokenize{billeder af software/B2 – tomt:ikke udfyldt.png}}
  \caption{B2-modul uden udfyldte felter.}
  \label{fig:app-b2-tomt}
  \FigureSource{Egen fremstilling (skærmbillede fra MVP).}
\end{figure}

\begin{figure}[t]
  \centering
  \includegraphics[width=\textwidth]{\detokenize{billeder af software/B2 – udfyldt formular.png}}
  \caption{B2-modul med udfyldt energi-input.}
  \label{fig:app-b2-udfyldt}
  \FigureSource{Egen fremstilling (skærmbillede fra MVP).}
\end{figure}

\begin{figure}[t]
  \centering
  \includegraphics[width=\textwidth]{\detokenize{billeder af software/B2 – CO₂‑estimat i UI.png}}
  \caption{B2-modul med beregnet CO$_2$-estimat.}
  \label{fig:app-b2-resultat}
  \FigureSource{Egen fremstilling (skærmbillede fra MVP).}
\end{figure}

\begin{figure}[t]
  \centering
  \includegraphics[width=\textwidth]{\detokenize{billeder af software/B2 – teknisk beregningstrace.png}}
  \caption{Beregningstrace for CO$_2$-estimat.}
  \label{fig:app-b2-trace}
  \FigureSource{Egen fremstilling (skærmbillede fra MVP).}
\end{figure}

\paragraph{Review og eksport} Review og eksport vises i figurer \ref{fig:app-review-topoverblik}, \ref{fig:app-reportoutput-preview} og \ref{fig:app-review-download}.

\begin{figure}[t]
  \centering
  \includegraphics[width=\textwidth]{\detokenize{billeder af software/Review – topoverblik.png}}
  \caption{Review-side med status og eksportmuligheder.}
  \label{fig:app-review-topoverblik}
  \FigureSource{Egen fremstilling (skærmbillede fra MVP).}
\end{figure}

\begin{figure}[t]
  \centering
  \includegraphics[width=\textwidth]{\detokenize{billeder af software/Review – PDF‑preview:sektion.png}}
  \caption{PDF-preview som del af rapportoutput.}
  \label{fig:app-reportoutput-preview}
  \FigureSource{Egen fremstilling (skærmbillede fra MVP).}
\end{figure}

\begin{figure}[t]
  \centering
  \includegraphics[width=\textwidth]{\detokenize{billeder af software/Review – download‑knapper .png}}
  \caption{Download-knapper til rapportoutput.}
  \label{fig:app-review-download}
  \FigureSource{Egen fremstilling (skærmbillede fra MVP).}
\end{figure}

\subsection{Teknisk dokumentation}
\label{app:teknisk-dokumentation}

Dette bilag sammenfatter det interne case- og softwaremateriale, der ligger til grund for beskrivelsen af MVP'en i kapitel \ref{sec:software-og-mvp}. Materialet er ikke offentligt tilgængeligt og er derfor gengivet her for sporbarhed.

\paragraph{Systemstruktur} MVP'en er en webbaseret løsning med adskilt brugergrænseflade, applikationslogik og datalagring. Strukturen understøtter et kontrolleret flow fra input til rapportoutput og gør det muligt at udvide komponenter uden at bryde den samlede proceslogik.

\paragraph{Datamodel og sporbarhed} Data organiseres omkring profilafgrænsning, modulregistreringer og et auditspor. Modellen sikrer, at ændringer, perioder og beregningsgrundlag kan efterprøves, og at data kan følges tilbage til deres kilde.

\paragraph{Datakvalitet og kontrol} Input valideres mod faste regler, enheder og obligatoriske felter. Versionshistorik gør det muligt at dokumentere udvikling over tid og at skelne mellem primære input og beregnede resultater.

\paragraph{Rapportoutput} Output omfatter et læsbart dokument og strukturerede data, så resultater kan genbruges i compliance- og beslutningsprocesser på tværs af interessenter.

\paragraph{Antagelser og begrænsninger} Bilaget beskriver MVP'ens aktuelle scope og forudsætter tilgængelige datakilder i konsistente formater. Det dækker ikke fuld drifts- eller sikkerhedsarkitektur, som typisk specificeres i en senere produktionsfase.

\subsubsection{Software-specifikation (intern)}
\label{app:software-spec}

\VerbatimInput[fontsize=\footnotesize,breaklines=true]{softwarebeskrivelser/software-spec.md}

\subsubsection{Detaljeret software-specifikation (intern)}
\label{app:detaljeret-software-spec}

\VerbatimInput[fontsize=\footnotesize,breaklines=true]{softwarebeskrivelser/detaljeret-software-spec.txt}


\end{document}
